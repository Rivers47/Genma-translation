\chapter[为了改良而不立直的场合]{追求改良时的五个切牌法则}
\section{改良的法则}
\subsection{优先当前的改良,重视改良的质量而非数量}\label{lec3:改良法则1}
如\hyperref[lec1:做牌法则1]{做牌法则1}中所述,与其追求改良,不如优先考虑进张数。而在比较改良时,应优先当前的改良,而不是数手之后才可能实现的改良。在当前的改良中,和\hyperref[lec1:做牌法则3]{做牌法则3·4}一致,应优先考虑能提升胡牌概率或向高得点改良的选项,而非仅仅追求改良的数量。

不过,如\hyperref[lec1:做牌法则5]{做牌法则5}中提到的情况,若是能够形成两面搭子以上的听牌,或提升到满贯以上的得点,通常优先考虑改良的数量。同时,这种情况下如果选择暂不立直而等待改良,则需注意若改良的可能性不足则寻求改良没有优势。

\subsection{如果存在特别高价值的浮牌,应优先考虑改良}\label{lec3:改良法则2}
在听牌前阶段,经常讨论的是边搭与37浮牌的取舍。关于这一问题,结论是通常优先进张而保留边搭,,整体更稳妥,但两者的差别并不显著。因此,如果某张37的浮牌明显具有特别高的价值,那么应该优先保留该浮牌,而舍弃较差的愚形塔子。

\subsection{如果有价值特别低的搭子,则优先考虑改良}
如果某愚形塔子的价值特别低,那么如\hyperref[lec1:做牌法则2]{做牌法则2} 所示,基本上应优先保留3-7浮牌。

\subsection{如果选择改良,应尽可能追求最大化}\label{lec3:改良法则4}
当目标是进行改良时,与其同时兼顾部分进张,不如即便暂时降低向听数,也应尽可能追求手牌的最大化改良(单骑听牌情况除外)。如果选择重视进张数,则应完全优先进张,而不是采取折中的方式,因为这种中途策略成功的情况较少。

\subsection{向听数越高的阶段越应重视改良}
在距离和牌较远的阶段,更倾向于追求手牌的变化。此时,即便是降低向听数的变动,其损失也相对较少,同时还有更多可以用于手变的浮牌。因此,尽可能追求手牌的变化,往往会更有利于后续的发展。

\section{关于等待改良的分类}
在判断是否等待手牌改良时,逐一数出改良的进张数既麻烦又耗时。因此,通过对手牌模式进行分类,可以在观察手牌形状时,快速判断是否适合等待改良(在巡目接近中盘结束或之后,末盘除非是为了追求更高的打点或规避风险,否则应直接选择立直)。

\subsection{附加型一向听(搭子不足的一向听)}

\textbf{在序盘阶段,等待改良的基准是6种可以提升1番的改良或9种能够变成好形的改良}
即使放弃听牌形成两个四连形的一向听,改良成两面的可能性只有8种,无法满足序盘阶段改良等待的条件。不过,如果通过两面听牌可以形成平和,则此类情况等待改良会具有优势。\hyperref[lec3:pai1-5]{牌1-牌2}

\begin{figure}
    \caption{浮牌型一向听} \label{lec3:pai1-5}
    \tekiririichi{1}{45679m2340p11122s}{4z}{2p}{没有平和改良所以即立}
    \par\bigskip
    \tekiri{2}{45679m2340p22789s}{4z}{9m}{能改良为两面听会有平和所以选择等待改良}
    \par\bigskip
    \tekiri{3}{12399m35678p0789s}{9s}{3p}{红5是浮牌的情况下则等待改良}
    \par\bigskip
    \tekiririichi{4}{67m123566778p155z}{9s}{1z}{浮牌较弱}
    \par\bigskip
    \tekiri{5}{67m1235667789p55z}{9s}{6m}{有较强的浮牌}
\end{figure}

通过四连形,中膨形\footnote{顺子中间的牌重复,好像顺子中间膨胀了}或附加型\footnote{指3面子1雀头凑齐时两个孤张等待凑成搭子的形状}等形状,将手牌变成包含两张特别有价值的浮牌(能够使打点倍增)的形状,是判断改良等待是否有利的一个重要基准\hyperref[lec3:改良法则2]{改良法则2}。

如果立直后的手牌仅为两番或以下,则即便改良的可能性稍低于基准,只要存在良形改良+1番提升或直接提升两番的改良,也可以选择等待。此外,如果附加型的改良可以形成染手听牌,那么即便仅有一张中张浮牌,也会选择等待改良\hyperref[lec3:改良法则1]{改良法则1}。\hyperref[lec3:pai1-5]{牌3-牌5}


\subsection{听牌后默听(单骑除外)}

\begin{figure}
    \caption{听牌后默听} \label{lec3:pai6-8}
    \tekiri{6}{12334566m66p3457s}{2m}{7s}{默听。改良面广且打点也会上升}
    \par\bigskip
    \tekiri{7}{46m234566p123456s}{4s}{2p}{默听。除了\raisebox{-2pt}{\mahjong[3ex]{37m}}还有\raisebox{-2pt}{\mahjong[3ex]{34p47s}}的改良}
    \par\bigskip
    \tekiri{8}{1377m345p345567s5z}{7s}{5z}{序盘默听。如果摸到中张的浮牌则打\raisebox{-2pt}{\mahjong[3ex]{1m}}}
\end{figure}

对于双碰听牌,由于能够在保持听牌的同时观察两组对子各自的改良,因此通常作为\hyperref[lec3:改良法则4]{改良法则4}的例外处理,选择默听会在改良方面更有优势。有时,即便未转变为两面听,也可能出现一些例如役牌附加、面子的滑动\footnote{234进5切3变为345等}提升打点等改良,而这些改良在选择拆牌后便可能失去其高价值。\hyperref[lec3:pai6-8]{牌6}

另一方面,仅在极少数情况下才会选择保持嵌张形听牌来等待改良。只有通过面子滑动能显著提升打点;或是序盘阶段有大量浮牌可以使改良形成更有利的一向听两种情况。\hyperref[lec3:pai6-8]{牌7-8}

\subsection{无雀头一向听}
\begin{figure}
    \caption{无雀头一向听} \label{lec3:pai9-11}
    \tekiri{9}{123340567m34p345s}{5s}{3m}{\parbox{0.56\textwidth}{打点上比单骑听牌更好。与打\raisebox{-2pt}{\mahjong[3ex]{1m}}相比,即使没有断幺也有满贯的情况下选择更容易和牌打\raisebox{-2pt}{\mahjong[3ex]{3m}}(如果没有宝牌则打\raisebox{-2pt}{\mahjong[3ex]{1m}})}}
    \par\bigskip
    \tekiri{10}{440m23345p456888s}{4z}{4m}{确保红5并且确定断幺}
    \par\bigskip
    \tekiri{11}{456m5566667p67s44z}{1m}{4z}{饼子的改良有\explainmahjong{4567p}}
\end{figure}

在追求改良时,通常会选择变为单骑听牌,或者舍弃一个搭子以形成附加型的一向听。然而,存在比如雀头非常容易形成,并且如果通过其他手段形成雀头,可以显著提升打点的一些例外情况,即舍弃雀头并转变为无雀头的一向听。\hyperref[lec3:pai9-11]{牌9-11}

\subsection{普通形的单骑听牌}
如果单骑的听牌并不容易和牌,那么即便排除振听的牌,多数情况下也满足改良的条件。然而,若单骑听的是较容易和的牌(例如,3枚无筋19单骑),即使转变为两单骑(如1234),和牌率也不会大幅上升。在这种情况下,如果打点变化也有限,则可以直接选择立直。

\subsection{七对子听牌}
对于1289的数牌或字牌单骑以及筋牌,建议直接立直。
对于无筋37或单筋456,在序盘阶段选择默听等待改良,中盘以后则直接立直。
对于双无筋456,原则上选择默听

\section{难以决定立直与否的微妙情况}
\begin{figure}
    \caption{有役改良丰富且打点足够的手牌} \label{lec3:pai12-13}
    \tekiri{12}{24567m123p788s555z}{8s}{7s}{默听。有役且改良丰富}
    \par\bigskip
    \tekiri{13}{2233m114468p1155s}{1s}{6p}{即使默听打点也足够高}
\end{figure}

\begin{figure}
    \caption{考虑到放铳的风险难以决定立直的手牌} \label{lec3:pai14-15}
    \tekiri{14}{12377m789p234557s}{3z}{7s}{比起打\explainmahjong{5s}坎听,选择双碰的改良面更广}
    \par\bigskip
    \tekiri{15}{24m345678p110999s}{4z}{2m}{不会切红5的孤张}
\end{figure}
当默听的打点已经足够,或需要考虑回避放铳风险时,即使没有明显的改良机会,不立直可能会更有利。

愚形默听的40符3翻及以上的手牌,即使除开改良可能性来看立直更有利,如果搭子中包含像24567这样的四连形,或者是较差的单骑听牌,有较多良形变换的可能性时,可以选择默听(例如,类似七对子+两枚红宝牌的默听25符4翻也是如此)。\hyperref[lec3:pai12-13]{牌12-13}

中盘以后,对于以下情形也需考虑是否立直:听牌率20\%以上但不足30\%的情况下,立直后2翻的手牌,或是听牌率30\%以上但不足40\%的情况下,立直后Nomi的手牌,只要有一定程度的改良面就会选择默听。此外,即使手牌中只有一张较高价值的浮牌,也可以选择舍弃听牌,追求改良。\hyperref[lec3:pai14-15]{牌14-15}

