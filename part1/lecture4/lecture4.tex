\chapter[听牌的选择]{需要从复合搭子选择听牌形状时的判断}

\section{打点相同时的听牌选择}

首先需要明确的是,\textbf{双碰比坎张更有优势}。当两种听牌形的剩余牌张数相同时,大多数情况下双碰听牌的和牌率会高于坎张听牌。只有在类似无筋坎张2和非筋的3~7之间形成的双碰对比,或无筋坎张3和无筋456之间形成的双碰对比时,坎张才可能稍占优势。除此之外,基本都是双碰占优。

从自己的弃牌来看,对于无法形成两面听的牌,对手往往会认为它比较安全,因此即使该牌剩余张数不多,也更容易和牌。除了「筋」以外,不会形成两面听的牌还包括:字牌以及No-Chance牌:即已经看到4张牌、从而确定不会是两面听搭子的那种牌。这种听牌在俗称上也被称作“好听”。

另外,如果河里已经有人切过赤5,那么一般认为其手牌里不会再持有其他5牌,这时3和7就可视同No-Chance牌来对待。再进一步,如果1或9也被切出,则相应的4和6也可当作No-Chance来看。

不过需要注意,如果其他家也已经听牌,并且他们看上去不太会打自己的听牌的话,那么“因为对手认为自己不会两面听,所以更容易和牌”的效果就不一定能体现。此外,当巡目已深、场上还有许多看起来更安全的牌时,这种“无法两面听”所带来的好处也会逐渐减弱。

\section{如何计算和牌率}
要判断哪种听牌更容易和牌,可以参考\fullref{lec4:tableA}
。对于表中未列出的情况,我们可以按照以下方法来估算它们的和牌率。当听牌牌有多个时,例如1与8的双碰听,假设场上已经切出了1一张,那么剩余1张的无筋(無スジ)1的和牌率为 $p$\%,剩余2张的无筋8的和牌率为$q$\%。根据概率和公式(\ref{lec4:shiki1})可以得出,不能简单地用 $p + q$,因为“既能和1又能和8”的情形会被重复计算,需要减去重复部分。将数值代入后,实际的和牌率约为 49.7\%,这比“4张的无筋28坎张”更容易和牌。

\begin{table}[h]
    \begin{talltblr}[
            caption = {立直时不同听牌的和牌率数据},
            label = {lec4:tableA},
            remark{注} = {\ref{lec2:table}的表A和表B合并后的数据}]{
            colspec = {ccccc},
            row{odd} = tableGray,
            row{1} = {black,fg=white},
            vline{2} = {1}{white,0.15em},
            vline{2} = {2-10}{black,0.15em},
            vline{3-5} = {1}{white,0.05em},
            vline{3-5} = {2-10}{0.05em}
        }
        听牌    & 4枚和牌率 & 3枚和牌率 & 2枚和牌率 & 1枚和牌率 \\
        无筋19  &       & 50.1  & 41.6  & 26.3  \\
        无筋28  & 42.0  & 38.2  & 31.7  & 19.2  \\
        无筋37  & 36.8  & 32.0  & 25.5  & 14.8  \\
        无筋456 & 31.0  & 26.7  & 20.3  & 11.8  \\
        单筋456 & 35.4  & 30.9  & 24.7  & 12.9  \\
        筋19   &       & 67.9  & 60.0  & 30.1  \\
        筋28   & 56.5  & 51.2  & 42.7  & 24.9  \\
        筋37   & 48.9  & 43.5  & 33.1  & 17.2  \\
        筋456  & 50.0  & 45.4  & 35.5  & 16.5  \\
    \end{talltblr}
\end{table}

\begin{shiki}[和牌率的计算方法]\label{lec4:shiki1}
    例如\explainmahjong{1p}和\explainmahjong{8s}的双碰,\explainmahjong{1p}场上已有一张时,
    假设余量为1的无筋1的和牌率为\(p\%\),余量为2的无筋8的和牌率为\(q\%\):
    \[\mathrm{和牌率} = p + q - \frac{pq}{100}\%\]
\end{shiki}

\begin{table}[h]
    \caption{具体例:以下均为立直后的手牌,6-10巡目}
    \begin{tblr}{
        colspec={X[7,l]X[l]}
        }
        \explainmahjong{25m}的两面,\explainmahjong{2m}2现,\explainmahjong{5m}1现\dotfill                              & 49.9\% \\
        \explainmahjong{147s}的3面听(11枚),\explainmahjong{1s}3现,\explainmahjong{4s}2现,\explainmahjong{7s}1现\dotfill & 60.1\% \\
        \explainmahjong{8p}坎听,2现\dotfill                                                                         & 31.7\% \\
        \explainmahjong{1p8s}双碰,均为无筋生牌\dotfill                                                                   & 60.1\% \\
        \explainmahjong{28m}双碰,筋牌生张\dotfill                                                                      & 67.1\%
    \end{tblr}
\end{table}

对于两面听也可以用同样的方法去估算(虽然表中没列出,但我们可以假定“4张的无筋19”的和牌率大约是 54\% 左右)如果是三面听,则可以先计算其中任意两面听的和牌率,再用与剩下的一个听牌的和牌率同样进行上述公式的计算。需要注意的是,两面听并不会在之后变成筋的听牌,所以实际的和牌率可能会比这个计算结果再稍微低一点。

字牌的和牌率一般高于28筋但略低于19筋。No Chance的牌若只能通过双碰或单骑构成和牌\footnote{即无法成为坎张、边张、两面等搭子},那么它的和牌率可与字牌相当。但如果该牌也可能在坎张、边张中出现,那么和牌率就与一般的筋牌相近。

因为字牌无法用于顺子的组合,场上已有的字牌的用处很低。单骑字牌时,余量2张比起余量3张(生张)更容易和牌。(余量2张时和牌率与筋19相当)。余量1张时(所谓地狱单骑)虽然不如3张的和牌率高,但还是比余量一张的19筋要高(大约等同于无筋19的3张)。对于数牌单骑,枚数多时当然更有利,不过整体来看,听3枚和听2枚之间的和牌率差距并不大。

\section{和牌率与打点的比较}
\begin{table}[h]
    \centering
    \caption{立直时的平均打点(包含立直一发自摸里宝,闲家,1巡目,1.5人攻)}
    \label{lec4:tableB}
    \begin{talltblr}[
            remark{注} = {来自《科学麻将》}
        ]{colspec={cccccccccc},
            row{1} = {black,fg=white},
            vline{2-10} = {1}{white,0.05em},
            vline{2-10} = {2}{black,0.05em},
            %width = \textwidth,
            column{1-7} = {colsep=2pt},
            column{8-10} = {colsep=2pt}}
        40符1番 & 30符2番 & 40符2番 & 30符3番 & 40符3番 & 30符4番 & 40符4番 & 5番    & 6番    & 7番    \\
        2550  & 3480  & 4640  & 6080  & 7150  & 8820  & 8920  & 10820 & 12900 & 14700
    \end{talltblr}
\end{table}


立直时的自摸、里宝牌与一发所包含的平均打点可参考
基本上,如果待牌数(枚数)差别不大,就选择能获得更高打点的听牌\hyperref[lec1:做牌法则4]{做牌法则4}。

好形低打点与愚形高打点比较时,当良形是立直且达到30符3番以上,或者鸣牌且达到40符3番以上时,如果双方的打点差距并不悬殊,一般都会倾向选择好形\hyperref[lec1:做牌法则5]{做牌法则5}。
若良形一方的打点比上述标准更低,在立直情况下,若愚形一方比好形高出1番+10符以上,就选择愚形。
在鸣牌情况下,如果好形一方仅有1番就优先选择好形,但若是2000点的好形与3900点的愚形比较,则倾向于3900点愚形;3900点好形和8000点愚形比较,则倾向于8000点愚形。\hyperref[lec4:pai1]{牌1}。
\begin{figure}[h]
    \caption{良形立直与愚形默听的比较}\label{lec4:pai1}
    \tekiri{1}{24599m234p234678s}{4z}{5m}{多1番+10符所以选择愚形立直}
\end{figure}
然而,要注意的是,若和不成时,还会有可能放铳或被对手自摸的失点,在鸣牌的状况下,好形1番与愚形2番比较,就选择好形。
即便在平场时,也要考虑到排名点的影响:把一手大牌进一步追求到更高打点带来的收益并不大(无论多大的牌,平均顺位也不会比1更高)。因此,就算是好形满贯与愚形倍满的比较,也会选择好形满贯。\hyperref[lec4:pai2]{牌2}
\begin{figure}[h]
    \caption{鸣牌时良形与愚形的比较}\label{lec4:pai2}
    \tekiri{2}{230m345678p22s-5'55z}{4z}{0m}{鸣牌的良形1番和愚形2番相比时选择良形}
\end{figure}
结合待牌别和牌率,进一步比较难解的场合。存在高目低目时,设高目的打点为 $m$ 点、和牌率为 $p$\%,安目的打点为 $n$ 点、和牌率为 $q$\%时,可以根据\ref{lec4:shiki2}来计算。不过,就像上面所说的,如果和牌率 × 打点的乘积差距不大,则应更多地考虑放铳损失或最终排名的影响,优先选择和牌率更高的一方。
\begin{shiki}[高低目的平均打点]\label{lec4:shiki2}
    \[\mathrm{平均打点} = \frac{mq+nq}{p+q}\]
\end{shiki}

\subsection{良形立直与愚形默听的比较}
\textbf{良形立直比愚形暗听更容易和牌。}虽然《科学麻将》中曾将两者的和牌率视为不相上下,但实际上,一旦立直,原本就不易荣和的那些牌,在选择暗听时同样也不容易被打出。而且如果他家不放守还更有可能会使得他家和牌。因此,只要打点方面的损失不算太大,良形立直通常会更有利。\hyperref[lec4:pai3]{牌3}
\begin{figure}[h]
    \caption{良形立直与愚形默听的比较}\label{lec4:pai3}
    \tekiri{3}{24588m234678p234s}{4z}{2m}{立直。但如果\explainmahjong{2m}是宝牌则打\explainmahjong{5m}默听}
\end{figure}

\subsection{与振听立直的比较}
从和牌率来看,振听的两面立直不如普通的愚形立直容易和牌。但振听时,只要和到就一定会有门清自摸的一番,\textbf{因此从点数上来看振听会更有利一点。}若是振听三面听与普通两面听相比,则两面更有优势。

不过,若考虑到其他家在对局中会采取多大程度的攻势,两者的优劣也会随之变化。极端来说,如果立直后能让其他所有对手都弃和,那么振听的坏处就不会显现。但如果可以确定总有一人会继续进攻,那么选择普通的愚形听牌通常会比较好。

\subsection{与听宝牌的比较}
若不听宝牌时两种听牌的和牌率差距并不大,则会倾向于选择听宝牌提升打点而不是防止和牌率下降。不过,假如不做宝牌待就已经有40符3番以上的手牌时,就会更加重视和牌率。像中张牌的宝牌单骑等情形,如果继续等待改良能让整体更容易和牌,就会考虑改良。

当好形(良形)与宝牌待的愚形二者比较时,即使好形仅有立直一番,也与相差两番的愚形基本相同。如果立直后有2番以上则好形更有利。如果尚未拉开如此差距,而当前局面急需高打点,则可选择宝牌;反之就倾向好形。\hyperref[lec4:pai4-5]{牌4-5}

\begin{figure}[h]
    \caption{良形与愚形听宝牌的比较}\label{lec4:pai4-5}
    \te{4}{111678m1235p5678s}{5p}{打\explainmahjong{8s}或是\explainmahjong{5p}立直基本相同}
    \par\bigskip
    \tekiri{5}{2233m11446p1155s3z}{6p}{6p}{特别需要打点时以外打\explainmahjong{6p}立直}
\end{figure}

关于红宝牌,由于每种花色仅有1枚红宝牌,如果像246(8未切)的单筋5与筋牌3那样,虽剩余张数相同,却会在和牌率上显现明显差异时,就会选择打6立直。

\subsection{听宝牌立直与默听宝牌的比较}
在\ref{lecture2}时提到,针对无筋456待且达到40符3番以上的手牌时,选择默听;但是对于宝牌,纵使默听时也不易被对手打出,原则上仍会选择立直。只有在单骑听牌等有较多变为好形的可能,或者切掉宝牌后依旧能立直并保持40符3番以上的情况下,才会选择默听。

\section{见逃的判断}
\subsection{考虑暗听时见逃可能性的立直判断}
\begin{figure}[h]
    \caption{鸣牌时应见逃的手牌}\label{lec4:pai6}
    \te{6}{23m12399p-1'23s77'7z}{2s}{见逃\explainmahjong{4m}}
\end{figure}
如果像\hyperref[lec4:pai6]{牌6}所示那样,高目是满贯,而低目2000 点,二者打点相差约四倍,那么当对手打出低目\explainmahjong{4m}时,就会选择见逃(但若自摸\explainmahjong{4m}则会和牌)。

而一旦已经立直,若再见逃就无法荣和了,所以在平场局势下几乎不会出现见逃。如果分差大到见逃会非常有利,为了在对手打出安目时能见逃却仍保持荣和机会,就会选择默听。\hyperref[lec4:pai7]{牌7}

\begin{figure}[h]
    \caption{应选择默听见逃的手牌}
    \label{lec4:pai7}
    \te{7}{23m122378999p123s}{4z}{默听见逃\explainmahjong{4m},如果自摸到\explainmahjong{4m}则振听立直}
\end{figure}

相反地,如果这手在暗听状态下,安目见逃有价值,但并没有低到在立直后让你去见逃安目,就会直接选择立直。\hyperref[lec4:pai8]{牌8}
\begin{figure}[h]
    \caption{应选择即立的手牌}
    \label{lec4:pai8}
    \te{8}{23m123456p12399s}{2s}{虽然如果默听的话会见逃\explainmahjong{4m},但应该直接立直}
\end{figure}

\subsection{等改良时选择见逃的情况}
\begin{figure}
    \caption{拒和选择振听立直的手牌}
    \label{lec4:pai9-10}
    \tekiri{9}{123m345678p12399s}{2s}{8p}{序盘则打\explainmahjong{8p}振听立直}
    \par\bigskip
    \tekiri{9}{34566777m345p444s}{4z}{6m}{打\explainmahjong{6m}立直}
\end{figure}

如果即使和牌牌已经出现也不和,而继续等待改良的可能,一般只会出现在单骑听这类手中。比方说在序盘阶段,暗听状态下的40符3番以下的七对或门清面子手,如果能变成一色手就有机会将打点提升到4倍,这就属于此类情形。\\
至于连自摸也选择不和的情况,则只会在打点差极端到一定程度时才会出现。例如从Nomi手演变成门清一色手,或者从三暗刻的自摸和牌,变成能听四暗刻单骑的情形。

\subsection{即使自摸和牌也要改为振听立直的情况}
如\hyperref[lec4:pai9-10]{牌9}所示,对于暗听状态下只有2番以下的手牌,若能变成振听三面张,并且立直后可附加平和,那么在序盘就可以选择打\explainmahjong{8p}来转为振听立直。倘若还有其他手役或听牌数量更多,到了中盘依然可能是振听立直更优的局面(但若判断对手大概率已经听牌,就该直接和牌)。类似\hyperref[lec4:pai9-10]{牌10}那样的手形,如果是改良后更加有利,并且还包含多面张的潜在听牌形,就需要特别注意这种转变时机。