\chapter{做牌的大前提}
\section{优先考虑进张而非改良}\label{lec1:做牌法则1}

摸牌后留在手中的有效牌称为“有效牌”。其中,可以让向听数减少(向听前进)的牌称为“进张”,而无法直接减少向听数的牌称为“改良”。换句话说,明确推进手牌进度的是“进张”,而让手牌更容易进步做准备的是“改良”。为了追求和牌,必须优先让向听前进,因此进张优先于改良。

\section{优先进张而非防守}\label{lec1:做牌法则2}
在麻将中,避免失分的最佳方式是自摸或荣和(即和牌)。即使选择放弃,也可能因被对手自摸或流局时无听而失分。因此,在追求和牌的阶段,原则上应优先选择能够增加进张的牌,而不是安全牌。提升手牌价值的改良也应优先考虑,而不是先考虑防守。

\section{优先接近和牌阶段的进张,而非眼前的进张}\label{lec1:做牌法则3}
虽然向听数减少会提高和牌概率,但可接纳的进张牌数量会减少,而进张牌数量减少会让进一步减少向听数变得困难。因此,比起眼前的进张,更应该选择能在接近和牌阶段增加进张数量的策略,以提高和牌的可能性。

\section{优先高打点的进张,而非进张数量}\label{lec1:做牌法则4}
过于注重当前的进张,可能会忽视手牌的打点,这也是现代战术中的常见倾向。即便在接近和牌阶段,进张牌数量减少一点,也几乎不会导致和牌率减半。然而,麻将的计分规则是,在符数为30且翻数为3以下时,每多1翻,得点翻倍。因此,比起单纯追求进张数量,更应该优先选择能带来高打点的进张。

\section{两面以上及满贯以上时优先进张数}\label{lec1:做牌法则5}
在符数为30且翻数为3以下时,每多1翻,得点会翻倍,但当追求更高打点时,得分增长的效率会下降。追求跳满(12000点)或更高点数时,难度会急剧增加。另外,即便两面听牌变成三面听,也不如愚形听牌(单面或边张)变成两面听时的和牌概率提高明显。而如果尝试制作三面听以上的待牌组合,往往会减少能形成两面听的进张牌。因此,比起能达成满贯以上点数或是两面以上的进张,应该优先进张数量。