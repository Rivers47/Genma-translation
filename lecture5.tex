\chapter[【一向听的技术】一向听的分类]{两面形,无雀头形及附加形面子手一向听的牌型(13枚)的分类}

\section{两面子一向听}
\begin{figure}[h]
    \caption{最常见的两面子一向听}
    \label{lec5:pai1-2}
    \te{1}{123m2378p34599s4z}{}{“日本人就是要吃米饭!”一样的王道形状}
    \par\bigskip
    \te{2}{123m22378p34599s}{}{像刚出锅热腾腾的米饭一样的王道}
\end{figure}
这是所有一向听中最常见的类型,类似“吉野家牛丼”那样的基本款。\hyperref[lec5:pai1-2]{牌1-2}
指的就是这种出现频度最高的两面子一向听。
就像将\hyperref[lec5:pai1-2]{牌1}称为“双两面的一向听”一样,在两面形中,根据剩下的两个搭子是什么,进一步可细分不同形态。比如\hyperref[lec5:pai1-2]{牌2}那种好形搭子 + 好形复合搭子的组合,通常也叫做完全一向听。

\section{无雀头形}
\begin{figure}[h]
    \caption{无雀头形一向听}
    \label{lec5:pai3-4}
    \te{3}{123m222378p345s4z}{}{祈祷能成为三面听的无雀头形}
    \par\bigskip
    \te{4}{123m234p345s456z}{}{感觉1秒后就能听牌一样的多进张形}
\end{figure}
\hyperref[lec5:pai3-4]{牌3-4}属于没有雀头的一向听。由于在形成面子与对子时,一般对子更容易做成,所以这种无雀头形一向听往往比两面形拥有更广的进张选择。此外,如果某个搭子直接升级成面子,也能瞬间转变为听牌,因此当尚有两个搭子时,比只剩一个搭子更有机会进张。\\
无雀头形根据剩余的搭子或浮牌不同,还可以再做更细的分类。因为没有雀头,所以对子形成的难易程度对这手牌的整体价值影响很大。

\section{附加形一向听}
\begin{figure}[h]
    \caption{附加形一向听}
    \label{lec5:pai5}
    \te{5}{123m1237p345799s}{}{并不总会好形听牌的附加形}
\end{figure}
附加形一向听是指没有任何搭子的一向听。搭子比起雀头更容易形成,所以一般来说进张选择比无雀头形更广。\\
根据剩下的两张浮牌是什么,也能再进行细分。
如\hyperref[lec5:pai5]{牌5}则是\inlinemahjong{7p}和\inlinemahjong{7s}的附加形一向听。

一向听时的形状比较的要素会比听牌形更多。尤其在面对不同类型的一向听相互比较时,可能会碰到相当复杂的情况。要想做到既准确又迅速地判断,就需要在手牌一向听成形的瞬间,就能快速识别它属于哪种类型,才更容易选出合理的组牌方针。

具体的打牌比较,将围绕如何把\ref{lecture1}的\ref{lec1:rule1}和\ref{lecture3}的\ref{lec3:rule3}应用到不同手牌的类型展开。接下来会结合若干道题目进行说明与解析。