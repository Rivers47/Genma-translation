\chapter[【一向听的技术】两面形的选择 其二]{从复合搭子中切一张时的解说 对于以下三种选择:
1.舍弃一组搭子,
2.从复合搭子中切出一张,
3.舍弃浮牌,
本次我们将讨论第3种情况。}
我们将讨论浮牌舍弃本身的相关内容,以及在浮牌与搭子等不同牌组之间进行比较的情况。

\section{浮牌的价值:浮牌与面子候补的比较}
麻将界自古以来就存在一个讨论:究竟应该保留边张的面子候补,还是保留3~7之间的浮牌?关于这一问题,通过对比以下两个平均数,我们已经得出了结论:
保留边张直到形成一组面子所需的平均巡目数;
保留3~7之间的浮牌直到形成一组面子所需的平均巡目数(包括在舍弃边张时浮牌形成面子的情况)。
研究表明,这两种情况的平均巡目并没有显著差异。

因此,我们可以得出以下结论:
比起3~7之间的浮牌,那些更容易形成面子的四连形、中腹形,以及形成面子后能实现高打点的浮牌(例如红宝牌或涉及役种的浮牌),应该优先于劣势形的面子候补保留下来。
虽然保留边张与保留3~7浮牌直到形成面子的平均巡目数相差不大,但由于麻将一局最多只有18巡,而在其他玩家的攻势下,终局巡目通常会更早。因此,如果平均巡目差距不大,更倾向于能在较早阶段听牌的选择通常会更为有利。

\section{浮牌之间的比较}
浮牌的价值会因为当前手牌中是否已具备足够的面子候补而有很大的变化。如果面子候补已经足够,那么在没有“更好的改良”情况下,优先保留安全度较高的浮牌是更理智的选择。

首先,这是一个显而易见的判断。在牌1中,手牌已经具备足够的面子候补,因此\inlinemahjong{5s}是多余的。此时,应选择舍弃\inlinemahjong{5s},保留安全度更高的\inlinemahjong{4z}。
在牌2中,同样由于面子候补足够,可以仅保留因摸入\inlinemahjong{7p}而形成三面张改良的可能性,同时选择舍弃\inlinemahjong{5s}。
这一判断的核心在于,当面子候补充足时,应优先考虑牌的安全性,并在此基础上兼顾可能的优质改良。

\section{要不要保留安全牌}
\begin{figure}[h]
    \caption{面子已经足够了的时候}
    \label{lec8:pai1-2}
    \tekiri{1}{3478m456p1235s334z}{1p}{5s}{当面子候补已经充足时,像\inlinemahjong{5s}这样的牌就像在吃饱后的方便面,多余且没有实际作用。}
    \par\bigskip
    \tekiri{2}{3478m4568p1235s33z}{1p}{5s}{通常会保留拥有形成两种两面搭子的\inlinemahjong{5s},但是由于已经具备足够的两面搭子,因此应当保留因摸入\inlinemahjong{7p}而形成的三面张变化。}
\end{figure}
\begin{figure}[h]
    \caption{优先安全牌而不是小概率机会}
    \label{lec8:pai3}
    \tekiri{3}{3478m4568p1235s334z}{1p}{8p}{\inlinemahjong{7p}的改良仅能达到门清听牌的形状,但通常情况下,这种形状并不足以支持立直的强度,因此不需要强行推进立直。}
\end{figure}
\begin{figure}[h]
    \caption{保留可能变成灰姑娘牌的浮牌}
    \label{lec8:pai4}
    \tekiri{4}{2344m3479p22678s4z}{1p}{4z}{愚形立直有很高的改良成立直平和一向听的价值}
\end{figure}
\begin{figure}[h]
    \caption{无法充分利用的作为浮牌的宝牌}
    \label{lec8:pai5}
    \tekiri{5}{3478m4568p1235s334z}{1p}{8p}{\inlinemahjong{23s}的引入改良非常强力,但由于当前已经是满贯1向听,因此可以直接打出\inlinemahjong{2s}
    \inlinemahjong{5m}不是赤宝牌,这种微妙的情况下需要结合其他家的手况进行判断。}
\end{figure}

守备优先于进张是原则,但在有手牌改良的浮牌和安全牌之间,应该优先选择哪一个呢?关于这一点,要根据“其他家进入先制听牌的可能性有多高”和“其他家听牌后局势变得危险的程度”这两个因素的场况来判断,因此实战中不能仅通过手牌来做决定。
不过,为了能够根据场况进行判断,也需要设定一定的基准。本书中对于中盘阶段的两面子形一向听情况,将“如果其他家立直会立刻选择防守的手牌,是否有从一向听起可以继续进攻的手牌变化”作为是否保留手牌改良的一个判断基准。
如果即便经过手牌改良,其他家立直后仍需要选择防守,那么保留手牌改良的意义较小。相反,保留安全牌可以在不切出危险牌的情况下赶上,从而最终发展到和牌的机会。反之,如果从一向听起,即使其他家立直后也可以继续进攻,那么保留手牌改良的意义也较小,此时通过减少需要切出的危险牌数量来提升和牌率更加重要。
例如,牌2如果摸入\inlinemahjong{4z},那么优先选择切\inlinemahjong{8p}来保留安全牌。从两面搭子变为三面张的改良确实是理想的进攻形,但需要根据实际情况来衡量其优先级。在这里,由于仅有摸入\inlinemahjong{7p}一种可能性能实现三面进张的改良,不需要过于重视这一改良。
不过,对于“在他家立直时,能从容弃牌但也有机会变成值得推进的手牌”的浮牌,应予以保留。例如像牌4中,\inlinemahjong{4m}的保留能让仅有立直的手牌变成立直平和。
在“牌5”中,立直后是满贯,因此不需保留作为浮牌的宝牌。但在牌6中,有很大可能用到宝牌,因此选择切\inlinemahjong{4z}。

\begin{figure}[h]
    \caption{作为浮牌的宝牌可能可以用上的时候}
    \label{lec8:pai6}
    \tekiri{6}{112233m2278p245s4z}{2s}{4z}{如果摸到\inlinemahjong{3s},即可听牌且能用到宝牌,因此选择切\inlinemahjong{4z}。若摸到\inlinemahjong{2s},\inlinemahjong{4s}或\inlinemahjong{0s},差别不大,因此切\inlinemahjong{2s}。}
\end{figure}

\begin{figure}[h]
    \caption{有较弱的搭子的时候}
    \label{lec8:pai7-10}
    \tekiri{7}{2345m113479p6778s}{1p}{9p}{保留四连形和中腹形,拆掉坎张。}
    \par\bigskip
    \tekiri{8}{116m1115678p1367s}{7s}{1s}{保留四连形和三色,拆掉坎张。}
    \par\bigskip
    \tekiri{9}{112378m9p77789s33z}{7m}{7s}{考虑全带和三色,切\inlinemahjong{7s}。}
    \par\bigskip
    \tekiri{10}{23489m11347p3567s}{2m}{9m}{\inlinemahjong{7m}已切两张时,舍弃仅剩两枚进张的愚形听牌。}
    \par\bigskip
\end{figure}

\section{强浮牌与弱搭子的比较}

基本上优先接受牌,而非追求变化【手牌构筑法则1】。但在存在特别有价值的浮牌(强浮牌)或特别低价值的搭子(弱搭子)时,可以考虑变化【手变法则2·3】。

强浮牌指以下几种牌型:
1. 4567(四连形)或中腹形等容易形成两面以上搭子的牌。
2. 除平和外,能使打点提升1番且有多种变化的牌(包括不确定但可追求高点数的三色或一气通贯等2番役的高价值牌)。
3. 能确实提升打点2番以上的变化的牌。
\section{弱搭子 vs 3~7的浮牌}
\begin{figure}[h]
    \caption{有较弱的搭子的时候}
    \label{lec8:pai11-13}
    \tekiri{11}{127m34678p115s444z}{8p}{5s}{根据改良后的待牌强度选择保留\inlinemahjong{7m}。}
    \par\bigskip
    \tekiri{12}{127m34678p110s444z}{8p}{1m}{\inlinemahjong{0s}(强浮牌)和\inlinemahjong{7m}(较弱的浮牌)都作为浮牌处理。中盘以前不在意向听数}
    \par\bigskip
    \tekiri{13}{334699m36778p567s}{7s}{6m}{不仅保留中腹形,也保留\inlinemahjong{3p}并退回两向听。}
\end{figure}
\begin{figure}[h]
    \caption{强浮牌vs优质复合搭子}
    \label{lec8:pai14-16}
    \te{14}{234677m1235699p3z}{3z}{序盘切\inlinemahjong{7m},之后切\inlinemahjong{3z}。}
    \par\bigskip
    \tekiri{15}{234677m1230699p3z}{3z}{3z}{由于已有一枚红宝牌,一上来就可以切宝牌}
    \par\bigskip
    \tekiri{16}{234677m1230699p3z}{3z}{3z}{中盘以后切\inlinemahjong{7m},在那之后切\inlinemahjong{5s}。}
    \par\bigskip
    
\end{figure}
另一方面,弱搭子指以下2类牌型:
1.进张后可形成面子的牌不足2张的愚形搭子,当在听牌状态下保留时,更倾向于防守而选择默听(或拆牌)的搭子。
2.进张与其他搭子重复,达成两面听的改良可能性较低的愚形搭子(如1334)。
\section{Case: 有两组搭子的一向听}
基本上,如果手牌是两个搭子加两个强浮牌的一向听,应舍弃价值较低的愚形搭子,将手牌退回到两向听。然而,中盘以后,应考虑形听保持一向听状态。牌7-9
此外,如果存在1类弱搭子,应在中盘以前将手牌退回两向听并保留3~7的浮牌。而对于2类弱搭子,则在中盘以前采取相同的处理方式。
即使不考虑弱搭子,也可能会出现3~7的浮牌保留优先度的问题。具体来说,这包括:边张、2468这样的搭子,虽然面子的进张没有重复,但切掉一张后会留下复合搭子的搭子,3556这样的搭子,虽然进张有所重叠,但仍保留两面变化的搭子等。
这些情况如果与两个37的浮牌相比,则会选择切浮牌并取一向听。而如果是37的浮牌加强浮牌的组合,中盘以前应当选择退向听。
这是因为在有强浮牌的情况下,从【手变法则4】的角度看,为了最大限度追求变化,退向听会更加有利。牌11-13

\section{两面对子的价值}
关于334这样的两面对子形态,需要进行一些补充说明。
从334的形态中保留3,不仅可以形成刻子的可能,还能在摸入2时形成中腹形,或在摸入5时,虽然不如中腹形,但会形成比普通3~7浮牌更有价值的准两面形。因此,即使放弃了不理想的听牌形态,也更容易转化为强劲的黏连一向听。
另外,如果在两个对子中固定一个两面搭子,就可能错过另一对头候补变成面子候补的改良。例如:
手牌\inlinemahjong{334m55p}(没有其他雀头,宝牌为\inlinemahjong{4p}),若选择保留\inlinemahjong{3m},那么在摸入宝牌\inlinemahjong{4p}时,可以将\inlinemahjong{3m}作为雀头候补,从而更方便地利用宝牌。这是一种容易被忽视的雀头候补转化为面子候补的变化。
\begin{figure}[h]
    \caption{强浮牌vs优质复合搭子}
    \label{lec8:pai17}
    \te{17}{234677m1230699p5s}{5s}{序盘切\inlinemahjong{7m},在那之后切\inlinemahjong{5s}。}
\end{figure}

\begin{figure}[h]
    \caption{4连形vs愚形复合搭子}
    \label{lec8:pai18-20}
    \te{18}{34789mm114567p468s}{4z}{中盘切\inlinemahjong{8s},在那之后切\inlinemahjong{7p}。}
    \par\bigskip
    \te{19}{34999m114567p579s}{4z}{序盘切\inlinemahjong{5s},在那之后切\inlinemahjong{7p}。}
    \par\bigskip
    \te{20}{34789m114567p468s}{5p}{中盘以后切\inlinemahjong{8s},在那之后切\inlinemahjong{7p}。}
\end{figure}

\begin{figure}[h]
    \caption{有较弱的搭子的时候}
    \label{lec8:pai21}
    \te{21}{23379m3440p11678s}{8p}{序盘切\inlinemahjong{9m},中盘切\inlinemahjong{3m},在那之后切\inlinemahjong{4p}。}
\end{figure}

\begin{figure}[h]
    \caption{有较弱的搭子的时候}
    \label{lec8:pai22}
    \te{22}{2347m1177p123788s}{4z}{中盘切\inlinemahjong{8s},在那之后切\inlinemahjong{7m}。}
\end{figure}

\begin{figure}[h]
    \caption{有较弱的搭子的时候}
    \label{lec8:pai23}
    \te{23}{345556m2345p1399s}{4z}{中盘切\inlinemahjong{1s},在那之后切\inlinemahjong{2p}。}
\end{figure}

\begin{figure}[h]
    \caption{有较弱的搭子的时候}
    \label{lec8:pai24}
    \te{24}{22234445m445p5678s}{4z}{序盘切\inlinemahjong{4m},在那之后切\inlinemahjong{8s}。}
\end{figure}

\begin{figure}[h]
    \caption{有较弱的搭子的时候}
    \label{lec8:pai25}
    \te{25}{4567m1124789p224s}{9p}{中盘切\inlinemahjong{4s},在那之后切\inlinemahjong{7m}。}
\end{figure}

\begin{figure}[h]
    \caption{有较弱的搭子的时候}
    \label{lec8:pai26}
    \te{26}{13889m22789p444s4z}{4p}{中盘切\inlinemahjong{1m},在那之后切\inlinemahjong{4z}。}
\end{figure}

\section{搭子+复合搭子+强浮牌}
当手牌包含搭子、复合搭子和强浮牌时,除了直接舍弃搭子外,也可以选择将复合搭子固定为搭子。以下对强浮牌以外的形态进行分类并逐一分析:
\section{case: 优质搭子 + 优质复合搭子}
这是“讲座5”中提到的完全一向听的形态。如果手牌中有符合48页中2或3条件的强浮牌,在序盘可以保留。如果手牌中的强浮牌2和3两个条件都符合,则可保留至中盘以后。
\section{case2: 优质搭子 + 愚形复合搭子}
当手牌为优质搭子和愚形复合搭子时,序盘阶段可以保留强浮牌。如果是符合48页“123”多个条件的特别强浮牌,则可保留至中盘以后。
愚形复合搭子的一张舍弃:通常情况下,优先考虑愚形复合搭子的一张切法,使得当优质搭子先填充完成并形成愚形听牌时更有利。
特殊情况:对于468的两嵌形搭子,除了考虑3的变换,还需注意可能先摸入赤5的情况,因此选择切8更为合适(牌18-20)

\section{愚形搭子+优质复合搭子}
良形听牌时如果能附加平和,序盘阶段选择舍弃愚形搭子,中盘阶段将优质复合搭子固定为搭子,中盘过后切强浮牌(但特别强的浮牌可保留至中盘过后)。具体来说,牌21中,序盘阶段切\inlinemahjong{9m}舍弃愚形搭子,中盘阶段切\inlinemahjong{3m}固定优质复合搭子,中盘过后切强浮牌\inlinemahjong{4p}。
如果不能附加平和,固定搭子的优势较少,因此在中盘阶段也应优先舍弃愚形搭子(不过如果愚形听牌仍较容易和牌,则可以选择固定搭子)。此外,如果愚形搭子是弱搭子,则按搭子✕2的情况处理,中盘之前优先舍弃愚形搭子。
牌22中,即使切\inlinemahjong{8s}固定搭子,接收面只减少\inlinemahjong{8s}的两张,因此在这种情况下例外保留3~7的浮牌。如果\inlinemahjong{123s}变为\inlinemahjong{111s},则考虑三暗刻,选择切\inlinemahjong{7m}。
另一方面,正如讲座7中提到的,如果固定搭子会显著减少进张,固定搭子会变得不利。因此中盘之前优先舍弃愚形搭子,中盘过后切强浮牌,如牌23的情况。
牌24的手牌中,切\inlinemahjong{4p}会错过摸\inlinemahjong{36m}听牌的机会,由于有一杯口,切浮牌会稍微增加打点,因此中盘后切\inlinemahjong{8s}。

\section{愚形搭子+愚形复合搭子}
如果将愚形复合搭子固定为愚形搭子,改良后的情况会显著逊于直接舍弃其他愚形搭子,因此如果固定的愚形搭子与手役无关,这种选择是有损牌效的。然而,如果固定为对子,且变换后固定的对子能成为雀头,则按优质搭子固定的规则处理。
牌25中,选择切\inlinemahjong{4s},即使万子部分发生变化,也能保留\inlinemahjong{112p}和雀头的inlinemahjong{2s},同时摸入\inlinemahjong{56p}还有一气通贯的可能性,因此序盘阶段切\inlinemahjong{4s}比直接舍弃筒子更好。
其他情况下,基本按照愚形搭子+优质复合搭子的标准选择打牌。然而,中盘阶段是否立直愚形听牌需要考虑回避失点的因素。如果没有役种或宝牌,应稍微优先追求变换并选择打出愚形搭子。
牌26中,在实战中选择打\inlinemahjong{1m},和牌的可能性极低,即使成功听牌,收益也可能不明显,因此应优先考虑即使概率低但能提升手牌价值的变换(如宝牌重叠或三暗刻)。如果规则中一发或里牌的价值较高,可以在早期切\inlinemahjong{4z};中盘之后,由于听牌奖励,应切\inlinemahjong{4z},但在实战中也应考虑他家的手牌情况,避免轻易打出宝牌。

\section{复合搭子✕2}
正如讲座7提到的,如果没有其他对子,优先固定雀头而非搭子,以提升接收能力。
牌27中,打\inlinemahjong{8s}保留\inlinemahjong{455p}的两面对子形,摸入\inlinemahjong{79m}或\inlinemahjong{35p}时,能形成优质形态且附加平和。序盘阶段可追求变换,选择打\inlinemahjong{9m},中盘切\inlinemahjong{8s},中盘过后切\inlinemahjong{4p}。
如果固定搭子后产生的多余牌是强浮牌,则根据之前的标准判断是否保留。
牌28中,切\inlinemahjong{5p},能接收\inlinemahjong{24578s}形成断幺,或摸入\inlinemahjong{346m}增加灵活性。由于\inlinemahjong{5m}是特别强的浮牌,可保留至中盘过后再切\inlinemahjong{5p}。

\section{舍弃雀头或面子的情况}
基本原则是面子>雀头>搭子,因此舍弃面子或唯一的对子并退回向听状态是非常少见的情况。
舍弃雀头的情况:即使没有雀头,其他地方也很容易形成雀头,同时能提升打点的情况。
舍弃面子的情况:仅限于剩余的牌全部是与手役相关的面子候补或雀头,并且能显著提升打点的情况。

\section{福地的碎碎念}
关于这次的内容,我的平衡思路和ネマタ的平衡思路略有不同。我这边更倾向于接受愚形听牌,并优先考虑听牌速度。很少会为了追求好形而降低向听数。
对于每个具体问题,很难给出明确的回答,这可能是因为我对愚形听牌的判断更多依赖场况。两面搭子无论如何都保持“两面搭子再差也还是两面”,但边张和坎张之间,则有“场况良好的边坎张”和“场况不佳的边坎张”这样显著的差异。这种思路的差别,或许可以看作是研究者与实战者之间的区别。
让我深有启发的是牌4。从今以后,我会更有意识地区分那些强浮牌和一般浮牌。即使是否完全按照这些例子中的方法打牌另当别论,我认为这次讲座中的例子非常值得参考。像牌14~17、牌18~20、牌21以及牌27,这些都不是仅限于某种特定形状的讨论,因此我觉得参考价值极高。
这一回的内容,即便正文部分依然有些难以理解,我也建议多次反复阅读上方列出的具体例子。