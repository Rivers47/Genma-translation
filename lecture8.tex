\chapter[【一向听的技术】两面形的选择 其二]{从复合搭子中切一张时的解说 对于以下三种选择:
1.舍弃一组搭子,
2.从复合搭子中切出一张,
3.舍弃浮牌,
本次我们将讨论第3种情况。}
我们将讨论浮牌舍弃本身的相关内容,以及在浮牌与搭子等不同牌组之间进行比较的情况。

\section{浮牌的价值:浮牌与面子候补的比较}
麻将界自古以来就存在一个讨论:究竟应该保留边张的面子候补,还是保留3~7之间的浮牌?关于这一问题,通过对比以下两个平均数,我们已经得出了结论:
保留边张直到形成一组面子所需的平均巡目数;
保留3~7之间的浮牌直到形成一组面子所需的平均巡目数(包括在舍弃边张时浮牌形成面子的情况)。
研究表明,这两种情况的平均巡目并没有显著差异。

因此,我们可以得出以下结论:
比起3~7之间的浮牌,那些更容易形成面子的四连形、中腹形,以及形成面子后能实现高打点的浮牌(例如红宝牌或涉及役种的浮牌),应该优先于劣势形的面子候补保留下来。
虽然保留边张与保留3~7浮牌直到形成面子的平均巡目数相差不大,但由于麻将一局最多只有18巡,而在其他玩家的攻势下,终局巡目通常会更早。因此,如果平均巡目差距不大,更倾向于能在较早阶段听牌的选择通常会更为有利。

\section{浮牌之间的比较}
浮牌的价值会因为当前手牌中是否已具备足够的面子候补而有很大的变化。如果面子候补已经足够,那么在没有“更好的改良”情况下,优先保留安全度较高的浮牌是更理智的选择。

首先,这是一个显而易见的判断。在牌1中,手牌已经具备足够的面子候补,因此\inlinemahjong{5s}是多余的。此时,应选择舍弃\inlinemahjong{5s},保留安全度更高的\inlinemahjong{4z}。
在牌2中,同样由于面子候补足够,可以仅保留因摸入\inlinemahjong{7p}而形成三面张改良的可能性,同时选择舍弃\inlinemahjong{5s}。
这一判断的核心在于,当面子候补充足时,应优先考虑牌的安全性,并在此基础上兼顾可能的优质改良。

\section{要不要保留安全牌}
\begin{figure}[h]
    \caption{面子已经足够了的时候}
    \label{lec8:pai1-2}
    \tekiri{1}{3478m456p1235s334z}{1p}{5s}{当面子候补已经充足时,像\inlinemahjong{5s}这样的牌就像在吃饱后的方便面,多余且没有实际作用。}
    \par\bigskip
    \tekiri{2}{3478m4568p1235s33z}{1p}{5s}{通常会保留拥有形成两种两面搭子的\inlinemahjong{5s},但是由于已经具备足够的两面搭子,因此应当保留因摸入\inlinemahjong{7p}而形成的三面张变化。}
\end{figure}
\begin{figure}[h]
    \caption{优先安全牌而不是小概率机会}
    \label{lec8:pai3}
    \tekiri{3}{3478m4568p1235s334z}{1p}{8p}{\inlinemahjong{7p}的改良仅能达到门清听牌的形状,但通常情况下,这种形状并不足以支持立直的强度,因此不需要强行推进立直。}
\end{figure}
\begin{figure}[h]
    \caption{保留可能变成灰姑娘牌的浮牌}
    \label{lec8:pai4}
    \tekiri{4}{2344m3479p22678s4z}{1p}{4z}{愚形立直有很高的改良成立直平和一向听的价值}
\end{figure}
\begin{figure}[h]
    \caption{浮牌的宝牌无法充分利用}
    \label{lec8:pai5}
    \tekiri{5}{3478m4568p1235s334z}{1p}{8p}{\inlinemahjong{23s}的引入改良非常强力,但由于当前已经是满贯1向听,因此可以直接打出\inlinemahjong{2s}
    \inlinemahjong{5m}不是赤宝牌,这种微妙的情况下需要结合其他家的手况进行判断。}
\end{figure}

守备优先于接纳是原则,但在有手牌变化的浮牌和安全牌之间,应该优先选择哪一个呢?关于这一点,要根据“其他家进入先制听牌的可能性有多高”和“其他家听牌后局势变得危险的程度”这两个因素的场况来判断,因此实战中不能仅通过手牌来做决定。

不过,为了能够根据场况进行判断,也需要设定一定的基准。本书中对于中盘阶段的两面子形一向听情况,将“如果其他家立直,能够立即撤退的手牌,是否有从一向听起可以继续进攻的手牌变化”作为是否保留手牌变化的一个判断基准。

如果即便经过手牌变化,其他家立直后仍需要立即撤退,那么保留手牌变化的意义较小。相反,保留安全牌可以在不切出危险牌的情况下赶上,从而最终连结到和牌的机会。反之,如果从一向听起,即使其他家立直后也可以继续押牌,那么保留手牌变化的意义也较小,此时通过减少需要切出的危险牌数量来提升和牌率更加重要。

例如,手牌中有\inlinemahjong{2p},如果摸入\inlinemahjong{4s},那么优先选择切\inlinemahjong{8p}来保留安全牌。在这种情况下,从两面听牌转为三面听牌的变化优先级较高,但由于\inlinemahjong{7p}只有一种变化,这种变化的重视程度较低。