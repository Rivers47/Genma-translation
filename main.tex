\documentclass[heading = true,scheme=chinese]{ctexbook}
\title{现代麻将技术论}
\author{ネマタ}
%\usepackage[shceme=chinese]{ctex}
%\setCJKmainfont{SarasaGothicSC-Regular.ttf}[BoldFont=SarasaGothicSC-Bold.ttf, ItalicFont=SarasaGothicSC-Italic.ttf] 
%\setCJKfamilyfont{zhsong}{SarasaGothicSC-Regular.ttf}[BoldFont=SarasaGothicSC-Bold.ttf, ItalicFont=SarasaGothicSC-Italic.ttf] 

\ctexset{punct=quanjiao}
\usepackage{tabularray}
\UseTblrLibrary{booktabs, siunitx}
\usepackage{hyperref}
\newcommand{\fullref}[1]{\hyperref[#1]{\ref*{#1}\ \nameref*{#1}}}

\usepackage{geometry}       
\usepackage{framed}        
\geometry{letterpaper, margin=1in}                   
\usepackage{xcolor}
\definecolor{tableGray}{RGB}{210,210,210}
%\setlength{\parindent}{1cm}

\usepackage{subcaption}

%\usepackage[empty]{fullpage}
%\definecolor{easy-eyes}{gray}{0.1}
%\color{easy-eyes}
\usepackage{graphicx}

\setlength{\belowcaptionskip}{10pt plus 3pt minus 2pt}

\usepackage{tikz}
\usetikzlibrary{arrows.meta}
\newcommand{\arrow}{\tikz[thick, -{Triangle[scale=0.8]}]{\draw (0,2ex) -- (0,0) -- (2ex,0)}}

\usepackage{stackengine}
\newcommand{\mahjongexplain}[2]{\stackengine{\Sstackgap}{#1}{\hskip9pt\relax \arrow #2}{U}{l}{\quietstack}{\useanchorwidth}{\stacktype}}
\newcommand{\explainmahjong}[1]{\raisebox{-2pt}{\mahjong[3ex]{#1}}}
\newcommand{\paiindex}[1]{\parbox{2em}{\linespread{1.0}\selectfont\centering\textbf{牌#1}}}
\newcommand{\te}[4]{
    \paiindex{#1}\ \mahjongexplain{\raisebox{-10pt}{\mahjong[6ex]{#2}}}{#4} \quad \stackunder{\raisebox{-10pt}{\mahjong[6ex]{#3}}}{宝牌}}
\newcommand{\tekiri}[5]{
    \paiindex{#1}\ \mahjongexplain{\raisebox{-10pt}{\mahjong[6ex]{#2}}}{#5}%
    \quad \stackunder{\raisebox{-10pt}{\mahjong[6ex]{#3}}}{宝牌}%
    \quad \stackunder{\raisebox{-10pt}{\mahjong[6ex]{#4}}}{切}
}
\newcommand{\tekiririichi}[5]{
    \paiindex{#1}\ \mahjongexplain{\raisebox{-10pt}{\mahjong[6ex]{#2}}}{#5}%
    \quad \stackunder{\raisebox{-10pt}{\mahjong[6ex]{#3}}}{宝牌}%
    \quad \stackon{\stackunder{\raisebox{-10pt}{\mahjong[6ex]{#4}}}{切}}{\scriptsize{立直}}
}
\newcommand{\pai}[3]{\textbf{牌#2}
\stackengine{\Sstackgap}
            {\raisebox{-10pt}{\includegraphics[height=6ex]{#1/te#2.png}}}
            {\hskip9pt\relax \arrow #3}
            {U}
            {l}
            {\quietstack}
            {\useanchorwidth}
            {\stacktype}
\quad
\stackunder{\raisebox{-10pt}{\includegraphics[height=6ex]{#1/te#2dora.png}}}{宝牌}}

\newtheorem{shiki}{公式}
\newcommand{\dashfill}{%
  \leaders\hbox to 1em{\hss-\hss}\hfill\kern0pt
}

\usepackage{mahjong}


\begin{document}
\maketitle
\tableofcontents

\part{做牌的诀窍}

\chapter[做牌的大前提]{进攻时怎样切牌的五个泛用法则}
\label{lecture1}
\section{优先考虑进张而非改良}\label{lec1:rule1}

摸牌后留在手中的有效牌称为“有效牌”。其中,可以让向听数减少(向听前进)的牌称为“进张”,而无法直接减少向听数的牌称为“改良”。换句话说,明确推进手牌进度的是“进张”,而让手牌更容易进步做准备的是“改良”。为了追求和牌,必须优先让向听前进,因此进张优先于改良。

\section{优先进张而非防守}\label{lec1:rule2}
在麻将中,避免失分的最佳方式是自摸或荣和(即和牌)。即使选择放弃,也可能因被对手自摸或流局时无听而失分。因此,在追求和牌的阶段,原则上应优先选择能够增加进张的牌,而不是安全牌。提升手牌价值的改良也应优先考虑,而不是先考虑防守。

\section{优先接近和牌阶段的进张,而非眼前的进张}\label{lec1:rule3}
虽然向听数减少会提高和牌概率,但可接纳的进张牌数量会减少,而进张牌数量减少会让进一步减少向听数变得困难。因此,比起眼前的进张,更应该选择能在接近和牌阶段增加进张数量的策略,以提高和牌的可能性。

\section{优先高打点的进张,而非进张数量}\label{lec1:rule4}
过于注重当前的进张,可能会忽视手牌的打点,这也是现代战术中的常见倾向。即便在接近和牌阶段,进张牌数量减少一点,也几乎不会导致和牌率减半。然而,麻将的计分规则是,在符数为30且翻数为3以下时,每多1翻,得点翻倍。因此,比起单纯追求进张数量,更应该优先选择能带来高打点的进张。

\section{两面以上及满贯以上时优先进张数}\label{lec1:rule5}
在符数为30且翻数为3以下时,每多1翻,得点会翻倍,但当追求更高打点时,得分增长的效率会下降。追求跳满(12000点)或更高点数时,难度会急剧增加。另外,即便两面听牌变成三面听,也不如愚形听牌(单面或边张)变成两面听时的和牌概率提高明显。而如果尝试制作三面听以上的待牌组合,往往会减少能形成两面听的进张牌。因此,比起能达成满贯以上点数或是两面以上的进张,应该优先进张数量。
\chapter[立直的判断]{越接近听牌,技术面的差距的影响就越大}
\label{lecture2}

\section{基本原则:听牌即立直}
当门清状态下优先听牌时,绝大多数情况下选择立直是更有利的策略。这可以通过以下三点理由解释:
\begin{enumerate}
    \item 立直后的打点提升通常大于默听(不立直)提高的和牌率\\
          参考\fullref{lec1:rule4}
    \item 优先听牌的情况下,放铳的风险较低\\
          参考\fullref{lec1:rule2}
    \item 等待手牌改良的情况不多\\
          参考\fullref{lec1:rule1}
\end{enumerate}
此外,巡目越早,立直的优势越大,因此在立直时有利的手牌下,不应拖延数巡选择默听,而是坚持听牌即立直的原则。越是常使用的技巧对成绩影响越大,因此掌握“门前清听牌时,大多数情况下应立刻立直”的原则尤为重要。记住这一点,就像咏唱咒语一样坚定地选择立直吧。

\section{先制听牌即立直的例外}
反过来说,有些情况下不立直可能更为合理。这些例外通常基于以下三个原因之一(或多个)
\begin{enumerate}
    \item 默听带来的和牌率提升优势显著,或者立直带来的打点提升较小。
    \item 放铳的风险特别高。
    \item 手牌改良的潜力较大。
\end{enumerate}

在平场下率先听牌,并且没有特别需要考虑的场况下,本文将重点讨论上述第1或第2条理由这两种情况下,不立直可能更有利的场景。

\subsection{当默听也有役并且足够高点数时,不选择立直}
如果默听跳满,通常选择默听。除非是在序盘下,听牌范围比两面更广(如复合听牌)、或是字牌或边张(如筋19),且立直后也大概率会被打出的听牌。\hyperref[lec2:pai1-3]{牌1-3}


\begin{figure}[h]
    \caption{默听时有役并且打点足够高的手牌}
    \label{lec2:pai1-3}
    \te{1}{23457m234p23488s}{8s}{默听}
    \par\bigskip
    \te{2}{34567m22334405p}{5p}{序盘时立直稍有利}
    \par\bigskip
    \te{3}{1199m112236677z}{5m}{序盘时立直稍有利}
\end{figure}

当默听5番时,选择立直或默听的策略需根据具体情况而定:
良形的话,序盘时倾向立直,中盘时立直略有优势,末盘倾向默听。
愚形听牌时,则要考虑立直后别家会不会打出和牌。
立直后其他玩家很难打出的牌(如无筋456的坎张)则选择默听。
对于其他无筋的边坎张(89),立直在中盘前略有优势。
更容易和出的听牌(如无筋19或骗筋牌)在序盘中立直更有优势。
\hyperref[lec2:pai4-6]{牌4-6}

\begin{figure}[h]
    \caption{默听5番的手牌}
    \label{lec2:pai4-6}
    \te{4}{123456789m23p88s}{8s}{序盘立直末盘默听}
    \par\bigskip
    \te{5}{12399m123p12999s}{4z}{略过中盘为止立直略有利}
    \par\bigskip
    \te{6}{3577m345p345678s}{7m}{默听}
\end{figure}


默听4番与默听5番的情况差别不大,由于立直无法确保跳满,因此稍微偏向默听。
\hyperref[lec2:pai7-9]{牌7}中盘阶段稍微倾向立直,而进入末盘则选择默听。不过,如果是自摸四暗刻的情况,由于役满和牌的可能性较高,倾向于立直(如果默听也能跳满以上,则选择默听)。\hyperref[lec2:pai7-9]{牌8-9}
\begin{figure}[h]
    \caption{默听4番的手牌}
    \label{lec2:pai7-9}
    \te{7}{23467m234p23488s}{4z}{到中盘为止立直略为有利}
    \par\bigskip
    \te{8}{13789m11p345555z}{5z}{略过中盘为止立直略为有利}
    \par\bigskip
    \te{9}{12346789m123p88s}{8s}{默听}
\end{figure}

如果默听已是满贯,而立直自摸也仅止于满贯的情况下,比如默听60符3番的手牌,通常选择默听更有利。而50符3番的情况,相较于默听4番,也更倾向于选择默听。\hyperref[lec2:pai10-12]{牌10-12}
\begin{figure}[h]
    \caption{默听50符3番以上的手牌}
    \label{lec2:pai10-12}
    \te{10}{3456m999p111s777z}{4z}{默听}
    \par\bigskip
    \te{11}{123m44p34888s777z}{4z}{序盘时立直略有利}
    \par\bigskip
    \te{12}{13789m111789p77z}{9p}{序盘时立直略有利}
\end{figure}

对于默听40符3翻的情况,相较于默听4翻,更倾向立直,但如果是无筋的456坎听,依然会选择默听。\hyperref[lec2:pai13-15]{牌13-15}

\begin{figure}[h]
    \caption{默听40符3番的手牌}
    \label{lec2:pai13-15}
    \te{13}{23467m222p23488s}{8s}{序盘立直末盘默听}
    \par\bigskip
    \te{14}{13789m111789p11s}{6p}{最序盘立直末盘默听}
    \par\bigskip
    \te{15}{12346789m123p88s}{1p}{默听}
\end{figure}

此外,即使是在平场,如果剩余局数不多,尤其是在东风战或半庄战的南场,打点需求不高,或者作为Top且领先二位满贯以上时,如果立直仅稍微占优,选择默听也可能是合理的。

\subsection{重视防守时选择不立直}
在默听的情况下,即使是低分牌或没有役的手牌,特别是听牌不佳的愚形听牌时,有时为了避免因放铳而失分,不应选择立直。根据表A和表B中不同听牌的和牌率,对于和牌率低于20\%的听牌(包括振听的愚形听牌),或者在中盘以后,和牌率介于20\%以上但低于30\%的听牌(例如:剩余1张筋的边张〈非19〉,剩余2张无筋的坎张37以下听牌,或剩余3张无筋的坎张456听牌),仅有立直可以选择的情况下,应避免立直。

虽然立直棒的支出也是一种失点,但除了临近流局时仅有立直选择的情况外,一般按照原则进行判断。

表A和表B是按照待牌类型统计的和牌率数据,基于所有选择立直的手牌,取6至10巡的数据。尽管这些数据比实际感知的要高,但在比较不同听牌时仍有参考价值。从这些数字来看,可以切实感受到边听的优势。
\begin{table}[h]
    \caption{来自まほ公のミクシィ日記,東風荘超ラン6-10巡}
    \label{lec2:table}
    \begin{subcaptionblock}{0.6\textwidth}
        \captionsetup{labelformat=empty}
        \caption{表A:余量3张以下的和牌率数据}
        \begin{tblr}{
                colspec = {ccccc},
                row{odd} = tableGray,
                row{1} = {black,fg=white},
                vline{2} = {1}{white,0.15em},
                vline{2} = {2-10}{black,0.15em},
                vline{3-4} = {1}{white,0.05em},
                vline{3-4} = {2-10}{0.05em},
            }
            听牌    & 3枚和牌率 & 2枚和牌率 & 1枚和牌率 \\
            无筋19  & 50.1  & 41.6  & 26.3  \\
            无筋28  & 38.2  & 31.7  & 19.2  \\
            无筋37  & 32.0  & 25.5  & 14.8  \\
            无筋456 & 26.7  & 20.3  & 11.8  \\
            单筋456 & 30.9  & 24.7  & 12.9  \\
            筋19   & 67.9  & 60.0  & 30.1  \\
            筋28   & 51.2  & 42.7  & 24.9  \\
            筋37   & 43.5  & 33.1  & 17.2  \\
            筋456  & 45.4  & 35.5  & 16.5  \\
        \end{tblr}
    \end{subcaptionblock}
    \begin{subcaptionblock}{0.3\textwidth}
        \captionsetup{labelformat=empty}
        \caption{表B:余量4张的和牌率数据}
        \begin{tblr}{
                colspec = {ccccc},
                row{odd} = tableGray,
                row{1} = {black,fg=white},
                vline{2} = {1}{white,0.15em},
                vline{3-5} = {1}{white,0.05em},
            }
            听牌    & 4枚和牌率 \\
            无筋28  & 42.0  \\
            无筋37  & 36.8  \\
            无筋456 & 31.0  \\
            单筋456 & 35.4  \\
            筋28   & 56.5  \\
            筋37   & 48.9  \\
            筋456  & 50.0  \\
        \end{tblr}

    \end{subcaptionblock}
\end{table}

\section{关于得点的思考方式}
\subsection{立直的得点应包含一发、自摸和宝牌}
子家的门清平和(门平)的基本得点是3900点,\textbf{但当子家为平和牌立直并和牌时,其平均得点并不是3900点,而是包含自摸、一发和宝牌在内的约6000点。}然而,尽管如此,子家的平和立直通常被称为“3900立直”,却很少听到有人称之为“约6000点的手牌”。

虽然规则中明确了自摸、一发和宝牌可能提升得点是常识,但当用“3900立直”这种表述时,可能会在无意识中将其视为“和牌后仅3900点的手牌”,从而导致牌局中牌型构筑或进攻与防守的判断偏离最佳策略。

传统上,“平和Nomi时选择默听”的说法来源于这样一种逻辑:为了将得点从1000点提升至2000点而支付1000点的立直棒,这样的投资并不划算。这是因为手牌被视为“和牌后仅得2000点”。

虽然在言语表述中,我们往往会用手役名称或表面得点来表达,但在实际对局中,应该意识到“一发、自摸和宝牌平均得点”的存在,并以此为基础进行判断和决策。”

\subsection{因为难以和牌而选择默听?还是因为难以和牌而选择立直?}
对于默听时达到40符3翻以上且已有较高得点,即使立直也无法显著提升得点的手牌,是否立直的判断往往受到场况和分数状况的影响,因此难以制定明确的基准。

围绕这个问题,存在两种对立的思考方式:“因为难以和牌,所以选择默听”和“因为难以和牌,所以为了阻止其他家的手牌进展而选择立直”。这两种观点看似完全相反,但都听起来很有道理。

究竟哪种方式更正确,归根结底是“即使立直也较容易和牌”以及“即使默听也较难和牌”的程度上问题。例如,如果是“默听时平和3宝牌的14搭子”与“默听时平和役+3宝牌的36搭子”进行比较,通常会认为前者更倾向于选择立直。\textbf{然而根据模拟结果,听14与听36之间并不需要改变立直判断。}这说明更大的判断基准才是决定性因素,而非这种程度的差异。

立直导致失败的典型情况是:“如果选择默听,和牌的牌在对局结束前有机会被打出或自摸,但立直后这些和牌机会完全消失”。而“立直后依然容易和牌”的思路则建立在这样一个假设上:即使对手为了防守而不打铳牌,山中仍然可能有1张以上的和牌残留。

另一方面,即使认为“默听也很难和牌”,在追求和牌的过程中,关键牌往往比无用牌少。常听到“不能让其他家为所欲为”的说法,但如果自身是先制的高得点听牌,不是反而希望对手可以“为所欲为”吗?

需要反复强调的是,这一切最终都是“程度问题”。例如,一般来说,如果是七对子中张牌的单骑听牌,无论是默听还是立直,和牌率的差别不大,因此倾向于选择立直。而如果是幺九牌的单骑听牌,立直后和牌的可能性会显著降低,而默听时还有一定的和牌机会,因此倾向选择默听。至于“阻止其他家的手牌进展”,如果某家已经鸣牌且可能听牌,而自身的待牌即使默听也不容易打出,立直则可能迫使该家弃和,此时选择立直反而更容易达成和牌。

\vspace*{\fill}
\noindent\makebox[\linewidth]{\rule{\textwidth}{0.4pt}}

{\footnotesize 
如果没有特别说明具体的局面,则默认场况为东场的南家,且其他家没有明显的攻击性行为。听牌和一向听的手牌通常视为中盘,两向听的手牌视为序盘,而未到两向听的手牌视为最序盘。\\
巡目的定义如下:最序盘是第1-3巡,序盘是第4-6巡,中盘是第7-9巡,稍过中盘是第10-12巡,末盘是第13-15巡,临近流局是第16巡及以后。

关于立直判断,这里参考了模拟结果。如果在实战中,通过观察场况或其他家的打牌方式,可以明确判断立直或默听会对荣和率产生不同的影响,则应根据实际情况调整判断基准。在模拟中,假设以下情况:两面搭子和无筋嵌张456时1人进攻;其他无筋嵌张时1.5人进攻;而选择默听时2.4人进攻。这里的“$n$人攻”指的是胡牌时自摸与荣和的比例接近$1:n$的状态。}

\chapter{为了改良而不立直的场合}
\section{改良的法则}
\subsection{优先当前的改良,重视改良的质量而非数量}\label{lec3:改良法则1}
如\hyperref[lec1:做牌法则1]{做牌法则1}中所述,与其追求改良,不如优先考虑进张数。而在比较改良时,应优先当前的改良,而不是数手之后才可能实现的改良。在当前的改良中,和\hyperref[lec1:做牌法则3]{做牌法则3·4}一致,应优先考虑能提升胡牌概率或向高得点改良的选项,而非仅仅追求改良的数量。

不过,如\hyperref[lec1:做牌法则5]{做牌法则5}中提到的情况,若是能够形成两面搭子以上的听牌,或提升到满贯以上的得点,通常优先考虑改良的数量。同时,这种情况下如果选择暂不立直而等待改良,则需注意若改良的可能性不足则寻求改良没有优势。

\subsection{如果存在特别高价值的浮牌,应优先考虑改良}\label{lec3:改良法则2}
在听牌前阶段,经常讨论的是边搭与37浮牌的取舍。关于这一问题,结论是通常优先进张而保留边搭,,整体更稳妥,但两者的差别并不显著。因此,如果某张37的浮牌明显具有特别高的价值,那么应该优先保留该浮牌,而舍弃较差的愚形塔子。

\subsection{如果有价值特别低的搭子,则优先考虑改良}
如果某愚形塔子的价值特别低,那么如\hyperref[lec1:做牌法则2]{做牌法则2} 所示,基本上应优先保留3-7浮牌。

\subsection{如果选择改良,应尽可能追求最大化}\label{lec3:改良法则4}
当目标是进行改良时,与其同时兼顾部分进张,不如即便暂时降低向听数,也应尽可能追求手牌的最大化改良(单骑听牌情况除外)。如果选择重视进张数,则应完全优先进张,而不是采取折中的方式,因为这种中途策略成功的情况较少。

\subsection{向听数越高的阶段越应重视改良}
在距离和牌较远的阶段,更倾向于追求手牌的变化。此时,即便是降低向听数的变动,其损失也相对较少,同时还有更多可以用于手变的浮牌。因此,尽可能追求手牌的变化,往往会更有利于后续的发展。

\section{关于等待改良的分类}
在判断是否等待手牌改良时,逐一数出改良的进张数既麻烦又耗时。因此,通过对手牌模式进行分类,可以在观察手牌形状时,快速判断是否适合等待改良(在巡目接近中盘结束或之后,末盘除非是为了追求更高的打点或规避风险,否则应直接选择立直)。

\subsection{附加型一向听(搭子不足的一向听)}

\textbf{在序盘阶段,等待改良的基准是6种可以提升1番的改良或9种能够变成好形的改良}
即使放弃听牌形成两个四连形的一向听,改良成两面的可能性只有8种,无法满足序盘阶段改良等待的条件。不过,如果通过两面听牌可以形成平和,则此类情况等待改良会具有优势。\hyperref[lec3:pai1-5]{牌1-牌2}

\begin{figure}
    \caption{浮牌型一向听} \label{lec3:pai1-5}
    \tekiririichi{1}{45679m2340p11122s}{4z}{2p}{没有平和改良所以即立}
    \par\bigskip
    \tekiri{2}{45679m2340p22789s}{4z}{9m}{能改良为两面听会有平和所以选择等待改良}
    \par\bigskip
    \tekiri{3}{12399m35678p0789s}{9s}{3p}{红5是浮牌的情况下则等待改良}
    \par\bigskip
    \tekiririichi{4}{67m123566778p155z}{9s}{1z}{浮牌较弱}
    \par\bigskip
    \tekiri{5}{67m1235667789p55z}{9s}{6m}{有较强的浮牌}
\end{figure}

通过四连形,中膨形\footnote{顺子中间的牌重复,好像顺子中间膨胀了}或附加型\footnote{指3面子1雀头凑齐时两个孤张等待凑成搭子的形状}等形状,将手牌变成包含两张特别有价值的浮牌(能够使打点倍增)的形状,是判断改良等待是否有利的一个重要基准\hyperref[lec3:改良法则2]{改良法则2}。

如果立直后的手牌仅为两番或以下,则即便改良的可能性稍低于基准,只要存在良形改良+1番提升或直接提升两番的改良,也可以选择等待。此外,如果附加型的改良可以形成染手听牌,那么即便仅有一张中张浮牌,也会选择等待改良\hyperref[lec3:改良法则1]{改良法则1}。\hyperref[lec3:pai1-5]{牌3-牌5}


\subsection{听牌后默听(单骑除外)}

\begin{figure}
    \caption{听牌后默听} \label{lec3:pai6-8}
    \tekiri{6}{12334566m66p3457s}{2m}{7s}{默听。改良面广且打点也会上升}
    \par\bigskip
    \tekiri{7}{46m234566p123456s}{4s}{2p}{默听。除了\raisebox{-2pt}{\mahjong[3ex]{37m}}还有\raisebox{-2pt}{\mahjong[3ex]{34p47s}}的改良}
    \par\bigskip
    \tekiri{8}{1377m345p345567s5z}{7s}{5z}{序盘默听。如果摸到中张的浮牌则打\raisebox{-2pt}{\mahjong[3ex]{1m}}}
\end{figure}

对于双碰听牌,由于能够在保持听牌的同时观察两组对子各自的改良,因此通常作为\hyperref[lec3:改良法则4]{改良法则4}的例外处理,选择默听会在改良方面更有优势。有时,即便未转变为两面听,也可能出现一些例如役牌附加、面子的滑动\footnote{234进5切3变为345等}提升打点等改良,而这些改良在选择拆牌后便可能失去其高价值。\hyperref[lec3:pai6-8]{牌6}

另一方面,仅在极少数情况下才会选择保持嵌张形听牌来等待改良。只有通过面子滑动能显著提升打点;或是序盘阶段有大量浮牌可以使改良形成更有利的一向听两种情况。\hyperref[lec3:pai6-8]{牌7-8}

\subsection{无雀头一向听}
\begin{figure}
    \caption{无雀头一向听} \label{lec3:pai9-11}
    \tekiri{9}{123340567m34p345s}{5s}{3m}{\parbox{0.56\textwidth}{打点上比单骑听牌更好。与打\raisebox{-2pt}{\mahjong[3ex]{1m}}相比,即使没有断幺也有满贯的情况下选择更容易和牌打\raisebox{-2pt}{\mahjong[3ex]{3m}}(如果没有宝牌则打\raisebox{-2pt}{\mahjong[3ex]{1m}})}}
    \par\bigskip
    \tekiri{10}{440m23345p456888s}{4z}{4m}{确保红5并且确定断幺}
    \par\bigskip
    \tekiri{11}{456m5566667p67s44z}{1m}{4z}{饼子的改良有\explainmahjong{4567p}}
\end{figure}

在追求改良时,通常会选择变为单骑听牌,或者舍弃一个搭子以形成附加型的一向听。然而,存在比如雀头非常容易形成,并且如果通过其他手段形成雀头,可以显著提升打点的一些例外情况,即舍弃雀头并转变为无雀头的一向听。\hyperref[lec3:pai9-11]{牌9-11}

\subsection{普通形的单骑听牌}
如果单骑的听牌并不容易和牌,那么即便排除振听的牌,多数情况下也满足改良的条件。然而,若单骑听的是较容易和的牌(例如,3枚无筋19单骑),即使转变为两单骑(如1234),和牌率也不会大幅上升。在这种情况下,如果打点变化也有限,则可以直接选择立直。

\subsection{七对子听牌}
对于1289的数牌或字牌单骑以及筋牌,建议直接立直。
对于无筋37或单筋456,在序盘阶段选择默听等待改良,中盘以后则直接立直。
对于双无筋456,原则上选择默听

\section{难以决定立直与否的微妙情况}
\begin{figure}
    \caption{有役改良丰富且打点足够的手牌} \label{lec3:pai12-13}
    \tekiri{12}{24567m123p788s555z}{8s}{7s}{默听。有役且改良丰富}
    \par\bigskip
    \tekiri{13}{2233m114468p1155s}{1s}{6p}{即使默听打点也足够高}
\end{figure}

\begin{figure}
    \caption{考虑到放铳的风险难以决定立直的手牌} \label{lec3:pai14-15}
    \tekiri{14}{12377m789p234557s}{3z}{7s}{比起打\explainmahjong{5s}坎听,选择双碰的改良面更广}
    \par\bigskip
    \tekiri{15}{24m345678p110999s}{4z}{2m}{不会切红5的孤张}
\end{figure}
当默听的打点已经足够,或需要考虑回避放铳风险时,即使没有明显的改良机会,不立直可能会更有利。

愚形默听的40符3翻及以上的手牌,即使除开改良可能性来看立直更有利,如果搭子中包含像24567这样的四连形,或者是较差的单骑听牌,有较多良形变换的可能性时,可以选择默听(例如,类似七对子+两枚红宝牌的默听25符4翻也是如此)。\hyperref[lec3:pai12-13]{牌12-13}

中盘以后,对于以下情形也需考虑是否立直:听牌率20\%以上但不足30\%的情况下,立直后2翻的手牌,或是听牌率30\%以上但不足40\%的情况下,立直后Nomi的手牌,只要有一定程度的改良面就会选择默听。此外,即使手牌中只有一张较高价值的浮牌,也可以选择舍弃听牌,追求改良。\hyperref[lec3:pai14-15]{牌14-15}


\chapter[【听牌的技术】听牌的选择]{需要从复合搭子选择听牌形状时的判断}

\section{打点相同时的听牌选择}

首先需要明确的是,\textbf{双碰比坎张更有优势}。当两种听牌形的剩余牌张数相同时,大多数情况下双碰听牌的和牌率会高于坎张听牌。只有在类似无筋坎张2和非筋的3~7之间形成的双碰对比,或无筋坎张3和无筋456之间形成的双碰对比时,坎张才可能稍占优势。除此之外,基本都是双碰占优。

从自己的弃牌来看,对于无法形成两面听的牌,对手往往会认为它比较安全,因此即使该牌剩余张数不多,也更容易和牌。除了「筋」以外,不会形成两面听的牌还包括:字牌以及No-Chance牌:即已经看到4张牌、从而确定不会是两面听搭子的那种牌。这种听牌在俗称上也被称作“好听”。

另外,如果河里已经有人切过赤5,那么一般认为其手牌里不会再持有其他5牌,这时3和7就可视同No-Chance牌来对待。再进一步,如果1或9也被切出,则相应的4和6也可当作No-Chance来看。

不过需要注意,如果其他家也已经听牌,并且他们看上去不太会打自己的听牌的话,那么“因为对手认为自己不会两面听,所以更容易和牌”的效果就不一定能体现。此外,当巡目已深、场上还有许多看起来更安全的牌时,这种“无法两面听”所带来的好处也会逐渐减弱。

\section{如何计算和牌率}
要判断哪种听牌更容易和牌,可以参考\fullref{lec4:tableA}
。对于表中未列出的情况,我们可以按照以下方法来估算它们的和牌率。当听牌牌有多个时,例如1与8的双碰听,假设场上已经切出了1一张,那么剩余1张的无筋(無スジ)1的和牌率为 $p$\%,剩余2张的无筋8的和牌率为$q$\%。根据概率和公式(\ref{lec4:shiki1})可以得出,不能简单地用 $p + q$,因为“既能和1又能和8”的情形会被重复计算,需要减去重复部分。将数值代入后,实际的和牌率约为 49.7\%,这比“4张的无筋28坎张”更容易和牌。

\begin{table}[h]
    \begin{talltblr}[
            caption = {立直时不同听牌的和牌率数据},
            label = {lec4:tableA},
            remark{注} = {\ref{lec2:table}的表A和表B合并后的数据}]{
            colspec = {ccccc},
            row{odd} = tableGray,
            row{1} = {black,fg=white},
            vline{2} = {1}{white,0.15em},
            vline{2} = {2-10}{black,0.15em},
            vline{3-5} = {1}{white,0.05em},
            vline{3-5} = {2-10}{0.05em}
        }
        听牌    & 4枚和牌率 & 3枚和牌率 & 2枚和牌率 & 1枚和牌率 \\
        无筋19  &       & 50.1  & 41.6  & 26.3  \\
        无筋28  & 42.0  & 38.2  & 31.7  & 19.2  \\
        无筋37  & 36.8  & 32.0  & 25.5  & 14.8  \\
        无筋456 & 31.0  & 26.7  & 20.3  & 11.8  \\
        单筋456 & 35.4  & 30.9  & 24.7  & 12.9  \\
        筋19   &       & 67.9  & 60.0  & 30.1  \\
        筋28   & 56.5  & 51.2  & 42.7  & 24.9  \\
        筋37   & 48.9  & 43.5  & 33.1  & 17.2  \\
        筋456  & 50.0  & 45.4  & 35.5  & 16.5  \\
    \end{talltblr}
\end{table}

\begin{shiki}[和牌率的计算方法]\label{lec4:shiki1}
    例如\inlinemahjong{1p}和\inlinemahjong{8s}的双碰,\inlinemahjong{1p}场上已有一张时,
    假设余量为1的无筋1的和牌率为\(p\%\),余量为2的无筋8的和牌率为\(q\%\):
    \[\mathrm{和牌率} = p + q - \frac{pq}{100}\%\]
\end{shiki}

\begin{table}[h]
    \caption{具体例:以下均为立直后的手牌,6-10巡目}
    \begin{tblr}{
        colspec={X[7,l]X[l]}
        }
        \inlinemahjong{25m}的两面,\inlinemahjong{2m}2现,\inlinemahjong{5m}1现\dotfill                              & 49.9\% \\
        \inlinemahjong{147s}的3面听(11枚),\inlinemahjong{1s}3现,\inlinemahjong{4s}2现,\inlinemahjong{7s}1现\dotfill & 60.1\% \\
        \inlinemahjong{8p}坎听,2现\dotfill                                                                         & 31.7\% \\
        \inlinemahjong{1p8s}双碰,均为无筋生牌\dotfill                                                                   & 60.1\% \\
        \inlinemahjong{28m}双碰,筋牌生张\dotfill                                                                      & 67.1\%
    \end{tblr}
\end{table}

对于两面听也可以用同样的方法去估算(虽然表中没列出,但我们可以假定“4张的无筋19”的和牌率大约是 54\% 左右)如果是三面听,则可以先计算其中任意两面听的和牌率,再用与剩下的一个听牌的和牌率同样进行上述公式的计算。需要注意的是,两面听并不会在之后变成筋的听牌,所以实际的和牌率可能会比这个计算结果再稍微低一点。

字牌的和牌率一般高于28筋但略低于19筋。No Chance的牌若只能通过双碰或单骑构成和牌\footnote{即无法成为坎张、边张、两面等搭子},那么它的和牌率可与字牌相当。但如果该牌也可能在坎张、边张中出现,那么和牌率就与一般的筋牌相近。

因为字牌无法用于顺子的组合,场上已有的字牌的用处很低。单骑字牌时,余量2张比起余量3张(生张)更容易和牌。(余量2张时和牌率与筋19相当)。余量1张时(所谓地狱单骑)虽然不如3张的和牌率高,但还是比余量一张的19筋要高(大约等同于无筋19的3张)。对于数牌单骑,枚数多时当然更有利,不过整体来看,听3枚和听2枚之间的和牌率差距并不大。

\section{和牌率与打点的比较}
\begin{table}[h]
    \centering
    \caption{立直时的平均打点(包含立直一发自摸里宝,闲家,1巡目,1.5人攻)}
    \label{lec4:tableB}
    \begin{talltblr}[
            remark{注} = {来自《科学麻将》}
        ]{colspec={cccccccccc},
            row{1} = {black,fg=white},
            vline{2-10} = {1}{white,0.05em},
            vline{2-10} = {2}{black,0.05em},
            %width = \textwidth,
            column{1-7} = {colsep=2pt},
            column{8-10} = {colsep=2pt}}
        40符1番 & 30符2番 & 40符2番 & 30符3番 & 40符3番 & 30符4番 & 40符4番 & 5番    & 6番    & 7番    \\
        2550  & 3480  & 4640  & 6080  & 7150  & 8820  & 8920  & 10820 & 12900 & 14700
    \end{talltblr}
\end{table}


立直时的自摸、里宝牌与一发所包含的平均打点可参考
基本上,如果待牌数(枚数)差别不大,就选择能获得更高打点的听牌\hyperref[lec1:rule4]{做牌法则4}。

好形低打点与愚形高打点比较时,当良形是立直且达到30符3番以上,或者鸣牌且达到40符3番以上时,如果双方的打点差距并不悬殊,一般都会倾向选择好形\hyperref[lec1:rule5]{做牌法则5}。
若良形一方的打点比上述标准更低,在立直情况下,若愚形一方比好形高出1番+10符以上,就选择愚形。
在鸣牌情况下,如果好形一方仅有1番就优先选择好形,但若是2000点的好形与3900点的愚形比较,则倾向于3900点愚形;3900点好形和8000点愚形比较,则倾向于8000点愚形。\hyperref[lec4:pai1]{牌1}。
\begin{figure}[h]
    \caption{良形立直与愚形默听的比较}\label{lec4:pai1}
    \tekiri{1}{24599m234p234678s}{4z}{5m}{多1番+10符所以选择愚形立直}
\end{figure}
然而,要注意的是,若和不成时,还会有可能放铳或被对手自摸的失点,在鸣牌的状况下,好形1番与愚形2番比较,就选择好形。
即便在平场时,也要考虑到排名点的影响:把一手大牌进一步追求到更高打点带来的收益并不大(无论多大的牌,平均顺位也不会比1更高)。因此,就算是好形满贯与愚形倍满的比较,也会选择好形满贯。\hyperref[lec4:pai2]{牌2}
\begin{figure}[h]
    \caption{鸣牌时良形与愚形的比较}\label{lec4:pai2}
    \tekiri{2}{230m345678p22s-5'55z}{4z}{0m}{鸣牌的良形1番和愚形2番相比时选择良形}
\end{figure}
结合待牌别和牌率,进一步比较难解的场合。存在高目低目时,设高目的打点为 $m$ 点、和牌率为 $p$\%,安目的打点为 $n$ 点、和牌率为 $q$\%时,可以根据\ref{lec4:shiki2}来计算。不过,就像上面所说的,如果和牌率 × 打点的乘积差距不大,则应更多地考虑放铳损失或最终排名的影响,优先选择和牌率更高的一方。
\begin{shiki}[高低目的平均打点]\label{lec4:shiki2}
    \[\mathrm{平均打点} = \frac{mq+nq}{p+q}\]
\end{shiki}

\subsection{良形立直与愚形默听的比较}
\textbf{良形立直比愚形暗听更容易和牌。}虽然《科学麻将》中曾将两者的和牌率视为不相上下,但实际上,一旦立直,原本就不易荣和的那些牌,在选择暗听时同样也不容易被打出。而且如果他家不放守还更有可能会使得他家和牌。因此,只要打点方面的损失不算太大,良形立直通常会更有利。\hyperref[lec4:pai3]{牌3}
\begin{figure}[h]
    \caption{良形立直与愚形默听的比较}\label{lec4:pai3}
    \tekiri{3}{24588m234678p234s}{4z}{2m}{立直。但如果\inlinemahjong{2m}是宝牌则打\inlinemahjong{5m}默听}
\end{figure}

\subsection{与振听立直的比较}
从和牌率来看,振听的两面立直不如普通的愚形立直容易和牌。但振听时,只要和到就一定会有门清自摸的一番,\textbf{因此从点数上来看振听会更有利一点。}若是振听三面听与普通两面听相比,则两面更有优势。

不过,若考虑到其他家在对局中会采取多大程度的攻势,两者的优劣也会随之变化。极端来说,如果立直后能让其他所有对手都弃和,那么振听的坏处就不会显现。但如果可以确定总有一人会继续进攻,那么选择普通的愚形听牌通常会比较好。

\subsection{与听宝牌的比较}
若不听宝牌时两种听牌的和牌率差距并不大,则会倾向于选择听宝牌提升打点而不是防止和牌率下降。不过,假如不做宝牌待就已经有40符3番以上的手牌时,就会更加重视和牌率。像中张牌的宝牌单骑等情形,如果继续等待改良能让整体更容易和牌,就会考虑改良。

当好形(良形)与宝牌待的愚形二者比较时,即使好形仅有立直一番,也与相差两番的愚形基本相同。如果立直后有2番以上则好形更有利。如果尚未拉开如此差距,而当前局面急需高打点,则可选择宝牌;反之就倾向好形。\hyperref[lec4:pai4-5]{牌4-5}

\begin{figure}[h]
    \caption{良形与愚形听宝牌的比较}\label{lec4:pai4-5}
    \te{4}{111678m1235p5678s}{5p}{打\inlinemahjong{8s}或是\inlinemahjong{5p}立直基本相同}
    \par\bigskip
    \tekiri{5}{2233m11446p1155s3z}{6p}{6p}{特别需要打点时以外打\inlinemahjong{6p}立直}
\end{figure}

关于红宝牌,由于每种花色仅有1枚红宝牌,如果像246(8未切)的单筋5与筋牌3那样,虽剩余张数相同,却会在和牌率上显现明显差异时,就会选择打6立直。

\subsection{听宝牌立直与默听宝牌的比较}
在\ref{lecture2}时提到,针对无筋456待且达到40符3番以上的手牌时,选择默听;但是对于宝牌,纵使默听时也不易被对手打出,原则上仍会选择立直。只有在单骑听牌等有较多变为好形的可能,或者切掉宝牌后依旧能立直并保持40符3番以上的情况下,才会选择默听。

\section{见逃的判断}
\subsection{考虑暗听时见逃可能性的立直判断}
\begin{figure}[h]
    \caption{鸣牌时应见逃的手牌}\label{lec4:pai6}
    \te{6}{23m12399p-1'23s77'7z}{2s}{见逃\inlinemahjong{4m}}
\end{figure}
如果像\hyperref[lec4:pai6]{牌6}所示那样,高目是满贯,而低目2000 点,二者打点相差约四倍,那么当对手打出低目\inlinemahjong{4m}时,就会选择见逃(但若自摸\inlinemahjong{4m}则会和牌)。

而一旦已经立直,若再见逃就无法荣和了,所以在平场局势下几乎不会出现见逃。如果分差大到见逃会非常有利,为了在对手打出安目时能见逃却仍保持荣和机会,就会选择默听。\hyperref[lec4:pai7]{牌7}

\begin{figure}[h]
    \caption{应选择默听见逃的手牌}
    \label{lec4:pai7}
    \te{7}{23m122378999p123s}{4z}{默听见逃\inlinemahjong{4m},如果自摸到\inlinemahjong{4m}则振听立直}
\end{figure}

相反地,如果这手在暗听状态下,安目见逃有价值,但并没有低到在立直后让你去见逃安目,就会直接选择立直。\hyperref[lec4:pai8]{牌8}
\begin{figure}[h]
    \caption{应选择即立的手牌}
    \label{lec4:pai8}
    \te{8}{23m123456p12399s}{2s}{虽然如果默听的话会见逃\inlinemahjong{4m},但应该直接立直}
\end{figure}

\subsection{等改良时选择见逃的情况}
\begin{figure}
    \caption{拒和选择振听立直的手牌}
    \label{lec4:pai9-10}
    \tekiri{9}{123m345678p12399s}{2s}{8p}{序盘则打\inlinemahjong{8p}振听立直}
    \par\bigskip
    \tekiri{9}{34566777m345p444s}{4z}{6m}{打\inlinemahjong{6m}立直}
\end{figure}

如果即使和牌牌已经出现也不和,而继续等待改良的可能,一般只会出现在单骑听这类手中。比方说在序盘阶段,暗听状态下的40符3番以下的七对或门清面子手,如果能变成一色手就有机会将打点提升到4倍,这就属于此类情形。\\
至于连自摸也选择不和的情况,则只会在打点差极端到一定程度时才会出现。例如从Nomi手演变成门清一色手,或者从三暗刻的自摸和牌,变成能听四暗刻单骑的情形。

\subsection{即使自摸和牌也要改为振听立直的情况}
如\hyperref[lec4:pai9-10]{牌9}所示,对于暗听状态下只有2番以下的手牌,若能变成振听三面张,并且立直后可附加平和,那么在序盘就可以选择打\inlinemahjong{8p}来转为振听立直。倘若还有其他手役或听牌数量更多,到了中盘依然可能是振听立直更优的局面(但若判断对手大概率已经听牌,就该直接和牌)。类似\hyperref[lec4:pai9-10]{牌10}那样的手形,如果是改良后更加有利,并且还包含多面张的潜在听牌形,就需要特别注意这种转变时机。





\end{document}
