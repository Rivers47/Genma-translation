\documentclass[heading = true,scheme=chinese]{ctexbook}
\title{现代麻将技术论}
\author{ネマタ}
%\usepackage[shceme=chinese]{ctex}
\setCJKmainfont{SarasaGothicSC-Regular.ttf}[BoldFont=SarasaGothicSC-Bold.ttf, ItalicFont=SarasaGothicSC-Italic.ttf] 
\setCJKfamilyfont{zhsong}{SarasaGothicSC-Regular.ttf}[BoldFont=SarasaGothicSC-Bold.ttf, ItalicFont=SarasaGothicSC-Italic.ttf] 
\ctexset{chapter={
name={第,讲},
number=\arabic{chapter},
}
}
\ctexset{punct=quanjiao}

\usepackage{tabularray}
\UseTblrLibrary{booktabs, siunitx}
\usepackage{xcolor}
\definecolor{tableGray}{RGB}{210,210,210}
%\setlength{\parindent}{1cm}

\usepackage{hyperref}
%\usepackage[nospace]{varioref}
\labelformat{chapter}{第\zhdigits{#1}讲}
\newcommand{\fullref}[1]{\hyperref[#1]{\ref*{#1}\ \nameref*{#1}}}

\usepackage{geometry}       
\usepackage{framed}        
%\geometry{letterpaper, margin=1in}                   
\usepackage{fancyhdr}
\pagestyle{fancy}
\setlength{\headheight}{13pt}


\usepackage{subcaption}
\captionsetup{justification=raggedright,singlelinecheck=false}
%\usepackage[empty]{fullpage}
%\definecolor{easy-eyes}{gray}{0.1}
%\color{easy-eyes}
\usepackage{graphicx}

\setlength{\belowcaptionskip}{10pt plus 3pt minus 2pt}

\usepackage{tikz}
\usetikzlibrary{arrows.meta}
\newcommand{\arrow}{\tikz[thick, -{Triangle[scale=0.8]}]{\draw (0,2ex) -- (0,0) -- (2ex,0)}}

\usepackage[most]{tcolorbox}

\usepackage{stackengine}
\newcommand{\mahjongexplain}[2]{\stackengine{\Sstackgap}{#1}{\hskip9pt\relax \arrow #2}{U}{l}{\quietstack}{\useanchorwidth}{\stacktype}}
\newcommand{\inlinemahjong}[1]{\raisebox{-2pt}{\mahjong[3ex]{#1}}}
\newcommand{\paiindex}[1]{\parbox{2em}{\linespread{1.0}\selectfont\centering\textbf{牌#1}}}

\newcommand{\ted}[3]{
    \paiindex{#1}\ \mahjongexplain{\raisebox{-10pt}{\mahjong[6ex]{#2}}}{#3}
}
\newcommand{\te}[4]{
    \paiindex{#1}\ \mahjongexplain{\raisebox{-10pt}{\mahjong[6ex]{#2}}}{#4} \ifx#3\undefined\else \quad \stackunder{\raisebox{-10pt}{\mahjong[6ex]{#3}}}{宝牌}\fi}
\newcommand{\tekiri}[5]{
    \paiindex{#1}\ \mahjongexplain{\raisebox{-10pt}{\mahjong[6ex]{#2}}}{#5}%
    \quad \stackunder{\raisebox{-10pt}{\mahjong[6ex]{#3}}}{宝牌}%
    \quad \stackunder{\raisebox{-10pt}{\mahjong[6ex]{#4}}}{切}
}
\newcommand{\tekiririichi}[5]{
    \paiindex{#1}\ \mahjongexplain{\raisebox{-10pt}{\mahjong[6ex]{#2}}}{#5}%
    \quad \stackunder{\raisebox{-10pt}{\mahjong[6ex]{#3}}}{宝牌}%
    \quad \stackon{\stackunder{\raisebox{-10pt}{\mahjong[6ex]{#4}}}{切}}{\scriptsize{立直}}
}
\newcommand{\pai}[3]{\textbf{牌#2}
\stackengine{\Sstackgap}
            {\raisebox{-10pt}{\includegraphics[height=6ex]{#1/te#2.png}}}
            {\hskip9pt\relax \arrow #3}
            {U}
            {l}
            {\quietstack}
            {\useanchorwidth}
            {\stacktype}
\quad
\stackunder{\raisebox{-10pt}{\includegraphics[height=6ex]{#1/te#2dora.png}}}{宝牌}}

\newtheorem{shiki}{公式}
\newcommand{\dashfill}{%
  \leaders\hbox to 1em{\hss-\hss}\hfill\kern0pt
}

\usepackage{mahjong}


\begin{document}
\maketitle
\tableofcontents

\part{做牌的诀窍}

\chapter[做牌的大前提]{进攻时怎样切牌的五个泛用法则}
\label{lecture1}
\section{优先考虑进张而非改良}\label{lec1:rule1}

摸牌后留在手中的有效牌称为“有效牌”。其中,可以让向听数减少(向听前进)的牌称为“进张”,而无法直接减少向听数的牌称为“改良”。换句话说,明确推进手牌进度的是“进张”,而让手牌更容易进步做准备的是“改良”。为了追求和牌,必须优先让向听前进,因此进张优先于改良。

\section{优先进张而非防守}\label{lec1:rule2}
在麻将中,避免失分的最佳方式是自摸或荣和(即和牌)。即使选择放弃,也可能因被对手自摸或流局时无听而失分。因此,在追求和牌的阶段,原则上应优先选择能够增加进张的牌,而不是安全牌。提升手牌价值的改良也应优先考虑,而不是先考虑防守。

\section{优先接近和牌阶段的进张,而非眼前的进张}\label{lec1:rule3}
虽然向听数减少会提高和牌概率,但可接纳的进张牌数量会减少,而进张牌数量减少会让进一步减少向听数变得困难。因此,比起眼前的进张,更应该选择能在接近和牌阶段增加进张数量的策略,以提高和牌的可能性。

\section{优先高打点的进张,而非进张数量}\label{lec1:rule4}
过于注重当前的进张,可能会忽视手牌的打点,这也是现代战术中的常见倾向。即便在接近和牌阶段,进张牌数量减少一点,也几乎不会导致和牌率减半。然而,麻将的计分规则是,在符数为30且翻数为3以下时,每多1翻,得点翻倍。因此,比起单纯追求进张数量,更应该优先选择能带来高打点的进张。

\section{两面以上及满贯以上时优先进张数}\label{lec1:rule5}
在符数为30且翻数为3以下时,每多1翻,得点会翻倍,但当追求更高打点时,得分增长的效率会下降。追求跳满(12000点)或更高点数时,难度会急剧增加。另外,即便两面听牌变成三面听,也不如愚形听牌(单面或边张)变成两面听时的和牌概率提高明显。而如果尝试制作三面听以上的待牌组合,往往会减少能形成两面听的进张牌。因此,比起能达成满贯以上点数或是两面以上的进张,应该优先进张数量。
\chapter[立直的判断]{越接近听牌,技术面的差距的影响就越大}
\label{lecture2}

\section{基本原则:听牌即立直}
当门清状态下优先听牌时,绝大多数情况下选择立直是更有利的策略。这可以通过以下三点理由解释:
\begin{enumerate}
    \item 立直后的打点提升通常大于默听(不立直)提高的和牌率\\
          参考\fullref{lec1:rule4}
    \item 优先听牌的情况下,放铳的风险较低\\
          参考\fullref{lec1:rule2}
    \item 等待手牌改良的情况不多\\
          参考\fullref{lec1:rule1}
\end{enumerate}
此外,巡目越早,立直的优势越大,因此在立直时有利的手牌下,不应拖延数巡选择默听,而是坚持听牌即立直的原则。越是常使用的技巧对成绩影响越大,因此掌握“门前清听牌时,大多数情况下应立刻立直”的原则尤为重要。记住这一点,就像咏唱咒语一样坚定地选择立直吧。

\section{先制听牌即立直的例外}
反过来说,有些情况下不立直可能更为合理。这些例外通常基于以下三个原因之一(或多个)
\begin{enumerate}
    \item 默听带来的和牌率提升优势显著,或者立直带来的打点提升较小。
    \item 放铳的风险特别高。
    \item 手牌改良的潜力较大。
\end{enumerate}

在平场下率先听牌,并且没有特别需要考虑的场况下,本文将重点讨论上述第1或第2条理由这两种情况下,不立直可能更有利的场景。

\subsection{当默听也有役并且足够高点数时,不选择立直}
如果默听跳满,通常选择默听。除非是在序盘下,听牌范围比两面更广(如复合听牌)、或是字牌或边张(如筋19),且立直后也大概率会被打出的听牌。\hyperref[lec2:pai1-3]{牌1-3}


\begin{figure}[h]
    \caption{默听时有役并且打点足够高的手牌}
    \label{lec2:pai1-3}
    \te{1}{23457m234p23488s}{8s}{默听}
    \par\bigskip
    \te{2}{34567m22334405p}{5p}{序盘时立直稍有利}
    \par\bigskip
    \te{3}{1199m112236677z}{5m}{序盘时立直稍有利}
\end{figure}

当默听5番时,选择立直或默听的策略需根据具体情况而定:
良形的话,序盘时倾向立直,中盘时立直略有优势,末盘倾向默听。
愚形听牌时,则要考虑立直后别家会不会打出和牌。
立直后其他玩家很难打出的牌(如无筋456的坎张)则选择默听。
对于其他无筋的边坎张(89),立直在中盘前略有优势。
更容易和出的听牌(如无筋19或骗筋牌)在序盘中立直更有优势。
\hyperref[lec2:pai4-6]{牌4-6}

\begin{figure}[h]
    \caption{默听5番的手牌}
    \label{lec2:pai4-6}
    \te{4}{123456789m23p88s}{8s}{序盘立直末盘默听}
    \par\bigskip
    \te{5}{12399m123p12999s}{4z}{略过中盘为止立直略有利}
    \par\bigskip
    \te{6}{3577m345p345678s}{7m}{默听}
\end{figure}


默听4番与默听5番的情况差别不大,由于立直无法确保跳满,因此稍微偏向默听。
\hyperref[lec2:pai7-9]{牌7}中盘阶段稍微倾向立直,而进入末盘则选择默听。不过,如果是自摸四暗刻的情况,由于役满和牌的可能性较高,倾向于立直(如果默听也能跳满以上,则选择默听)。\hyperref[lec2:pai7-9]{牌8-9}
\begin{figure}[h]
    \caption{默听4番的手牌}
    \label{lec2:pai7-9}
    \te{7}{23467m234p23488s}{4z}{到中盘为止立直略为有利}
    \par\bigskip
    \te{8}{13789m11p345555z}{5z}{略过中盘为止立直略为有利}
    \par\bigskip
    \te{9}{12346789m123p88s}{8s}{默听}
\end{figure}

如果默听已是满贯,而立直自摸也仅止于满贯的情况下,比如默听60符3番的手牌,通常选择默听更有利。而50符3番的情况,相较于默听4番,也更倾向于选择默听。\hyperref[lec2:pai10-12]{牌10-12}
\begin{figure}[h]
    \caption{默听50符3番以上的手牌}
    \label{lec2:pai10-12}
    \te{10}{3456m999p111s777z}{4z}{默听}
    \par\bigskip
    \te{11}{123m44p34888s777z}{4z}{序盘时立直略有利}
    \par\bigskip
    \te{12}{13789m111789p77z}{9p}{序盘时立直略有利}
\end{figure}

对于默听40符3翻的情况,相较于默听4翻,更倾向立直,但如果是无筋的456坎听,依然会选择默听。\hyperref[lec2:pai13-15]{牌13-15}

\begin{figure}[h]
    \caption{默听40符3番的手牌}
    \label{lec2:pai13-15}
    \te{13}{23467m222p23488s}{8s}{序盘立直末盘默听}
    \par\bigskip
    \te{14}{13789m111789p11s}{6p}{最序盘立直末盘默听}
    \par\bigskip
    \te{15}{12346789m123p88s}{1p}{默听}
\end{figure}

此外,即使是在平场,如果剩余局数不多,尤其是在东风战或半庄战的南场,打点需求不高,或者作为Top且领先二位满贯以上时,如果立直仅稍微占优,选择默听也可能是合理的。

\subsection{重视防守时选择不立直}
在默听的情况下,即使是低分牌或没有役的手牌,特别是听牌不佳的愚形听牌时,有时为了避免因放铳而失分,不应选择立直。根据表A和表B中不同听牌的和牌率,对于和牌率低于20\%的听牌(包括振听的愚形听牌),或者在中盘以后,和牌率介于20\%以上但低于30\%的听牌(例如:剩余1张筋的边张〈非19〉,剩余2张无筋的坎张37以下听牌,或剩余3张无筋的坎张456听牌),仅有立直可以选择的情况下,应避免立直。

虽然立直棒的支出也是一种失点,但除了临近流局时仅有立直选择的情况外,一般按照原则进行判断。

表A和表B是按照待牌类型统计的和牌率数据,基于所有选择立直的手牌,取6至10巡的数据。尽管这些数据比实际感知的要高,但在比较不同听牌时仍有参考价值。从这些数字来看,可以切实感受到边听的优势。
\begin{table}[h]
    \caption{来自まほ公のミクシィ日記,東風荘超ラン6-10巡}
    \label{lec2:table}
    \begin{subcaptionblock}{0.6\textwidth}
        \captionsetup{labelformat=empty}
        \caption{表A:余量3张以下的和牌率数据}
        \begin{tblr}{
                colspec = {ccccc},
                row{odd} = tableGray,
                row{1} = {black,fg=white},
                vline{2} = {1}{white,0.15em},
                vline{2} = {2-10}{black,0.15em},
                vline{3-4} = {1}{white,0.05em},
                vline{3-4} = {2-10}{0.05em},
            }
            听牌    & 3枚和牌率 & 2枚和牌率 & 1枚和牌率 \\
            无筋19  & 50.1  & 41.6  & 26.3  \\
            无筋28  & 38.2  & 31.7  & 19.2  \\
            无筋37  & 32.0  & 25.5  & 14.8  \\
            无筋456 & 26.7  & 20.3  & 11.8  \\
            单筋456 & 30.9  & 24.7  & 12.9  \\
            筋19   & 67.9  & 60.0  & 30.1  \\
            筋28   & 51.2  & 42.7  & 24.9  \\
            筋37   & 43.5  & 33.1  & 17.2  \\
            筋456  & 45.4  & 35.5  & 16.5  \\
        \end{tblr}
    \end{subcaptionblock}
    \begin{subcaptionblock}{0.3\textwidth}
        \captionsetup{labelformat=empty}
        \caption{表B:余量4张的和牌率数据}
        \begin{tblr}{
                colspec = {ccccc},
                row{odd} = tableGray,
                row{1} = {black,fg=white},
                vline{2} = {1}{white,0.15em},
                vline{3-5} = {1}{white,0.05em},
            }
            听牌    & 4枚和牌率 \\
            无筋28  & 42.0  \\
            无筋37  & 36.8  \\
            无筋456 & 31.0  \\
            单筋456 & 35.4  \\
            筋28   & 56.5  \\
            筋37   & 48.9  \\
            筋456  & 50.0  \\
        \end{tblr}

    \end{subcaptionblock}
\end{table}

\section{关于得点的思考方式}
\subsection{立直的得点应包含一发、自摸和宝牌}
子家的门清平和(门平)的基本得点是3900点,\textbf{但当子家为平和牌立直并和牌时,其平均得点并不是3900点,而是包含自摸、一发和宝牌在内的约6000点。}然而,尽管如此,子家的平和立直通常被称为“3900立直”,却很少听到有人称之为“约6000点的手牌”。

虽然规则中明确了自摸、一发和宝牌可能提升得点是常识,但当用“3900立直”这种表述时,可能会在无意识中将其视为“和牌后仅3900点的手牌”,从而导致牌局中牌型构筑或进攻与防守的判断偏离最佳策略。

传统上,“平和Nomi时选择默听”的说法来源于这样一种逻辑:为了将得点从1000点提升至2000点而支付1000点的立直棒,这样的投资并不划算。这是因为手牌被视为“和牌后仅得2000点”。

虽然在言语表述中,我们往往会用手役名称或表面得点来表达,但在实际对局中,应该意识到“一发、自摸和宝牌平均得点”的存在,并以此为基础进行判断和决策。”

\subsection{因为难以和牌而选择默听?还是因为难以和牌而选择立直?}
对于默听时达到40符3翻以上且已有较高得点,即使立直也无法显著提升得点的手牌,是否立直的判断往往受到场况和分数状况的影响,因此难以制定明确的基准。

围绕这个问题,存在两种对立的思考方式:“因为难以和牌,所以选择默听”和“因为难以和牌,所以为了阻止其他家的手牌进展而选择立直”。这两种观点看似完全相反,但都听起来很有道理。

究竟哪种方式更正确,归根结底是“即使立直也较容易和牌”以及“即使默听也较难和牌”的程度上问题。例如,如果是“默听时平和3宝牌的14搭子”与“默听时平和役+3宝牌的36搭子”进行比较,通常会认为前者更倾向于选择立直。\textbf{然而根据模拟结果,听14与听36之间并不需要改变立直判断。}这说明更大的判断基准才是决定性因素,而非这种程度的差异。

立直导致失败的典型情况是:“如果选择默听,和牌的牌在对局结束前有机会被打出或自摸,但立直后这些和牌机会完全消失”。而“立直后依然容易和牌”的思路则建立在这样一个假设上:即使对手为了防守而不打铳牌,山中仍然可能有1张以上的和牌残留。

另一方面,即使认为“默听也很难和牌”,在追求和牌的过程中,关键牌往往比无用牌少。常听到“不能让其他家为所欲为”的说法,但如果自身是先制的高得点听牌,不是反而希望对手可以“为所欲为”吗?

需要反复强调的是,这一切最终都是“程度问题”。例如,一般来说,如果是七对子中张牌的单骑听牌,无论是默听还是立直,和牌率的差别不大,因此倾向于选择立直。而如果是幺九牌的单骑听牌,立直后和牌的可能性会显著降低,而默听时还有一定的和牌机会,因此倾向选择默听。至于“阻止其他家的手牌进展”,如果某家已经鸣牌且可能听牌,而自身的待牌即使默听也不容易打出,立直则可能迫使该家弃和,此时选择立直反而更容易达成和牌。

\vspace*{\fill}
\noindent\makebox[\linewidth]{\rule{\textwidth}{0.4pt}}

{\footnotesize 
如果没有特别说明具体的局面,则默认场况为东场的南家,且其他家没有明显的攻击性行为。听牌和一向听的手牌通常视为中盘,两向听的手牌视为序盘,而未到两向听的手牌视为最序盘。\\
巡目的定义如下:最序盘是第1-3巡,序盘是第4-6巡,中盘是第7-9巡,稍过中盘是第10-12巡,末盘是第13-15巡,临近流局是第16巡及以后。

关于立直判断,这里参考了模拟结果。如果在实战中,通过观察场况或其他家的打牌方式,可以明确判断立直或默听会对荣和率产生不同的影响,则应根据实际情况调整判断基准。在模拟中,假设以下情况:两面搭子和无筋嵌张456时1人进攻;其他无筋嵌张时1.5人进攻;而选择默听时2.4人进攻。这里的“$n$人攻”指的是胡牌时自摸与荣和的比例接近$1:n$的状态。}

\chapter{为了改良而不立直的场合}
\section{改良的法则}
\subsection{优先当前的改良,重视改良的质量而非数量}\label{lec3:改良法则1}
如\hyperref[lec1:做牌法则1]{做牌法则1}中所述,与其追求改良,不如优先考虑进张数。而在比较改良时,应优先当前的改良,而不是数手之后才可能实现的改良。在当前的改良中,和\hyperref[lec1:做牌法则3]{做牌法则3·4}一致,应优先考虑能提升胡牌概率或向高得点改良的选项,而非仅仅追求改良的数量。

不过,如\hyperref[lec1:做牌法则5]{做牌法则5}中提到的情况,若是能够形成两面搭子以上的听牌,或提升到满贯以上的得点,通常优先考虑改良的数量。同时,这种情况下如果选择暂不立直而等待改良,则需注意若改良的可能性不足则寻求改良没有优势。

\subsection{如果存在特别高价值的浮牌,应优先考虑改良}\label{lec3:改良法则2}
在听牌前阶段,经常讨论的是边搭与37浮牌的取舍。关于这一问题,结论是通常优先进张而保留边搭,,整体更稳妥,但两者的差别并不显著。因此,如果某张37的浮牌明显具有特别高的价值,那么应该优先保留该浮牌,而舍弃较差的愚形塔子。

\subsection{如果有价值特别低的搭子,则优先考虑改良}
如果某愚形塔子的价值特别低,那么如\hyperref[lec1:做牌法则2]{做牌法则2} 所示,基本上应优先保留3-7浮牌。

\subsection{如果选择改良,应尽可能追求最大化}\label{lec3:改良法则4}
当目标是进行改良时,与其同时兼顾部分进张,不如即便暂时降低向听数,也应尽可能追求手牌的最大化改良(单骑听牌情况除外)。如果选择重视进张数,则应完全优先进张,而不是采取折中的方式,因为这种中途策略成功的情况较少。

\subsection{向听数越高的阶段越应重视改良}
在距离和牌较远的阶段,更倾向于追求手牌的变化。此时,即便是降低向听数的变动,其损失也相对较少,同时还有更多可以用于手变的浮牌。因此,尽可能追求手牌的变化,往往会更有利于后续的发展。

\section{关于等待改良的分类}
在判断是否等待手牌改良时,逐一数出改良的进张数既麻烦又耗时。因此,通过对手牌模式进行分类,可以在观察手牌形状时,快速判断是否适合等待改良(在巡目接近中盘结束或之后,末盘除非是为了追求更高的打点或规避风险,否则应直接选择立直)。

\subsection{附加型一向听(搭子不足的一向听)}

\textbf{在序盘阶段,等待改良的基准是6种可以提升1番的改良或9种能够变成好形的改良}
即使放弃听牌形成两个四连形的一向听,改良成两面的可能性只有8种,无法满足序盘阶段改良等待的条件。不过,如果通过两面听牌可以形成平和,则此类情况等待改良会具有优势。\hyperref[lec3:pai1-5]{牌1-牌2}

\begin{figure}
    \caption{浮牌型一向听} \label{lec3:pai1-5}
    \tekiririichi{1}{45679m2340p11122s}{4z}{2p}{没有平和改良所以即立}
    \par\bigskip
    \tekiri{2}{45679m2340p22789s}{4z}{9m}{能改良为两面听会有平和所以选择等待改良}
    \par\bigskip
    \tekiri{3}{12399m35678p0789s}{9s}{3p}{红5是浮牌的情况下则等待改良}
    \par\bigskip
    \tekiririichi{4}{67m123566778p155z}{9s}{1z}{浮牌较弱}
    \par\bigskip
    \tekiri{5}{67m1235667789p55z}{9s}{6m}{有较强的浮牌}
\end{figure}

通过四连形,中膨形\footnote{顺子中间的牌重复,好像顺子中间膨胀了}或附加型\footnote{指3面子1雀头凑齐时两个孤张等待凑成搭子的形状}等形状,将手牌变成包含两张特别有价值的浮牌(能够使打点倍增)的形状,是判断改良等待是否有利的一个重要基准\hyperref[lec3:改良法则2]{改良法则2}。

如果立直后的手牌仅为两番或以下,则即便改良的可能性稍低于基准,只要存在良形改良+1番提升或直接提升两番的改良,也可以选择等待。此外,如果附加型的改良可以形成染手听牌,那么即便仅有一张中张浮牌,也会选择等待改良\hyperref[lec3:改良法则1]{改良法则1}。\hyperref[lec3:pai1-5]{牌3-牌5}


\subsection{听牌后默听(单骑除外)}

\begin{figure}
    \caption{听牌后默听} \label{lec3:pai6-8}
    \tekiri{6}{12334566m66p3457s}{2m}{7s}{默听。改良面广且打点也会上升}
    \par\bigskip
    \tekiri{7}{46m234566p123456s}{4s}{2p}{默听。除了\raisebox{-2pt}{\mahjong[3ex]{37m}}还有\raisebox{-2pt}{\mahjong[3ex]{34p47s}}的改良}
    \par\bigskip
    \tekiri{8}{1377m345p345567s5z}{7s}{5z}{序盘默听。如果摸到中张的浮牌则打\raisebox{-2pt}{\mahjong[3ex]{1m}}}
\end{figure}

对于双碰听牌,由于能够在保持听牌的同时观察两组对子各自的改良,因此通常作为\hyperref[lec3:改良法则4]{改良法则4}的例外处理,选择默听会在改良方面更有优势。有时,即便未转变为两面听,也可能出现一些例如役牌附加、面子的滑动\footnote{234进5切3变为345等}提升打点等改良,而这些改良在选择拆牌后便可能失去其高价值。\hyperref[lec3:pai6-8]{牌6}

另一方面,仅在极少数情况下才会选择保持嵌张形听牌来等待改良。只有通过面子滑动能显著提升打点;或是序盘阶段有大量浮牌可以使改良形成更有利的一向听两种情况。\hyperref[lec3:pai6-8]{牌7-8}

\subsection{无雀头一向听}
\begin{figure}
    \caption{无雀头一向听} \label{lec3:pai9-11}
    \tekiri{9}{123340567m34p345s}{5s}{3m}{\parbox{0.56\textwidth}{打点上比单骑听牌更好。与打\raisebox{-2pt}{\mahjong[3ex]{1m}}相比,即使没有断幺也有满贯的情况下选择更容易和牌打\raisebox{-2pt}{\mahjong[3ex]{3m}}(如果没有宝牌则打\raisebox{-2pt}{\mahjong[3ex]{1m}})}}
    \par\bigskip
    \tekiri{10}{440m23345p456888s}{4z}{4m}{确保红5并且确定断幺}
    \par\bigskip
    \tekiri{11}{456m5566667p67s44z}{1m}{4z}{饼子的改良有\explainmahjong{4567p}}
\end{figure}

在追求改良时,通常会选择变为单骑听牌,或者舍弃一个搭子以形成附加型的一向听。然而,存在比如雀头非常容易形成,并且如果通过其他手段形成雀头,可以显著提升打点的一些例外情况,即舍弃雀头并转变为无雀头的一向听。\hyperref[lec3:pai9-11]{牌9-11}

\subsection{普通形的单骑听牌}
如果单骑的听牌并不容易和牌,那么即便排除振听的牌,多数情况下也满足改良的条件。然而,若单骑听的是较容易和的牌(例如,3枚无筋19单骑),即使转变为两单骑(如1234),和牌率也不会大幅上升。在这种情况下,如果打点变化也有限,则可以直接选择立直。

\subsection{七对子听牌}
对于1289的数牌或字牌单骑以及筋牌,建议直接立直。
对于无筋37或单筋456,在序盘阶段选择默听等待改良,中盘以后则直接立直。
对于双无筋456,原则上选择默听

\section{难以决定立直与否的微妙情况}
\begin{figure}
    \caption{有役改良丰富且打点足够的手牌} \label{lec3:pai12-13}
    \tekiri{12}{24567m123p788s555z}{8s}{7s}{默听。有役且改良丰富}
    \par\bigskip
    \tekiri{13}{2233m114468p1155s}{1s}{6p}{即使默听打点也足够高}
\end{figure}

\begin{figure}
    \caption{考虑到放铳的风险难以决定立直的手牌} \label{lec3:pai14-15}
    \tekiri{14}{12377m789p234557s}{3z}{7s}{比起打\explainmahjong{5s}坎听,选择双碰的改良面更广}
    \par\bigskip
    \tekiri{15}{24m345678p110999s}{4z}{2m}{不会切红5的孤张}
\end{figure}
当默听的打点已经足够,或需要考虑回避放铳风险时,即使没有明显的改良机会,不立直可能会更有利。

愚形默听的40符3翻及以上的手牌,即使除开改良可能性来看立直更有利,如果搭子中包含像24567这样的四连形,或者是较差的单骑听牌,有较多良形变换的可能性时,可以选择默听(例如,类似七对子+两枚红宝牌的默听25符4翻也是如此)。\hyperref[lec3:pai12-13]{牌12-13}

中盘以后,对于以下情形也需考虑是否立直:听牌率20\%以上但不足30\%的情况下,立直后2翻的手牌,或是听牌率30\%以上但不足40\%的情况下,立直后Nomi的手牌,只要有一定程度的改良面就会选择默听。此外,即使手牌中只有一张较高价值的浮牌,也可以选择舍弃听牌,追求改良。\hyperref[lec3:pai14-15]{牌14-15}


\chapter[【听牌的技术】听牌的选择]{需要从复合搭子选择听牌形状时的判断}

\section{打点相同时的听牌选择}

首先需要明确的是,\textbf{双碰比坎张更有优势}。当两种听牌形的剩余牌张数相同时,大多数情况下双碰听牌的和牌率会高于坎张听牌。只有在类似无筋坎张2和非筋的3~7之间形成的双碰对比,或无筋坎张3和无筋456之间形成的双碰对比时,坎张才可能稍占优势。除此之外,基本都是双碰占优。

从自己的弃牌来看,对于无法形成两面听的牌,对手往往会认为它比较安全,因此即使该牌剩余张数不多,也更容易和牌。除了「筋」以外,不会形成两面听的牌还包括:字牌以及No-Chance牌:即已经看到4张牌、从而确定不会是两面听搭子的那种牌。这种听牌在俗称上也被称作“好听”。

另外,如果河里已经有人切过赤5,那么一般认为其手牌里不会再持有其他5牌,这时3和7就可视同No-Chance牌来对待。再进一步,如果1或9也被切出,则相应的4和6也可当作No-Chance来看。

不过需要注意,如果其他家也已经听牌,并且他们看上去不太会打自己的听牌的话,那么“因为对手认为自己不会两面听,所以更容易和牌”的效果就不一定能体现。此外,当巡目已深、场上还有许多看起来更安全的牌时,这种“无法两面听”所带来的好处也会逐渐减弱。

\section{如何计算和牌率}
要判断哪种听牌更容易和牌,可以参考\fullref{lec4:tableA}
。对于表中未列出的情况,我们可以按照以下方法来估算它们的和牌率。当听牌牌有多个时,例如1与8的双碰听,假设场上已经切出了1一张,那么剩余1张的无筋(無スジ)1的和牌率为 $p$\%,剩余2张的无筋8的和牌率为$q$\%。根据概率和公式(\ref{lec4:shiki1})可以得出,不能简单地用 $p + q$,因为“既能和1又能和8”的情形会被重复计算,需要减去重复部分。将数值代入后,实际的和牌率约为 49.7\%,这比“4张的无筋28坎张”更容易和牌。

\begin{table}[h]
    \begin{talltblr}[
            caption = {立直时不同听牌的和牌率数据},
            label = {lec4:tableA},
            remark{注} = {\ref{lec2:table}的表A和表B合并后的数据}]{
            colspec = {ccccc},
            row{odd} = tableGray,
            row{1} = {black,fg=white},
            vline{2} = {1}{white,0.15em},
            vline{2} = {2-10}{black,0.15em},
            vline{3-5} = {1}{white,0.05em},
            vline{3-5} = {2-10}{0.05em}
        }
        听牌    & 4枚和牌率 & 3枚和牌率 & 2枚和牌率 & 1枚和牌率 \\
        无筋19  &       & 50.1  & 41.6  & 26.3  \\
        无筋28  & 42.0  & 38.2  & 31.7  & 19.2  \\
        无筋37  & 36.8  & 32.0  & 25.5  & 14.8  \\
        无筋456 & 31.0  & 26.7  & 20.3  & 11.8  \\
        单筋456 & 35.4  & 30.9  & 24.7  & 12.9  \\
        筋19   &       & 67.9  & 60.0  & 30.1  \\
        筋28   & 56.5  & 51.2  & 42.7  & 24.9  \\
        筋37   & 48.9  & 43.5  & 33.1  & 17.2  \\
        筋456  & 50.0  & 45.4  & 35.5  & 16.5  \\
    \end{talltblr}
\end{table}

\begin{shiki}[和牌率的计算方法]\label{lec4:shiki1}
    例如\inlinemahjong{1p}和\inlinemahjong{8s}的双碰,\inlinemahjong{1p}场上已有一张时,
    假设余量为1的无筋1的和牌率为\(p\%\),余量为2的无筋8的和牌率为\(q\%\):
    \[\mathrm{和牌率} = p + q - \frac{pq}{100}\%\]
\end{shiki}

\begin{table}[h]
    \caption{具体例:以下均为立直后的手牌,6-10巡目}
    \begin{tblr}{
        colspec={X[7,l]X[l]}
        }
        \inlinemahjong{25m}的两面,\inlinemahjong{2m}2现,\inlinemahjong{5m}1现\dotfill                              & 49.9\% \\
        \inlinemahjong{147s}的3面听(11枚),\inlinemahjong{1s}3现,\inlinemahjong{4s}2现,\inlinemahjong{7s}1现\dotfill & 60.1\% \\
        \inlinemahjong{8p}坎听,2现\dotfill                                                                         & 31.7\% \\
        \inlinemahjong{1p8s}双碰,均为无筋生牌\dotfill                                                                   & 60.1\% \\
        \inlinemahjong{28m}双碰,筋牌生张\dotfill                                                                      & 67.1\%
    \end{tblr}
\end{table}

对于两面听也可以用同样的方法去估算(虽然表中没列出,但我们可以假定“4张的无筋19”的和牌率大约是 54\% 左右)如果是三面听,则可以先计算其中任意两面听的和牌率,再用与剩下的一个听牌的和牌率同样进行上述公式的计算。需要注意的是,两面听并不会在之后变成筋的听牌,所以实际的和牌率可能会比这个计算结果再稍微低一点。

字牌的和牌率一般高于28筋但略低于19筋。No Chance的牌若只能通过双碰或单骑构成和牌\footnote{即无法成为坎张、边张、两面等搭子},那么它的和牌率可与字牌相当。但如果该牌也可能在坎张、边张中出现,那么和牌率就与一般的筋牌相近。

因为字牌无法用于顺子的组合,场上已有的字牌的用处很低。单骑字牌时,余量2张比起余量3张(生张)更容易和牌。(余量2张时和牌率与筋19相当)。余量1张时(所谓地狱单骑)虽然不如3张的和牌率高,但还是比余量一张的19筋要高(大约等同于无筋19的3张)。对于数牌单骑,枚数多时当然更有利,不过整体来看,听3枚和听2枚之间的和牌率差距并不大。

\section{和牌率与打点的比较}
\begin{table}[h]
    \centering
    \caption{立直时的平均打点(包含立直一发自摸里宝,闲家,1巡目,1.5人攻)}
    \label{lec4:tableB}
    \begin{talltblr}[
            remark{注} = {来自《科学麻将》}
        ]{colspec={cccccccccc},
            row{1} = {black,fg=white},
            vline{2-10} = {1}{white,0.05em},
            vline{2-10} = {2}{black,0.05em},
            %width = \textwidth,
            column{1-7} = {colsep=2pt},
            column{8-10} = {colsep=2pt}}
        40符1番 & 30符2番 & 40符2番 & 30符3番 & 40符3番 & 30符4番 & 40符4番 & 5番    & 6番    & 7番    \\
        2550  & 3480  & 4640  & 6080  & 7150  & 8820  & 8920  & 10820 & 12900 & 14700
    \end{talltblr}
\end{table}


立直时的自摸、里宝牌与一发所包含的平均打点可参考
基本上,如果待牌数(枚数)差别不大,就选择能获得更高打点的听牌\hyperref[lec1:rule4]{做牌法则4}。

好形低打点与愚形高打点比较时,当良形是立直且达到30符3番以上,或者鸣牌且达到40符3番以上时,如果双方的打点差距并不悬殊,一般都会倾向选择好形\hyperref[lec1:rule5]{做牌法则5}。
若良形一方的打点比上述标准更低,在立直情况下,若愚形一方比好形高出1番+10符以上,就选择愚形。
在鸣牌情况下,如果好形一方仅有1番就优先选择好形,但若是2000点的好形与3900点的愚形比较,则倾向于3900点愚形;3900点好形和8000点愚形比较,则倾向于8000点愚形。\hyperref[lec4:pai1]{牌1}。
\begin{figure}[h]
    \caption{良形立直与愚形默听的比较}\label{lec4:pai1}
    \tekiri{1}{24599m234p234678s}{4z}{5m}{多1番+10符所以选择愚形立直}
\end{figure}
然而,要注意的是,若和不成时,还会有可能放铳或被对手自摸的失点,在鸣牌的状况下,好形1番与愚形2番比较,就选择好形。
即便在平场时,也要考虑到排名点的影响:把一手大牌进一步追求到更高打点带来的收益并不大(无论多大的牌,平均顺位也不会比1更高)。因此,就算是好形满贯与愚形倍满的比较,也会选择好形满贯。\hyperref[lec4:pai2]{牌2}
\begin{figure}[h]
    \caption{鸣牌时良形与愚形的比较}\label{lec4:pai2}
    \tekiri{2}{230m345678p22s-5'55z}{4z}{0m}{鸣牌的良形1番和愚形2番相比时选择良形}
\end{figure}
结合待牌别和牌率,进一步比较难解的场合。存在高目低目时,设高目的打点为 $m$ 点、和牌率为 $p$\%,安目的打点为 $n$ 点、和牌率为 $q$\%时,可以根据\ref{lec4:shiki2}来计算。不过,就像上面所说的,如果和牌率 × 打点的乘积差距不大,则应更多地考虑放铳损失或最终排名的影响,优先选择和牌率更高的一方。
\begin{shiki}[高低目的平均打点]\label{lec4:shiki2}
    \[\mathrm{平均打点} = \frac{mq+nq}{p+q}\]
\end{shiki}

\subsection{良形立直与愚形默听的比较}
\textbf{良形立直比愚形暗听更容易和牌。}虽然《科学麻将》中曾将两者的和牌率视为不相上下,但实际上,一旦立直,原本就不易荣和的那些牌,在选择暗听时同样也不容易被打出。而且如果他家不放守还更有可能会使得他家和牌。因此,只要打点方面的损失不算太大,良形立直通常会更有利。\hyperref[lec4:pai3]{牌3}
\begin{figure}[h]
    \caption{良形立直与愚形默听的比较}\label{lec4:pai3}
    \tekiri{3}{24588m234678p234s}{4z}{2m}{立直。但如果\inlinemahjong{2m}是宝牌则打\inlinemahjong{5m}默听}
\end{figure}

\subsection{与振听立直的比较}
从和牌率来看,振听的两面立直不如普通的愚形立直容易和牌。但振听时,只要和到就一定会有门清自摸的一番,\textbf{因此从点数上来看振听会更有利一点。}若是振听三面听与普通两面听相比,则两面更有优势。

不过,若考虑到其他家在对局中会采取多大程度的攻势,两者的优劣也会随之变化。极端来说,如果立直后能让其他所有对手都弃和,那么振听的坏处就不会显现。但如果可以确定总有一人会继续进攻,那么选择普通的愚形听牌通常会比较好。

\subsection{与听宝牌的比较}
若不听宝牌时两种听牌的和牌率差距并不大,则会倾向于选择听宝牌提升打点而不是防止和牌率下降。不过,假如不做宝牌待就已经有40符3番以上的手牌时,就会更加重视和牌率。像中张牌的宝牌单骑等情形,如果继续等待改良能让整体更容易和牌,就会考虑改良。

当好形(良形)与宝牌待的愚形二者比较时,即使好形仅有立直一番,也与相差两番的愚形基本相同。如果立直后有2番以上则好形更有利。如果尚未拉开如此差距,而当前局面急需高打点,则可选择宝牌;反之就倾向好形。\hyperref[lec4:pai4-5]{牌4-5}

\begin{figure}[h]
    \caption{良形与愚形听宝牌的比较}\label{lec4:pai4-5}
    \te{4}{111678m1235p5678s}{5p}{打\inlinemahjong{8s}或是\inlinemahjong{5p}立直基本相同}
    \par\bigskip
    \tekiri{5}{2233m11446p1155s3z}{6p}{6p}{特别需要打点时以外打\inlinemahjong{6p}立直}
\end{figure}

关于红宝牌,由于每种花色仅有1枚红宝牌,如果像246(8未切)的单筋5与筋牌3那样,虽剩余张数相同,却会在和牌率上显现明显差异时,就会选择打6立直。

\subsection{听宝牌立直与默听宝牌的比较}
在\ref{lecture2}时提到,针对无筋456待且达到40符3番以上的手牌时,选择默听;但是对于宝牌,纵使默听时也不易被对手打出,原则上仍会选择立直。只有在单骑听牌等有较多变为好形的可能,或者切掉宝牌后依旧能立直并保持40符3番以上的情况下,才会选择默听。

\section{见逃的判断}
\subsection{考虑暗听时见逃可能性的立直判断}
\begin{figure}[h]
    \caption{鸣牌时应见逃的手牌}\label{lec4:pai6}
    \te{6}{23m12399p-1'23s77'7z}{2s}{见逃\inlinemahjong{4m}}
\end{figure}
如果像\hyperref[lec4:pai6]{牌6}所示那样,高目是满贯,而低目2000 点,二者打点相差约四倍,那么当对手打出低目\inlinemahjong{4m}时,就会选择见逃(但若自摸\inlinemahjong{4m}则会和牌)。

而一旦已经立直,若再见逃就无法荣和了,所以在平场局势下几乎不会出现见逃。如果分差大到见逃会非常有利,为了在对手打出安目时能见逃却仍保持荣和机会,就会选择默听。\hyperref[lec4:pai7]{牌7}

\begin{figure}[h]
    \caption{应选择默听见逃的手牌}
    \label{lec4:pai7}
    \te{7}{23m122378999p123s}{4z}{默听见逃\inlinemahjong{4m},如果自摸到\inlinemahjong{4m}则振听立直}
\end{figure}

相反地,如果这手在暗听状态下,安目见逃有价值,但并没有低到在立直后让你去见逃安目,就会直接选择立直。\hyperref[lec4:pai8]{牌8}
\begin{figure}[h]
    \caption{应选择即立的手牌}
    \label{lec4:pai8}
    \te{8}{23m123456p12399s}{2s}{虽然如果默听的话会见逃\inlinemahjong{4m},但应该直接立直}
\end{figure}

\subsection{等改良时选择见逃的情况}
\begin{figure}
    \caption{拒和选择振听立直的手牌}
    \label{lec4:pai9-10}
    \tekiri{9}{123m345678p12399s}{2s}{8p}{序盘则打\inlinemahjong{8p}振听立直}
    \par\bigskip
    \tekiri{9}{34566777m345p444s}{4z}{6m}{打\inlinemahjong{6m}立直}
\end{figure}

如果即使和牌牌已经出现也不和,而继续等待改良的可能,一般只会出现在单骑听这类手中。比方说在序盘阶段,暗听状态下的40符3番以下的七对或门清面子手,如果能变成一色手就有机会将打点提升到4倍,这就属于此类情形。\\
至于连自摸也选择不和的情况,则只会在打点差极端到一定程度时才会出现。例如从Nomi手演变成门清一色手,或者从三暗刻的自摸和牌,变成能听四暗刻单骑的情形。

\subsection{即使自摸和牌也要改为振听立直的情况}
如\hyperref[lec4:pai9-10]{牌9}所示,对于暗听状态下只有2番以下的手牌,若能变成振听三面张,并且立直后可附加平和,那么在序盘就可以选择打\inlinemahjong{8p}来转为振听立直。倘若还有其他手役或听牌数量更多,到了中盘依然可能是振听立直更优的局面(但若判断对手大概率已经听牌,就该直接和牌)。类似\hyperref[lec4:pai9-10]{牌10}那样的手形,如果是改良后更加有利,并且还包含多面张的潜在听牌形,就需要特别注意这种转变时机。
\chapter[一向听的分类]{两面形,无雀头形及附加形面子手一向听的牌型(13枚)的分类}

\subsection{两面子一向听}
\begin{figure}
    \caption{最常见的两面子一向听}
    \label{lec5:pai1-2}
    \te{1}{123m2378p34599s4z}{}{“日本人就是要吃米饭!”一样的王道形状}
    \par\bigskip
    \te{2}{123m22378p34599s}{}{像刚出锅热腾腾的米饭一样的王道}
\end{figure}
这是所有一向听中最常见的类型,类似“吉野家牛丼”那样的基本款。\hyperref[lec5:pai1-2]{牌1-2}
指的就是这种出现频度最高的两面子一向听。
就像将\hyperref[lec5:pai1-2]{牌1}称为“双两面的一向听”一样,在两面形中,根据剩下的两个搭子是什么,进一步可细分不同形态。比如\hyperref[lec5:pai1-2]{牌2}那种好形搭子 + 好形复合搭子的组合,通常也叫做完全一向听。

\subsection{无雀头形}
\begin{figure}
    \caption{无雀头形一向听}
    \label{lec5:pai3-4}
    \te{3}{123m222378p345s4z}{}{祈祷能成为三面听的无雀头形}
    \par\bigskip
    \te{4}{123m234p345s456z}{}{感觉1秒后就能听牌一样的多进张形}
\end{figure}
\hyperref[lec5:pai3-4]{牌3-4}属于没有雀头的一向听。由于在形成面子与对子时,一般对子更容易做成,所以这种无雀头形一向听往往比两面形拥有更广的进张选择。此外,如果某个搭子直接升级成面子,也能瞬间转变为听牌,因此当尚有两个搭子时,比只剩一个搭子更有机会进张。\\
无雀头形根据剩余的搭子或浮牌不同,还可以再做更细的分类。因为没有雀头,所以对子形成的难易程度对这手牌的整体价值影响很大。

\subsection{附加形一向听}
\begin{figure}
    \caption{附加形一向听}
    \label{lec5:pai5}
    \te{5}{123m1237p345799s}{}{并不总会好形听牌的附加形}
\end{figure}
附加形一向听是指没有任何搭子的一向听。搭子比起雀头更容易形成,所以一般来说进张选择比无雀头形更广。\\
根据剩下的两张浮牌是什么,也能再进行细分。
如\hyperref[lec5:pai5]{牌5}则是\explainmahjong{7p}和\explainmahjong{7s}的附加形一向听。

一向听时的形状比较的要素会比听牌形更多。尤其在面对不同类型的一向听相互比较时,可能会碰到相当复杂的情况。要想做到既准确又迅速地判断,就需要在手牌一向听成形的瞬间,就能快速识别它属于哪种类型,才更容易选出合理的组牌方针。

具体的打牌比较,将围绕如何把\ref{lec1:rule1}和\ref{lec3:rule3}应用到不同手牌的类型展开。接下来会结合若干道题目进行说明与解析。
\chapter[【一向听的技术】两面形的选择 其一]{一向听的最常见形状:两面形的处理}

手牌可分为五种面子、雀头、搭子、复合搭子、浮牌五种组成成分。
其中,已经完成的面子和作为雀头的对子原则上是不会切的,因此大多数情况下的打牌可分为以下三类:\begin{enumerate}
    \item 切搭子
    \item 从复合搭子中切一张(固定搭子、固定雀头)
    \item 舍弃浮牌
\end{enumerate}

\section{在听牌时要尽量保留更易和牌的搭子}
人们常说“坎张优于边张”,但由于更容易先进入听牌,所以\textbf{比起追求改良,更应优先考虑听牌后的和牌难易度。}

\begin{figure}[h]
    \caption{优先容易和牌的搭子而不是改良}
    \label{lec6:pai1}
    \tekiri{1}{1268m5699p234999s}{9p}{8m}{比起两面改良优先骗筋}
\end{figure}

\hyperref[lec6:pai1]{牌1}与其期待摸进\inlinemahjong{5m}后形成两面,不如切\inlinemahjong{6m}形成骗筋。若差距不大(类似无筋37与无筋456之差),就倾向于舍弃搭子有更多改良的搭子。

\begin{figure}[h]
    \caption{比起改良优先}
    \label{lec6:pai2-4}
    \tekiri{2}{1146m1134578p666s}{5p}{4m}{优先幺九的双碰}
    \par\bigskip
    \tekiri{3}{1146m1134578p666s}{5m}{1p}{优先宝牌进张}
    \par\bigskip
    \tekiri{4}{1166m1134578p666s}{5m}{1m}{优先提高打点的改良}
\end{figure}
\hyperref[lec6:pai2-4]{牌2}切掉对子、保留坎张更容易转变成两面,且有机会摸到最理想的\inlinemahjong{0m}。然而,比起摸进\inlinemahjong{5m}进入听牌,更常出现的是摸进\inlinemahjong{69p}后听牌;此时边张的双碰比坎听\inlinemahjong{5m}要容易和牌得多(并且因为之前切了\inlinemahjong{4m},\inlinemahjong{1m}也能骗筋)。由于\inlinemahjong{0m}只有一枚,因此更推荐舍弃坎张。就算摸进\inlinemahjong{3m},也可扩大进张,所以切\inlinemahjong{4m}。\\
\hyperref[lec6:pai2-4]{牌3}
这里\inlinemahjong{5m}是宝牌,优先考虑保留宝牌的进张。若摸进\inlinemahjong{2p}则确立两面,且能形成高价的一气通贯,比起只等坎张宝牌的面子候补更强,所以选择切\inlinemahjong{1p}。\\
\hyperlink{lec6:pai2-4}{牌4}同样为了优先保留宝牌进张与一气改良,而选择切\inlinemahjong{1m}。

\section{保留高打点的搭子}
\begin{figure}[h]
    \caption{优先高打点而不是改良}
    \label{lec6:pai5}
    \tekiri{5}{12346789m46p2267s}{7s}{4p}{优先一气通贯}
\end{figure}
\hyperlink{lec6:pai5}{牌5}中,切\inlinemahjong{1m}在会保留后续断幺九的可能,但眼前的一通打点更高,因此优先选择这一手。再加上还有678三色的改良,所以先从\inlinemahjong{4p}开始切(在进张数量无差的情况下,就以后续改良可能性来决定)。

\section{尽量保留复合搭子}
\begin{figure}[h]
    \caption{比起改良优先和牌率}
    \label{lec6:pai6}
    \tekiri{6}{1133m34567p56799s}{5s}{3m}{万子变为复合搭子。基本上应把复合搭子做成奇数枚。}
\end{figure}
\hyperlink{lec6:pai6}{牌6}中,万子里作为搭子(对子)部分的牌有4张,切掉其中1张就能让复合搭子保留,从而在总体进张上占优。若切\inlinemahjong{1m},确实能在摸进\inlinemahjong{4m}时变成完全一向听,不过相比之下,更重视听牌后的和牌难易度,因此选择打\inlinemahjong{3m}。

\begin{figure}[h]
    \caption{存在复合进张的场合}
    \label{lec6:pai7-8}
    \tekiri{7}{123m2245667p5699s}{5s}{2p}{不要漏看\inlinemahjong{3p}的坎张}
    \par\bigskip
    \tekiri{8}{5699p2334567788s}{5s}{8s}{略难,打\inlinemahjong{8s}后233456778自摸\inlinemahjong{1469s}即可平和听牌}
\end{figure}
\begin{figure}
    \caption{尽量提高当前的打点}
    \label{lec6:pai9-12}
    \tekiri{9}{1223446m345p1178s}{8s}{1m}{切\inlinemahjong{2m}进张多一枚,但优先一杯口}
    \par\bigskip
    \tekiri{10}{1223446m345p78s77z}{8s}{6m}{切\inlinemahjong{2m}进张多一枚,但优先一杯口}
    \par\bigskip
    \tekiri{11}{1223446789m1178s}{8s}{4m}{带一气的平和}
    \par\bigskip
    \tekiri{12}{1223446789m78s77z}{8s}{2m}{\inlinemahjong{7z}或一气的可能性}
\end{figure}

像\hyperlink{lec6:pai7-8}{牌7}那样,通过切\inlinemahjong{2p}便能获得\inlinemahjong{3p}进张的手牌十分常见。对于245667这类两面坎张(358进张)很容易被忽略,需要格外注意。\\
\hyperlink{lec6:pai7-8}{牌8}所示,也有能形成两面复合进张的形状。若有多种方法能保留复合搭子,则优先考虑当下打点更高的方案。\hyperlink{lec6:pai9-12}{牌9到牌12}中,都是为了保留手役的可能性而作出的打牌选择,属于在多种可选打牌中选出最优先的一种情况。

\section{优先接近和牌阶段的进张,而非眼前的进张}
\begin{figure}
    \caption{优先两面以上的进张}
    \label{lec6:pai13-14}
    \tekiri{13}{57m4455p12345699s}{4z}{7m}{尽量两面听牌}
    \par\bigskip
    \tekiri{14}{57m3445667p12399s}{4z}{5m}{尽量两面以上听牌}
\end{figure}
\begin{figure}
    \caption{比起进张数优先进张的强度}
    \label{lec6:pai15}
    \tekiri{15}{1245m12789p34588s}{3s}{1p}{(\inlinemahjong{3p}2现)所以选择看上去更难受但枚数更多的一边}
\end{figure}
\hyperlink{lec6:pai13-14}{牌13-14}虽然切\inlinemahjong{4p}时能获得更多听牌进张,但更看重能形成两面以上听牌的进张数量。
在牌13中,为了考虑456三色的改良而选择切\inlinemahjong{7m}。在牌14中,万子部分变动后能形成端靠边的两面,更利于和牌,所以切\inlinemahjong{5m}。\\
\hyperlink{lec6:pai15}{牌15}中,\inlinemahjong{3m}的进张与其他部分重叠。虽然连续切\inlinemahjong{1m}\inlinemahjong{2m}可以增加总体进张数量,但为了避免出现(只剩2枚的)边张\inlinemahjong{3p}听牌,优先考虑最终听牌的强度,所以选择切\inlinemahjong{1p}。

\section{比进张数量更优先考虑高打点的进张}
\begin{figure}[h]
    \caption{比进张数量更优先考虑高打点的进张}
    \label{lec6:pai16-21}
    \tekiri{16}{34m2245667p56799s}{4z}{9s}{比起坎\inlinemahjong{3p}的进张选择断幺}
    \par\bigskip
    \tekiri{17}{4566677m4588p999s}{4z}{4m}{一杯口和三暗刻的可能性}
    \par\bigskip
    \tekiri{18}{1133m23499p23479s}{2p}{9s}{一杯口的可能性}
    \par\bigskip
    \tekiri{19}{1245689m34789p22s}{4z}{4p}{一气>平和}
    \par\bigskip
    \tekiri{20}{1256789m34789p22s}{4z}{4p}{一气>平和,但打\inlinemahjong{2m}也可以,如果有1宝牌则打\inlinemahjong{2m}}
    \par\bigskip
    \tekiri{21}{1345689m34789p22s}{4z}{9m}{平和>一气}
\end{figure}

\hyperlink{lec6:pai16-21}{牌16}中,虽然切\inlinemahjong{2p}能获得更多进张,但为了向断幺发展而优先选择该进张。
同理,牌17中虽然切\inlinemahjong{7m}进张面最广,但更应该优先一杯口和三暗刻的机会。\\
在牌18容易被忽略的情况是:若摸\inlinemahjong{2m}可形成一杯口,所以切\inlinemahjong{9s}(若摸\inlinemahjong{6s}则优先保留更佳形状,改切\inlinemahjong{3m})。\\
牌19里,即使先完成愚形,也因为\textbf{愚形立直一气>平和立直},所以会舍弃两面。\\
牌20中,舍弃边张可确保形成良形,虽然一气尚不确定,但仍然是放弃两面更好。如果\inlinemahjong{9m}是宝牌,那么在摸\inlinemahjong{4m}的情况下也是\textbf{立平宝1>愚形立直一气宝1},所以选择舍弃边张。\\
即便如此,牌21中,如果切\inlinemahjong{9m},就会留下134568这类两坎张形状,进张数量差距明显,因此舍弃一气。

\section{两面以上、满贯以上时优先考虑进张数量}
\begin{figure}[h]
    \caption{打点已经足够高所以优先考虑进张数}
    \label{lec6:pai22}
    \tekiri{22}{4566677m3488p567s}{8p}{7m}{自摸\inlinemahjong{3m}也有满贯所以优先进张数打\inlinemahjong{7m}}
\end{figure}
先前提到的例子中,往往是为了打点而牺牲进张数量的情况比较多;但是如果能确保“两面以上、满贯以上”的条件时,就会优先考虑进张数量。\\
\hyperlink{lec6:pai22}{牌22}中,有2枚宝牌,且断幺九已经确定。与其去考虑一杯口,不如优先保留更多的进张数量。\\
\begin{figure}[h]
    \caption{优先两面以上的进张}
    \label{lec6:pai23}
    \tekiri{23}{34678m56799p3467s}{4z}{4s}{保留678三色可能的同时扩大进张面}
\end{figure}
\hyperlink{lec6:pai23}{牌23}中,与其追求摸\inlinemahjong{5s}后形成三面听,不如先优先让自己两面听牌的进张更多,故选择切\inlinemahjong{4s},并且还能保留带有高打点的\inlinemahjong{8s}三色改良。
若宝牌是\inlinemahjong{5s},就要更重视高打点的进张,切\inlinemahjong{4m}。
另外,若宝牌是\inlinemahjong{5s}且\inlinemahjong{5p}为红5,由于能到满贯以上,此时仍优先进张数量,改切\inlinemahjong{4s}。

\section{考虑改良}
\begin{figure}
    \caption{注意改良}
    \label{lec6:pai24-26}
    \tekiri{24}{1279m34578p12366s}{1s}{1m}{从边张的外侧开始切}
    \par\bigskip
    \tekiri{25}{3479m345888p2256s}{4z}{7m}{因为已经确定良形所以从内测开始}
    \par\bigskip
    \tekiri{26}{34568m1168p22256s}{1p}{8p}{留下连续形}
    \par\bigskip
    \tekiri{27}{2478m1168p222567s}{1p}{2m}{比起自摸\inlinemahjong{5m}自摸\inlinemahjong{5p}更好}
\end{figure}
在舍弃搭子时,基本会先观察是否能变成更好的形状,因此通常从外侧开始切。
\hyperlink{lec6:pai24-26}{牌24}中,先切\inlinemahjong{1m};若随后摸进\inlinemahjong{3m},则会留下振听形状并切\inlinemahjong{9m}。如果先摸进\inlinemahjong{14m},就不会变成振听;就算最后留着振听,也不比坎\inlinemahjong{8m}的听牌更糟。因此当要切掉\inlinemahjong{12m}时,依然先切\inlinemahjong{1m}。不过,这种改良并不算太强,所以如果接下来摸到字牌等安全牌,就会在下一巡改切\inlinemahjong{2m}。\\
另外,若如牌25所示,手牌已确定是良形,就应先切安全度较低的一侧(通常是靠里面的那张)。\\
在牌26这样的手牌里,对\inlinemahjong{68m}和\inlinemahjong{68p}作比较时,保留与其他面子相连的搭子(连续形搭子)在后续改良里更有优势。连续形的威力是个具有普遍性的概念。\\
至于牌27这种情况,若摸\inlinemahjong{5m}会变成\inlinemahjong{4m}\inlinemahjong{5m}\inlinemahjong{7m}\inlinemahjong{8m}的重复进张,而摸\inlinemahjong{5p}更有利。所以要舍弃与其他搭子“距离更近”的那个搭子,让后续的改良空间更佳。

\section{}[h]
\begin{figure}
    \caption{注意后续的变化与改良}
    \label{lec6:pai28-29}
    \tekiri{28}{45689m1189p22256s}{1p}{9m}{切\inlinemahjong{9m}后摸\inlinemahjong{7m}比切\inlinemahjong{9p}后摸\inlinemahjong{7p}要好}
    \par\bigskip
    \tekiri{29}{23455m88p2356678s}{1p}{5m}{变成浮牌附加形保留万子的改良}
\end{figure}

牌28中,虽然两处都是边张,但如果先切\inlinemahjong{9m},再摸进\inlinemahjong{7m},就会留下振听的三面张,反而使牌型更往前推进。这样后续跟进也更灵活,因此选择保留饼子的边张。

牌29中,比起切\inlinemahjong{8p},更推荐切\inlinemahjong{5m}。当万子变成两面形状后,可以顺势解除索子部分的重复进张,因此考虑到一手后的改良这样切更好。像这样先切掉一张对子,让浮牌有机会改良的打法,被称作“做浮牌”。在此牌形中,若不切\inlinemahjong{5m}或\inlinemahjong{8p},其余选择都不及做浮牌更有利。

\begin{figure}[h]
    \caption{优先考虑改良质量而不是数量}
    \label{lec6:pai30-31}
    \tekiri{30}{12356678m45p4588s}{8p}{6m}{摸到里目时也能提升手牌的价值}
    \par\bigskip
    \tekiri{31}{5688m12356789p67s}{4z}{5m}{比起顾虑里目优先确定手役}
\end{figure}
\hyperlink{lec6:pai30-31}{牌30}中,如果保留万子的连续形搭子,确实会得到摸\inlinemahjong{5689m}等多种变化。然而,比起这些,更重要的是在摸进\inlinemahjong{4m}时能附带断幺而获得更高打点,所以选择切\inlinemahjong{6m}。若摸到\inlinemahjong{9m},则能兼顾一气,通过舍弃\inlinemahjong{45p}或\inlinemahjong{45s}来转变。像这样,当摸到里目\footnote{指如切掉一个搭子后没有摸到其他搭子的进张却摸到了切掉的搭子的进张这样的情况}时也能提升手牌价值的打牌,实质上就是在扩大自己的有効牌范围。

容易产生混淆的是牌31。这里涉及到在567三色以及筒子的一气之间的选择。若想完成三色,需要\inlinemahjong{7m}和\inlinemahjong{5s}这2张,但如果要做筒子的一气,只需要\inlinemahjong{4p}这一张即可,因此选择追求一气更有利。

\section{最大限度地追求变化}
\begin{figure}[h]
    \caption{最大限度地追求变化}
    \label{lec6:pai32}
    \tekiri{32}{1244789m44499p24s}{4z}{2s}{比起立直nomi,从一开始就要寻求手役}
\end{figure}
\hyperlink{lec6:pai32}{牌32}中,若要让进张面最广,可以选择切\inlinemahjong{1m},但听牌后大概率只是愚形立直nomi。与其之后再寻求改良,不如在这个阶段就考虑\inlinemahjong{56m}上张后朝一气发展,因此这里选择切\inlinemahjong{2s}。

\section{平和的进张与其他手役及改良的比较}
\begin{figure}[h]
    \caption{平和的进张与其他手役及改良的比较}
    \label{lec6:pai33-37}
    \tekiri{33}{1335m33567p34789s}{1s}{5m}{如果优先改良则打\inlinemahjong{1m}}
    \par\bigskip
    \tekiri{34}{1335m22567p34789s}{1s}{3m}{确定良形听牌的改良只有\inlinemahjong{3p}所以打\inlinemahjong{3m}}
    \par\bigskip
    \tekiri{35}{1335m11567p34678s}{1s}{1m}{如果摸到\inlinemahjong{6m}就有断平,所以打\inlinemahjong{1m}}
    \par\bigskip
    \tekiri{36}{2445667m234p1178s}{8s}{2m}{优先满贯以上的进张寻求一杯口}
    \par\bigskip
    \tekiri{37}{2456677m234p1178s}{3m}{2m}{对于万子的复合形的处理方法取决于打点上的要求}
\end{figure}

平和因为没有符,即使同为1番役,往往也比其他1番役稍显逊色。因此,在实际对局中,与其急着做平和,不如优先考虑改良或等待更好的听牌形的情况。

\hyperlink{lec6:pai33-37}{牌33}与其为了立即做平和而切\inlinemahjong{3m}把万子部分处理成两坎张,不如优先考虑摸\inlinemahjong{24p}时保证形成良形的可能性。
不过,与摸到\inlinemahjong{6m}的改良相比,更重视摸到\inlinemahjong{25s}时能让\inlinemahjong{2m}骗筋的可能性,所以这里选择切\inlinemahjong{5m}。\\
对于类似的手牌:牌34中,如果只有摸到\inlinemahjong{3p}才能获得良形听牌的改良,就可以切\inlinemahjong{3m}来接受两面搭子的平和进张;牌35中,如果摸到\inlinemahjong{6m}就能做成断幺九平和,因此更加重视这种强力的改良选择切\inlinemahjong{1m}。\\
牌36如果切\inlinemahjong{4m},可以确定平和,但为了优先考虑摸到\inlinemahjong{5m}后能形成一杯口(イーペーコー)并达到满贯,所以改为切\inlinemahjong{2m}。假设\inlinemahjong{1s}宝牌的话,那么摸到\inlinemahjong{3m}也能达到满贯,所以那时就会选择切\inlinemahjong{4m}。这就是\ref{lec1:rule5}所说的:“满贯以上的进张要优先考虑”。\\
对于牌37来说,由于摸到\inlinemahjong{69s}依然可以继续保留一杯口,所以即使宝牌是\inlinemahjong{3m},也会切\inlinemahjong{2m}。如果宝牌变成\inlinemahjong{4m},则一杯口无法成立,就要改为切\inlinemahjong{7m}。
\chapter[【一向听的技术】两面形的选择 其二]{从复合搭子中切一张时的解说}

\section{搭子固定与雀头固定}
\begin{figure}[h]
    \caption{搭子固定与雀头固定}
    \label{lec7:pai1-2}
    \tekiri{1}{12367m123667p556s}{1m}{6s}{比起固定搭子更应该先固定雀头}
    \par\bigskip
    \tekiri{2}{788m23488p223789s}{4z}{8m}{留下789三色的变化}
\end{figure}\hyperref[lec7:pai1-2]{牌1}
这种牌型,如果切\inlinemahjong{6p}或\inlinemahjong{5s},就是“搭子固定”;若切\inlinemahjong{7p}或\inlinemahjong{6s},则是“雀头固定”。与其将两面搭子固定,不如优先将雀头固定来扩大进张面。如果像打\inlinemahjong{6p}那样把两面搭子固定起来,确实抓到\inlinemahjong{4s}或\inlinemahjong{7s}后可以完成一个面子,从而获得较大的牌型变化,但相比之下,更应该优先考虑听牌速度。只要能清楚区分“作为雀头的对子”和“作为面子候补的对子”,就不会像这样错切\inlinemahjong{6p}而漏掉更好的进张面。与其等\inlinemahjong{47s},不如保留\inlinemahjong{58p}(无论是听牌后的出牌率还是红宝牌接受度都更好),因此选择打\inlinemahjong{6s}。
\hyperref[lec7:pai1-2]{牌2}
这里是“面子候补固定”的思路,需要在打\inlinemahjong{8m}和打\inlinemahjong{2s}之间作选择。乍看之下似乎差别不大,但如果先切\inlinemahjong{8m},那么抓到\inlinemahjong{7p}时,就可以通过切\inlinemahjong{3s}、把\inlinemahjong{2s}固定为雀头,从而追求789的三色同顺。

当雀头候补发生变化时,往往会出现容易被忽视的手变可能性。遇到“似乎切哪张都一样”的情况时,不妨再多加确认,看看是否真的完全相同。

\section{优先进张面而非防守}

\begin{figure}[h]
    \caption{比起防守更注重进张}
    \label{lec7:pai3}
    \tekiri{3}{112367m123667p5s4z}{5s}{4z}{完全一向听的进张}
\end{figure}
\begin{figure}[h]
    \caption{增加的进张面价值很低}
    \label{lec7:pai4}
    \tekiri{4}{12367m123667p55s4z}{7p}{6p}{先切宝牌指示牌的\inlinemahjong{6p}}
\end{figure}
\begin{figure}[h]
    \caption{不同的搭子固定方式的比较}
    \label{lec7:pai5-7}
    \tekiri{5}{223m44599p123s555z}{1s}{2m}{\inlinemahjong{14m}比起\inlinemahjong{36p}和牌率更高}
    \par\bigskip
    \tekiri{6}{33456789m334p788s}{4z}{3p}{有多种能够断幺的改良}
    \par\bigskip
    \tekiri{7}{122789m34599p446s}{4z}{4s}{比起对子保留坎张。注意两面的改良留下\inlinemahjong{5s}而不是\inlinemahjong{3m}}
\end{figure}

在不缩小进张面的前提下,基本上就不必特地保留安全牌。\hyperref[lec7:pai3]{牌3}为例,与其切\inlinemahjong{6p}来留一张安全牌,不如直接切\inlinemahjong{4z}。

不过,若换\hyperref[lec7:pai4]{牌4},因为抓到\inlinemahjong{6p}或\inlinemahjong{5s}之后会导致必须切出宝牌,所以先切\inlinemahjong{6p}。
即便因为抓到\inlinemahjong{5s}而错过了听牌,也能形成比立直nomi更高打点、而且进张面更广的一向听。


\section{比较不同的搭子固定方式}
固定搭子时的基本思路是:在保留高价值进张面的同时,让留下的搭子也能最大化地提升价值。
\hyperref[lec7:pai5-7]{牌5}中,\inlinemahjong{14m}的两面搭子比\inlinemahjong{63p}更靠近边张,和牌成功率相对更高,因此我们选择固定\inlinemahjong{14m}\inlinemahjong{4m}这组两面。而保留相对较弱的\inlinemahjong{445p}。
\hyperref[lec7:pai5-7]{牌6}虽然可以为了\inlinemahjong{0p}这1张进张而在边张时获得稍高的和牌率直接切\inlinemahjong{8s};不过,在当前这副手牌里,若摸到\inlinemahjong{6m}并切\inlinemahjong{9m},可以转成断幺九,再加上如果摸到\inlinemahjong{4m}并切\inlinemahjong{7s},还能形成包含高目一杯口(在内的断幺九变化,因此这里选择切\inlinemahjong{3p}。

另外,\hyperref[lec7:pai5-7]{牌7}一样
若牌中已有3组以上的对子,则与其优先保留对子,不如保留坎张搭子,更能确保进张数量上的优势。

\section{保留高打点的进张}
\begin{figure}[h]
    \caption{保留高打点的进张}
    \label{lec7:pai8-10}
    \tekiri{8}{222566m235p11789s}{4z}{6m}{}
    \par\bigskip
    \tekiri{9}{111079m123p44667s}{4z}{9m}{}
    \par\bigskip
    \tekiri{10}{122888m34599p446s}{4z}{1m}{}
\end{figure}
根据\ref{lec1:rule4},即使牺牲进张的张数,也会有需要保留高打点进张的情况。

\hyperref[lec7:pai8-10]{牌8}中,虽然如果切\inlinemahjong{0p}可以获得最大的进张数,但这里选择切\inlinemahjong{6m}。这样就还能保留进\inlinemahjong{5p}或\inlinemahjong{6p}时的变化。接着如果摸到\inlinemahjong{1p}或\inlinemahjong{4p}都没问题;但若先完成万子的两面进张,则比起愚形40符2飜的坎张立直,只要良形立直更好,所以会切\inlinemahjong{0p}立直。

\hyperref[lec7:pai8-10]{牌9}即使是坎张立直,也能得到更高的打点,因此不切\inlinemahjong{6s},而是切\inlinemahjong{9m}。这样若自摸\inlinemahjong{46s}就会成为50符2飜立直,比良形立直nomi略有优势。不过在中盘以后,为了优先速度,会切\inlinemahjong{6s}。

\par\bigskip
\textbf{50符坎张立直1宝牌 > 立直nomi的两面 > 40符坎张立直1宝牌}
\par\bigskip

即使有3个以上对子,\hyperref[lec7:pai5-7]{牌7}不同,对于\hyperref[lec7:pai8-10]{牌10}那样能够看出三暗刻希望的手牌,也会保留三暗刻的可能性并固定对子。

\section{雀头固定之间的比较}
\begin{figure}[h]
    \caption{雀头固定之间的比较}
    \label{lec7:pai11-13}
    \tekiri{11}{244789m577p45s555z}{4z}{2m}{注意\inlinemahjong{4p}的改良}
    \par\bigskip
    \tekiri{12}{334568m45789p668s}{4z}{8m}{摸到里目也能让手牌前进}
    \par\bigskip
    \tekiri{13}{34568m68p45789s33z}{4z}{8p}{\hyperref[lec7:pai11-13]{牌12}不同。打\inlinemahjong{8p}留下连续形}
\end{figure}

雀头固定的基本原则,是将包括后续变化在内不容易做成面子的塔子固定下来。
例如\hyperref[lec7:pai11-13]{牌11}中,比较\inlinemahjong{244m}与\inlinemahjong{577p},因为\inlinemahjong{244m}更难变成两面,因此切\inlinemahjong{2m}。

\hyperref[lec7:pai11-13]{牌12}中,则是比较切\inlinemahjong{8m}与切\inlinemahjong{8s}。即使摸到\inlinemahjong{7m}导致错过聴牌,也通过打\inlinemahjong{8m}来保留振听三面张。
看似类似\hyperref[lec7:pai11-13]{牌13},则要保留连续形,因此切\inlinemahjong{8p}。

\section{复合塔子的进阶形}
\begin{figure}[h]
    \caption{存在烟囱形时}
    \label{lec7:pai14-16}
    \tekiri{14}{345556m66p223445s}{4z}{2s}{留下包含34555的烟囱形的三枚搭子}
    \par\bigskip
    \tekiri{15}{345556m67p223445s}{4z}{6m}{打\inlinemahjong{6m}和\inlinemahjong{6s}的进张数相同}
    \par\bigskip
    \tekiri{16}{344555m67p223445s}{4z}{4s}{考虑到自摸\inlinemahjong{4m}的可能性打\inlinemahjong{4s}更好}
\end{figure}

我们将来探讨一些较为复杂的复合塔子。首先是类似34555的“烟囱”形\hyperref[lec7:pai14-16]{牌14}中,由于345556这一形状的威力,相较于切\inlinemahjong{5m}或\inlinemahjong{6m},切\inlinemahjong{2s}能够获得更大的进张数量。

而\hyperref[lec7:pai14-16]{牌15}中,若切\inlinemahjong{6m}来固定万子,进张数量并不会减少,因此考虑到一盃口的可能性,切\inlinemahjong{6m}更为有利。

不过\hyperref[lec7:pai14-16]{牌16}里,如果在烟囱形之外还具备一盃口形时,切\inlinemahjong{4s}进张数量更多。

\par\bigskip\noindent\ignorespaces
\begin{figure}[h]
    \caption{存在33577形时}
    \label{lec7:pai17}
    \tekiri{17}{33577m345666p357s}{4z}{7s}{留下345三色}
\end{figure}
另一个常见的复合形是像33577这样的五张复合塔子,一旦切去其中任意一张,都将大幅减少进张数量。
\hyperref[lec7:pai17]{牌17}中,比起切\inlinemahjong{3m}或\inlinemahjong{7m},切\inlinemahjong{3s}或\inlinemahjong{7s}能够获得更多进张。考虑到345三色,最终选择切\inlinemahjong{7s}。



\begin{figure}[h]
    \caption{存在13556形时}
    \label{lec7:pai18}
    \tekiri{18}{13556m123789p556s}{1p}{6s}{偏重万子的进张}
\end{figure}
对\hyperref[lec7:pai18]{牌18},让索子做雀头固定,在进张数量和最终形状上都更为有利。这也是一个常见的形状。

\begin{figure}[h]
    \caption{存在复数的坎张对子时}
    \label{lec7:pai19}
    \tekiri{19}{33557m123789p557s}{6m}{7s}{因为宝牌是\inlinemahjong{6m}并且有一杯口可能所以打\inlinemahjong{7s}}
\end{figure}\hyperref[lec7:pai19]{牌19}是带有多个坎张对子的手牌。像3355这类含“跳对子”的复合塔子,如果固定塔子并切\inlinemahjong{5m},会错过摸\inlinemahjong{4m}时一杯口听牌的机会。切\inlinemahjong{7m}与切\inlinemahjong{7s}在进张数量上相同。

\begin{figure}[h]
    \caption{存在两翼形时}
    \label{lec7:pai20-21}
    \tekiri{20}{45m3456p23345667s}{4p}{6p}{如果为了断幺切\inlinemahjong{2s}则没有留\inlinemahjong{3p}的意义}
    \par\bigskip
    \tekiri{21}{33445667m789p778s}{9p}{7s}{万子已经提供了雀头,所以索子固定两面}
\end{figure}

有些复合塔子会包含雀头。

例如,从类似334566(除此之外没有雀头)的两翼形\footnote{指223455这样,有时还会直接指代在此两侧再附加一枚或两枚的形状},再往前推进两手后,便会出现13345668或33445667这种形状。虽然它们并非高频出现,但如果能将这类手形视为两翼形的进阶形来理解,就会在实际对局中更容易判断。
\hyperref[lec7:pai20-21]{牌20}的23345667是两翼形的典型进阶形。其特征在于,无论是摸14还是58,都能形成2个面子 + 雀头。在这手牌中,即使摸\inlinemahjong{1s}后断幺(タンヤオ)崩坏,只要还能保留高目三色,就更容易达到满贯以上(优先考虑满贯以上的进张),因此与其切\inlinemahjong{3p},选择切\inlinemahjong{6p}更好。
\hyperref[lec7:pai20-21]{牌21}同样是两翼形的进阶形
在这副牌里,如果摸\inlinemahjong{23458m},就能让万子变成「2个面子 + 雀头」。因此,通过切\inlinemahjong{7s}把索子固定成两面,能够获得最多的进张数量。

\par\bigskip
\begin{figure}[h]
    \caption{包含烟囱形的手牌}
    \label{lec7:pai22-24}
    \tekiri{22}{345556m12399p577s}{4z}{5m}{尽量最终形成两面听牌}
    \par\bigskip
    \tekiri{23}{345556m12399p077s}{4z}{7s}{为了把\inlinemahjong{0s}留到最后}
    \par\bigskip
    \tekiri{24}{345556m11199p577s}{4z}{6m}{保留摸到\inlinemahjong{6s}时的良形听牌和三暗刻的可能性}
\end{figure}
最后,举几个关于“当下进张数量”与“打点”如何取舍的例子。其中也包含之前谈到的“烟囱形”。
\hyperref[lec7:pai22-24]{牌22}中,
为了扩大进张面应该切\inlinemahjong{7s},但为了优先考虑听牌时的待牌强度,这里选择切\inlinemahjong{5m}。

\hyperref[lec7:pai22-24]{牌23}这里,则是打算将\inlinemahjong{0s}用到底,因此选择切\inlinemahjong{7s}。

至\hyperref[lec7:pai22-24]{牌24},
则是通过切\inlinemahjong{6m}保留在摸\inlinemahjong{6s}时的良形听牌与三暗刻的可能性。


\vspace*{\fill}
\begin{tcolorbox}[
    title={福地的碎碎念}, fonttitle=\bfseries\Large
]
我认为这次的课程内容相当基础。这里所说的“基础”并不是“简单”的意思,而是指能在根本上支撑整套打法的重要内容。如果有人觉得这些内容并不是理所当然,那么反复多读几遍、把里面的内容彻底掌握之后,恐怕对牌效率的理解会有明显提升。这也是本书相当高质量的部分之一。

至于那些容易忽略的役种可能性,我觉得对成绩的影响并不大,不\hyperref[lec7:pai11-13]{牌11}\hyperref[lec7:pai11-13]{牌13}里的做法,最好是能做到闭着眼都毫无疑问地处理才好。

不过呢,这也可以算是对我自己写的书的一个吐槽:\hyperref[lec7:pai5-7]{牌5}那样的讨论终究是纸上谈兵。因为在实际对局中,一旦进入中盘,就一定会出现\inlinemahjong{12m}再场上的枚数差异,人们会根据那些实际切出去的张数来决定打法。毕竟没有完全没有边张的场况。

另外,我跟ネマタ在大方向上也有很大不同。还是\hyperref[lec7:pai5-7]{牌5}来说,我经常会切\inlinemahjong{4p},原因是我不想在中盘之后,留下456这类牌作为“多余的手牌”。我宁愿最后剩下2378,从危险度上来说更好。
\hyperref[lec7:pai3]{牌3}也一样,我会因为讨厌听牌时多出的那张\inlinemahjong{6p},而倾向于保留\inlinemahjong{4z},切\inlinemahjong{6p}。

我和ネマタ在主要打牌的场合不同。ネマタ主要打网络麻将,所以往往走“先制立直”或“弃和”的两极化道路;而我是在歌舞伎町的自由雀庄,人们通常不会轻易弃和,而是激烈对攻。这样的场合差异,也会反映在组牌思路上。

至于听牌时把等待尽量往外扩以占优势,还是尽量让听牌时多余的牌危险度更低来降低风险,究竟哪个更有利,目前并没有定论。这里就只能看你更想相信哪种思路了。

\end{tcolorbox}


\end{document}
