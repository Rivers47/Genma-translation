\chapter[【一向听的技术】两面形的选择 其一]{一向听的最常见形状:两面形的处理}

手牌可分为五种面子、雀头、搭子、复合搭子、浮牌五种组成成分。
其中,已经完成的面子和作为雀头的对子原则上是不会切的,因此大多数情况下的打牌可分为以下三类:\begin{enumerate}
    \item 切搭子
    \item 从复合搭子中切一张(固定搭子、固定雀头)
    \item 舍弃浮牌
\end{enumerate}

\section{在听牌时要尽量保留更易和牌的搭子}
人们常说“坎张优于边张”,但由于更容易先进入听牌,所以\textbf{比起追求改良,更应优先考虑听牌后的和牌难易度。}

\begin{figure}[h]
    \caption{优先容易和牌的搭子而不是改良}
    \label{lec6:pai1}
    \tekiri{1}{1268m5699p234999s}{9p}{8m}{比起两面改良优先骗筋}
\end{figure}

\hyperref[lec6:pai1]{牌1}与其期待摸进\inlinemahjong{5m}后形成两面,不如切\inlinemahjong{6m}形成骗筋。若差距不大(类似无筋37与无筋456之差),就倾向于舍弃搭子有更多改良的搭子。

\begin{figure}[h]
    \caption{比起改良优先}
    \label{lec6:pai2-4}
    \tekiri{2}{1146m1134578p666s}{5p}{4m}{优先幺九的双碰}
    \par\bigskip
    \tekiri{3}{1146m1134578p666s}{5m}{1p}{优先宝牌进张}
    \par\bigskip
    \tekiri{4}{1166m1134578p666s}{5m}{1m}{优先提高打点的改良}
\end{figure}
\hyperref[lec6:pai2-4]{牌2}切掉对子、保留坎张更容易转变成两面,且有机会摸到最理想的\inlinemahjong{0m}。然而,比起摸进\inlinemahjong{5m}进入听牌,更常出现的是摸进\inlinemahjong{69p}后听牌;此时边张的双碰比坎听\inlinemahjong{5m}要容易和牌得多(并且因为之前切了\inlinemahjong{4m},\inlinemahjong{1m}也能骗筋)。由于\inlinemahjong{0m}只有一枚,因此更推荐舍弃坎张。就算摸进\inlinemahjong{3m},也可扩大进张,所以切\inlinemahjong{4m}。\\
\hyperref[lec6:pai2-4]{牌3}
这里\inlinemahjong{5m}是宝牌,优先考虑保留宝牌的进张。若摸进\inlinemahjong{2p}则确立两面,且能形成高价的一气通贯,比起只等坎张宝牌的面子候补更强,所以选择切\inlinemahjong{1p}。\\
\hyperlink{lec6:pai2-4}{牌4}同样为了优先保留宝牌进张与一气改良,而选择切\inlinemahjong{1m}。

\section{保留高打点的搭子}
\begin{figure}[h]
    \caption{优先高打点而不是改良}
    \label{lec6:pai5}
    \tekiri{5}{12346789m46p2267s}{7s}{4p}{优先一气通贯}
\end{figure}
\hyperlink{lec6:pai5}{牌5}中,切\inlinemahjong{1m}在会保留后续断幺九的可能,但眼前的一通打点更高,因此优先选择这一手。再加上还有678三色的改良,所以先从\inlinemahjong{4p}开始切(在进张数量无差的情况下,就以后续改良可能性来决定)。

\section{尽量保留复合搭子}
\begin{figure}[h]
    \caption{比起改良优先和牌率}
    \label{lec6:pai6}
    \tekiri{6}{1133m34567p56799s}{5s}{3m}{万子变为复合搭子。基本上应把复合搭子做成奇数枚。}
\end{figure}
\hyperlink{lec6:pai6}{牌6}中,万子里作为搭子(对子)部分的牌有4张,切掉其中1张就能让复合搭子保留,从而在总体进张上占优。若切\inlinemahjong{1m},确实能在摸进\inlinemahjong{4m}时变成完全一向听,不过相比之下,更重视听牌后的和牌难易度,因此选择打\inlinemahjong{3m}。

\begin{figure}[h]
    \caption{存在复合进张的场合}
    \label{lec6:pai7-8}
    \tekiri{7}{123m2245667p5699s}{5s}{2p}{不要漏看\inlinemahjong{3p}的坎张}
    \par\bigskip
    \tekiri{8}{5699p2334567788s}{5s}{8s}{略难,打\inlinemahjong{8s}后233456778自摸\inlinemahjong{1469s}即可平和听牌}
\end{figure}
\begin{figure}
    \caption{尽量提高当前的打点}
    \label{lec6:pai9-12}
    \tekiri{9}{1223446m345p1178s}{8s}{1m}{切\inlinemahjong{2m}进张多一枚,但优先一杯口}
    \par\bigskip
    \tekiri{10}{1223446m345p78s77z}{8s}{6m}{切\inlinemahjong{2m}进张多一枚,但优先一杯口}
    \par\bigskip
    \tekiri{11}{1223446789m1178s}{8s}{4m}{带一气的平和}
    \par\bigskip
    \tekiri{12}{1223446789m78s77z}{8s}{2m}{\inlinemahjong{7z}或一气的可能性}
\end{figure}

像\hyperlink{lec6:pai7-8}{牌7}那样,通过切\inlinemahjong{2p}便能获得\inlinemahjong{3p}进张的手牌十分常见。对于245667这类两面坎张(358进张)很容易被忽略,需要格外注意。\\
\hyperlink{lec6:pai7-8}{牌8}所示,也有能形成两面复合进张的形状。若有多种方法能保留复合搭子,则优先考虑当下打点更高的方案。\hyperlink{lec6:pai9-12}{牌9到牌12}中,都是为了保留手役的可能性而作出的打牌选择,属于在多种可选打牌中选出最优先的一种情况。

\section{优先接近和牌阶段的进张,而非眼前的进张}
\begin{figure}
    \caption{优先两面以上的进张}
    \label{lec6:pai13-14}
    \tekiri{13}{57m4455p12345699s}{4z}{7m}{尽量两面听牌}
    \par\bigskip
    \tekiri{14}{57m3445667p12399s}{4z}{5m}{尽量两面以上听牌}
\end{figure}
\begin{figure}
    \caption{比起进张数优先进张的强度}
    \label{lec6:pai15}
    \tekiri{15}{1245m12789p34588s}{3s}{1p}{(\inlinemahjong{3p}2现)所以选择看上去更难受但枚数更多的一边}
\end{figure}
\hyperlink{lec6:pai13-14}{牌13-14}虽然切\inlinemahjong{4p}时能获得更多听牌进张,但更看重能形成两面以上听牌的进张数量。
在牌13中,为了考虑456三色的改良而选择切\inlinemahjong{7m}。在牌14中,万子部分变动后能形成端靠边的两面,更利于和牌,所以切\inlinemahjong{5m}。\\
\hyperlink{lec6:pai15}{牌15}中,\inlinemahjong{3m}的进张与其他部分重叠。虽然连续切\inlinemahjong{1m}\inlinemahjong{2m}可以增加总体进张数量,但为了避免出现(只剩2枚的)边张\inlinemahjong{3p}听牌,优先考虑最终听牌的强度,所以选择切\inlinemahjong{1p}。

\section{比进张数量更优先考虑高打点的进张}
\begin{figure}[h]
    \caption{比进张数量更优先考虑高打点的进张}
    \label{lec6:pai16-21}
    \tekiri{16}{34m2245667p56799s}{4z}{9s}{比起坎\inlinemahjong{3p}的进张选择断幺}
    \par\bigskip
    \tekiri{17}{4566677m4588p999s}{4z}{4m}{一杯口和三暗刻的可能性}
    \par\bigskip
    \tekiri{18}{1133m23499p23479s}{2p}{9s}{一杯口的可能性}
    \par\bigskip
    \tekiri{19}{1245689m34789p22s}{4z}{4p}{一气>平和}
    \par\bigskip
    \tekiri{20}{1256789m34789p22s}{4z}{4p}{一气>平和,但打\inlinemahjong{2m}也可以,如果有1宝牌则打\inlinemahjong{2m}}
    \par\bigskip
    \tekiri{21}{1345689m34789p22s}{4z}{9m}{平和>一气}
\end{figure}

\hyperlink{lec6:pai16-21}{牌16}中,虽然切\inlinemahjong{2p}能获得更多进张,但为了向断幺发展而优先选择该进张。
同理,牌17中虽然切\inlinemahjong{7m}进张面最广,但更应该优先一杯口和三暗刻的机会。\\
在牌18容易被忽略的情况是:若摸\inlinemahjong{2m}可形成一杯口,所以切\inlinemahjong{9s}(若摸\inlinemahjong{6s}则优先保留更佳形状,改切\inlinemahjong{3m})。\\
牌19里,即使先完成愚形,也因为\textbf{愚形立直一气>平和立直},所以会舍弃两面。\\
牌20中,舍弃边张可确保形成良形,虽然一气尚不确定,但仍然是放弃两面更好。如果\inlinemahjong{9m}是宝牌,那么在摸\inlinemahjong{4m}的情况下也是\textbf{立平宝1>愚形立直一气宝1},所以选择舍弃边张。\\
即便如此,牌21中,如果切\inlinemahjong{9m},就会留下134568这类两坎张形状,进张数量差距明显,因此舍弃一气。

\section{两面以上、满贯以上时优先考虑进张数量}
\begin{figure}[h]
    \caption{打点已经足够高所以优先考虑进张数}
    \label{lec6:pai22}
    \tekiri{22}{4566677m3488p567s}{8p}{7m}{自摸\inlinemahjong{3m}也有满贯所以优先进张数打\inlinemahjong{7m}}
\end{figure}
先前提到的例子中,往往是为了打点而牺牲进张数量的情况比较多;但是如果能确保“两面以上、满贯以上”的条件时,就会优先考虑进张数量。\\
\hyperlink{lec6:pai22}{牌22}中,有2枚宝牌,且断幺九已经确定。与其去考虑一杯口,不如优先保留更多的进张数量。\\
\begin{figure}[h]
    \caption{优先两面以上的进张}
    \label{lec6:pai23}
    \tekiri{23}{34678m56799p3467s}{4z}{4s}{保留678三色可能的同时扩大进张面}
\end{figure}
\hyperlink{lec6:pai23}{牌23}中,与其追求摸\inlinemahjong{5s}后形成三面听,不如先优先让自己两面听牌的进张更多,故选择切\inlinemahjong{4s},并且还能保留带有高打点的\inlinemahjong{8s}三色改良。
若宝牌是\inlinemahjong{5s},就要更重视高打点的进张,切\inlinemahjong{4m}。
另外,若宝牌是\inlinemahjong{5s}且\inlinemahjong{5p}为红5,由于能到满贯以上,此时仍优先进张数量,改切\inlinemahjong{4s}。

\section{考虑改良}
\begin{figure}
    \caption{注意改良}
    \label{lec6:pai24-26}
    \tekiri{24}{1279m34578p12366s}{1s}{1m}{从边张的外侧开始切}
    \par\bigskip
    \tekiri{25}{3479m345888p2256s}{4z}{7m}{因为已经确定良形所以从内测开始}
    \par\bigskip
    \tekiri{26}{34568m1168p22256s}{1p}{8p}{留下连续形}
    \par\bigskip
    \tekiri{27}{2478m1168p222567s}{1p}{2m}{比起自摸\inlinemahjong{5m}自摸\inlinemahjong{5p}更好}
\end{figure}
在舍弃搭子时,基本会先观察是否能变成更好的形状,因此通常从外侧开始切。
\hyperlink{lec6:pai24-26}{牌24}中,先切\inlinemahjong{1m};若随后摸进\inlinemahjong{3m},则会留下振听形状并切\inlinemahjong{9m}。如果先摸进\inlinemahjong{14m},就不会变成振听;就算最后留着振听,也不比坎\inlinemahjong{8m}的听牌更糟。因此当要切掉\inlinemahjong{12m}时,依然先切\inlinemahjong{1m}。不过,这种改良并不算太强,所以如果接下来摸到字牌等安全牌,就会在下一巡改切\inlinemahjong{2m}。\\
另外,若如牌25所示,手牌已确定是良形,就应先切安全度较低的一侧(通常是靠里面的那张)。\\
在牌26这样的手牌里,对\inlinemahjong{68m}和\inlinemahjong{68p}作比较时,保留与其他面子相连的搭子(连续形搭子)在后续改良里更有优势。连续形的威力是个具有普遍性的概念。\\
至于牌27这种情况,若摸\inlinemahjong{5m}会变成\inlinemahjong{4m}\inlinemahjong{5m}\inlinemahjong{7m}\inlinemahjong{8m}的重复进张,而摸\inlinemahjong{5p}更有利。所以要舍弃与其他搭子“距离更近”的那个搭子,让后续的改良空间更佳。

\section{}[h]
\begin{figure}
    \caption{注意后续的变化与改良}
    \label{lec6:pai28-29}
    \tekiri{28}{45689m1189p22256s}{1p}{9m}{切\inlinemahjong{9m}后摸\inlinemahjong{7m}比切\inlinemahjong{9p}后摸\inlinemahjong{7p}要好}
    \par\bigskip
    \tekiri{29}{23455m88p2356678s}{1p}{5m}{变成浮牌附加形保留万子的改良}
\end{figure}

牌28中,虽然两处都是边张,但如果先切\inlinemahjong{9m},再摸进\inlinemahjong{7m},就会留下振听的三面张,反而使牌型更往前推进。这样后续跟进也更灵活,因此选择保留饼子的边张。

牌29中,比起切\inlinemahjong{8p},更推荐切\inlinemahjong{5m}。当万子变成两面形状后,可以顺势解除索子部分的重复进张,因此考虑到一手后的改良这样切更好。像这样先切掉一张对子,让浮牌有机会改良的打法,被称作“做浮牌”。在此牌形中,若不切\inlinemahjong{5m}或\inlinemahjong{8p},其余选择都不及做浮牌更有利。

\begin{figure}[h]
    \caption{优先考虑改良质量而不是数量}
    \label{lec6:pai30-31}
    \tekiri{30}{12356678m45p4588s}{8p}{6m}{摸到里目时也能提升手牌的价值}
    \par\bigskip
    \tekiri{31}{5688m12356789p67s}{4z}{5m}{比起顾虑里目优先确定手役}
\end{figure}
\hyperlink{lec6:pai30-31}{牌30}中,如果保留万子的连续形搭子,确实会得到摸\inlinemahjong{5689m}等多种变化。然而,比起这些,更重要的是在摸进\inlinemahjong{4m}时能附带断幺而获得更高打点,所以选择切\inlinemahjong{6m}。若摸到\inlinemahjong{9m},则能兼顾一气,通过舍弃\inlinemahjong{45p}或\inlinemahjong{45s}来转变。像这样,当摸到里目\footnote{指如切掉一个搭子后没有摸到其他搭子的进张却摸到了切掉的搭子的进张这样的情况}时也能提升手牌价值的打牌,实质上就是在扩大自己的有効牌范围。

容易产生混淆的是牌31。这里涉及到在567三色以及筒子的一气之间的选择。若想完成三色,需要\inlinemahjong{7m}和\inlinemahjong{5s}这2张,但如果要做筒子的一气,只需要\inlinemahjong{4p}这一张即可,因此选择追求一气更有利。

\section{最大限度地追求变化}
\begin{figure}[h]
    \caption{最大限度地追求变化}
    \label{lec6:pai32}
    \tekiri{32}{1244789m44499p24s}{4z}{2s}{比起立直nomi,从一开始就要寻求手役}
\end{figure}
\hyperlink{lec6:pai32}{牌32}中,若要让进张面最广,可以选择切\inlinemahjong{1m},但听牌后大概率只是愚形立直nomi。与其之后再寻求改良,不如在这个阶段就考虑\inlinemahjong{56m}上张后朝一气发展,因此这里选择切\inlinemahjong{2s}。

\section{平和的进张与其他手役及改良的比较}
\begin{figure}[h]
    \caption{平和的进张与其他手役及改良的比较}
    \label{lec6:pai33-37}
    \tekiri{33}{1335m33567p34789s}{1s}{5m}{如果优先改良则打\inlinemahjong{1m}}
    \par\bigskip
    \tekiri{34}{1335m22567p34789s}{1s}{3m}{确定良形听牌的改良只有\inlinemahjong{3p}所以打\inlinemahjong{3m}}
    \par\bigskip
    \tekiri{35}{1335m11567p34678s}{1s}{1m}{如果摸到\inlinemahjong{6m}就有断平,所以打\inlinemahjong{1m}}
    \par\bigskip
    \tekiri{36}{2445667m234p1178s}{8s}{2m}{优先满贯以上的进张寻求一杯口}
    \par\bigskip
    \tekiri{37}{2456677m234p1178s}{3m}{2m}{对于万子的复合形的处理方法取决于打点上的要求}
\end{figure}

平和因为没有符,即使同为1番役,往往也比其他1番役稍显逊色。因此,在实际对局中,与其急着做平和,不如优先考虑改良或等待更好的听牌形的情况。

\hyperlink{lec6:pai33-37}{牌33}与其为了立即做平和而切\inlinemahjong{3m}把万子部分处理成两坎张,不如优先考虑摸\inlinemahjong{24p}时保证形成良形的可能性。
不过,与摸到\inlinemahjong{6m}的改良相比,更重视摸到\inlinemahjong{25s}时能让\inlinemahjong{2m}骗筋的可能性,所以这里选择切\inlinemahjong{5m}。\\
对于类似的手牌:牌34中,如果只有摸到\inlinemahjong{3p}才能获得良形听牌的改良,就可以切\inlinemahjong{3m}来接受两面搭子的平和进张;牌35中,如果摸到\inlinemahjong{6m}就能做成断幺九平和,因此更加重视这种强力的改良选择切\inlinemahjong{1m}。\\
牌36如果切\inlinemahjong{4m},可以确定平和,但为了优先考虑摸到\inlinemahjong{5m}后能形成一杯口(イーペーコー)并达到满贯,所以改为切\inlinemahjong{2m}。假设\inlinemahjong{1s}宝牌的话,那么摸到\inlinemahjong{3m}也能达到满贯,所以那时就会选择切\inlinemahjong{4m}。这就是\ref{lec1:rule5}所说的:“满贯以上的进张要优先考虑”。\\
对于牌37来说,由于摸到\inlinemahjong{69s}依然可以继续保留一杯口,所以即使宝牌是\inlinemahjong{3m},也会切\inlinemahjong{2m}。如果宝牌变成\inlinemahjong{4m},则一杯口无法成立,就要改为切\inlinemahjong{7m}。