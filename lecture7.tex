\chapter[【一向听的技术】两面形的选择 其二]{从复合搭子中切一张时的解说}

\section{搭子固定与雀头固定}
\begin{figure}[h]
    \caption{搭子固定与雀头固定}
    \label{lec7:pai1-2}
    \tekiri{1}{12367m123667p556s}{1m}{6s}{比起固定搭子更应该先固定雀头}
    \par\bigskip
    \tekiri{2}{788m23488p223789s}{4z}{8m}{留下789三色的变化}
\end{figure}\hyperref[lec7:pai1-2]{牌1}
这种牌型,如果切\inlinemahjong{6p}或\inlinemahjong{5s},就是“搭子固定”;若切\inlinemahjong{7p}或\inlinemahjong{6s},则是“雀头固定”。与其将两面搭子固定,不如优先将雀头固定来扩大进张面。如果像打\inlinemahjong{6p}那样把两面搭子固定起来,确实抓到\inlinemahjong{4s}或\inlinemahjong{7s}后可以完成一个面子,从而获得较大的牌型变化,但相比之下,更应该优先考虑听牌速度。只要能清楚区分“作为雀头的对子”和“作为面子候补的对子”,就不会像这样错切\inlinemahjong{6p}而漏掉更好的进张面。与其等\inlinemahjong{47s},不如保留\inlinemahjong{58p}(无论是听牌后的出牌率还是红宝牌接受度都更好),因此选择打\inlinemahjong{6s}。
\hyperref[lec7:pai1-2]{牌2}
这里是“面子候补固定”的思路,需要在打\inlinemahjong{8m}和打\inlinemahjong{2s}之间作选择。乍看之下似乎差别不大,但如果先切\inlinemahjong{8m},那么抓到\inlinemahjong{7p}时,就可以通过切\inlinemahjong{3s}、把\inlinemahjong{2s}固定为雀头,从而追求789的三色同顺。

当雀头候补发生变化时,往往会出现容易被忽视的手变可能性。遇到“似乎切哪张都一样”的情况时,不妨再多加确认,看看是否真的完全相同。

\section{优先进张面而非防守}

\begin{figure}[h]
    \caption{比起防守更注重进张}
    \label{lec7:pai3}
    \tekiri{3}{112367m123667p5s4z}{5s}{4z}{完全一向听的进张}
\end{figure}
\begin{figure}[h]
    \caption{增加的进张面价值很低}
    \label{lec7:pai4}
    \tekiri{4}{12367m123667p55s4z}{7p}{6p}{先切宝牌指示牌的\inlinemahjong{6p}}
\end{figure}
\begin{figure}[h]
    \caption{不同的搭子固定方式的比较}
    \label{lec7:pai5-7}
    \tekiri{5}{223m44599p123s555z}{1s}{2m}{\inlinemahjong{14m}比起\inlinemahjong{36p}和牌率更高}
    \par\bigskip
    \tekiri{6}{33456789m334p788s}{4z}{3p}{有多种能够断幺的改良}
    \par\bigskip
    \tekiri{7}{122789m34599p446s}{4z}{4s}{比起对子保留坎张。注意两面的改良留下\inlinemahjong{5s}而不是\inlinemahjong{3m}}
\end{figure}

没有必要不惜缩小进张面也要特地保留安全牌。\hyperref[lec7:pai3]{牌3}为例,与其切\inlinemahjong{6p}来留一张安全牌,不如直接切\inlinemahjong{4z}。

不过,若换\hyperref[lec7:pai4]{牌4},因为抓到\inlinemahjong{6p}或\inlinemahjong{5s}之后会导致必须切出宝牌,所以先切\inlinemahjong{6p}。
即便因为抓到\inlinemahjong{5s}而错过了听牌,也能形成比立直nomi更高打点、而且进张面更广的一向听。


\section{比较不同的搭子固定方式}
固定搭子时的基本思路是:在保留高价值进张面的同时,让留下的搭子也能最大化地提升价值。
\hyperref[lec7:pai5-7]{牌5}中,\inlinemahjong{14m}的两面搭子比\inlinemahjong{63p}更靠近边张,和牌成功率相对更高,因此我们选择固定\inlinemahjong{14m}这组两面。而保留相对较弱的\inlinemahjong{445p}。
\hyperref[lec7:pai5-7]{牌6}虽然可以为了\inlinemahjong{0p}这1张进张而在边张时获得稍高的和牌率直接切\inlinemahjong{8s};不过,在当前这副手牌里,若摸到\inlinemahjong{6m}并切\inlinemahjong{9m},可以转成断幺九,再加上如果摸到\inlinemahjong{4m}并切\inlinemahjong{7s},还能形成包含高目一杯口在内的断幺九变化,因此这里选择切\inlinemahjong{3p}。

另外,\hyperref[lec7:pai5-7]{牌7}一样
若牌中已有3组以上的对子,则与其优先保留对子,不如保留坎张搭子,更能确保进张数量上的优势。

\section{保留高打点的进张}
\begin{figure}[h]
    \caption{保留高打点的进张}
    \label{lec7:pai8-10}
    \tekiri{8}{222566m230p11789s}{4z}{6m}{}
    \par\bigskip
    \tekiri{9}{111079m123p44667s}{4z}{9m}{}
    \par\bigskip
    \tekiri{10}{122888m34599p446s}{4z}{1m}{}
\end{figure}
根据\ref{lec1:rule4},即使牺牲进张的张数,也会有需要保留高打点进张的情况。

\hyperref[lec7:pai8-10]{牌8}中,虽然如果切\inlinemahjong{0p}可以获得最大的进张数,但这里选择切\inlinemahjong{6m}。这样就还能保留进\inlinemahjong{5p}或\inlinemahjong{6p}时的变化。接着如果摸到\inlinemahjong{1p}或\inlinemahjong{4p}都没问题;但若先完成万子的两面进张,则比起愚形40符2飜的坎张立直,良形立直更好,所以会切\inlinemahjong{0p}立直。

\hyperref[lec7:pai8-10]{牌9}即使是坎张立直,也能得到更高的打点,因此不切\inlinemahjong{6s},而是切\inlinemahjong{9m}。这样若自摸\inlinemahjong{46s}就会成为50符2飜立直,比良形立直nomi略有优势。不过在中盘以后,为了优先速度,会切\inlinemahjong{6s}。

\par\bigskip
\textbf{50符坎张立直1宝牌 > 立直nomi的两面 > 40符坎张立直1宝牌}
\par\bigskip

即使有3个以上对子,与\hyperref[lec7:pai5-7]{牌7}不同,对于\hyperref[lec7:pai8-10]{牌10}那样能够看出三暗刻希望的手牌,也会保留三暗刻的可能性并固定对子。

\section{雀头固定之间的比较}
\begin{figure}[h]
    \caption{雀头固定之间的比较}
    \label{lec7:pai11-13}
    \tekiri{11}{244789m577p45s555z}{4z}{2m}{注意\inlinemahjong{4p}的改良}
    \par\bigskip
    \tekiri{12}{334568m45789p668s}{4z}{8m}{摸到里目也能让手牌前进}
    \par\bigskip
    \tekiri{13}{34568m68p45789s33z}{4z}{8p}{\hyperref[lec7:pai11-13]{牌12}不同。打\inlinemahjong{8p}留下连续形}
\end{figure}

雀头固定的基本原则,是将包括后续变化在内不容易做成面子的塔子固定下来。
例如\hyperref[lec7:pai11-13]{牌11}中,比较\inlinemahjong{244m}与\inlinemahjong{577p},因为\inlinemahjong{244m}更难变成两面,因此切\inlinemahjong{2m}。

\hyperref[lec7:pai11-13]{牌12}中,则是比较切\inlinemahjong{8m}与切\inlinemahjong{8s}。只要打\inlinemahjong{8m},即使摸到\inlinemahjong{7m}导致错过聴牌,也能通过打\inlinemahjong{8s}来保留振听的三面进张。
看似类似的\hyperref[lec7:pai11-13]{牌13}则要保留连续形,因此切\inlinemahjong{8p}。

\section{复合塔子的进阶形}
\begin{figure}[h]
    \caption{存在烟囱形时}
    \label{lec7:pai14-16}
    \tekiri{14}{345556m66p223445s}{4z}{2s}{留下包含34555的烟囱形的三枚搭子}
    \par\bigskip
    \tekiri{15}{345556m67p223445s}{4z}{6m}{打\inlinemahjong{6m}和\inlinemahjong{4s}的进张数相同}
    \par\bigskip
    \tekiri{16}{344555m67p223445s}{4z}{4s}{考虑到自摸\inlinemahjong{4m}的可能性打\inlinemahjong{4s}更好}
\end{figure}

我们将来探讨一些较为复杂的复合塔子。首先是类似34555的“烟囱”形。\hyperref[lec7:pai14-16]{牌14}中,由于345556这一形状的威力,相较于切\inlinemahjong{5m}或\inlinemahjong{6m},切\inlinemahjong{2s}能够获得更大的进张数量。

而\hyperref[lec7:pai14-16]{牌15}中,若切\inlinemahjong{6m}来固定万子,进张数量并不会减少,因此考虑到一盃口的可能性,切\inlinemahjong{6m}更为有利。

不过\hyperref[lec7:pai14-16]{牌16}里,如果在烟囱形之外还具备一杯口形时,切\inlinemahjong{4s}进张数量更多。

\par\bigskip\noindent\ignorespaces
\begin{figure}[h]
    \caption{存在33577形时}
    \label{lec7:pai17}
    \tekiri{17}{33577m345666p357s}{4z}{7s}{留下345三色}
\end{figure}
另一个常见的复合形是像33577这样的五张复合塔子,一旦切去其中任意一张,都将大幅减少进张数量。
\hyperref[lec7:pai17]{牌17}中,比起切\inlinemahjong{3m}或\inlinemahjong{7m},切\inlinemahjong{3s}或\inlinemahjong{7s}能够获得更多进张。考虑到345三色,最终选择切\inlinemahjong{7s}。



\begin{figure}[h]
    \caption{存在13556形时}
    \label{lec7:pai18}
    \tekiri{18}{13556m123789p556s}{1p}{6s}{偏重万子的进张}
\end{figure}
对\hyperref[lec7:pai18]{牌18},让索子做雀头固定,在进张数量和最终形状上都更为有利。这也是一个常见的形状。

\begin{figure}[h]
    \caption{存在复数的坎张对子时}
    \label{lec7:pai19}
    \tekiri{19}{33557m123789p557s}{6m}{7s}{因为宝牌是\inlinemahjong{6m}并且有一杯口可能所以打\inlinemahjong{7s}}
\end{figure}\hyperref[lec7:pai19]{牌19}是带有多个坎张对子的手牌。像3355这类含“跳对子”的复合塔子,如果固定塔子并切\inlinemahjong{5m},会错过摸\inlinemahjong{4m}时一杯口听牌的机会。切\inlinemahjong{7m}与切\inlinemahjong{7s}在进张数量上相同。

\begin{figure}[h]
    \caption{存在两翼形时}
    \label{lec7:pai20-21}
    \tekiri{20}{45m3456p23345667s}{4p}{6p}{如果为了断幺切\inlinemahjong{2s}则没有留\inlinemahjong{3p}的意义}
    \par\bigskip
    \tekiri{21}{33445667m789p778s}{9p}{7s}{万子已经提供了雀头,所以索子固定两面}
\end{figure}

有些复合塔子会包含雀头。

例如,从类似334566(除此之外没有雀头)的两翼形\footnote{指223455这样,有时还会直接指代在此两侧再附加一枚或两枚的形状},再往前推进两手后,便会出现13345668或33445667这种形状。虽然它们并非高频出现,但如果能将这类手形视为两翼形的进阶形来理解,就会在实际对局中更容易判断。
\hyperref[lec7:pai20-21]{牌20}的23345667是两翼形的典型进阶形。其特征在于,无论是摸14还是58,都能形成2个面子 + 雀头。在这手牌中,即使摸\inlinemahjong{1s}后失去断幺,只要还能保留高目三色,就更容易达到满贯以上(优先考虑满贯以上的进张),因此与其切\inlinemahjong{3p},选择切\inlinemahjong{6p}更好。
\hyperref[lec7:pai20-21]{牌21}同样是两翼形的进阶形
在这副牌里,如果摸\inlinemahjong{23458m},就能让万子变成「2个面子 + 雀头」。因此,通过切\inlinemahjong{7s}把索子固定成两面,能够获得最多的进张数量。

\par\bigskip
\begin{figure}[h]
    \caption{包含烟囱形的手牌}
    \label{lec7:pai22-24}
    \tekiri{22}{345556m12399p577s}{4z}{5m}{尽量最终形成两面听牌}
    \par\bigskip
    \tekiri{23}{345556m12399p077s}{4z}{7s}{为了把\inlinemahjong{0s}留到最后}
    \par\bigskip
    \tekiri{24}{345556m11199p577s}{4z}{6m}{保留摸到\inlinemahjong{6s}时的良形听牌和三暗刻的可能性}
\end{figure}
最后,举几个关于“当下进张数量”与“打点”如何取舍的例子。其中也包含之前谈到的“烟囱形”。
\hyperref[lec7:pai22-24]{牌22}中,
为了扩大进张面应该切\inlinemahjong{7s},但为了优先考虑听牌时的待牌强度,这里选择切\inlinemahjong{5m}。

\hyperref[lec7:pai22-24]{牌23}这里,则是打算将\inlinemahjong{0s}用到底,因此选择切\inlinemahjong{7s}。

至于\hyperref[lec7:pai22-24]{牌24},
则是通过切\inlinemahjong{6m}保留在摸\inlinemahjong{6s}时的良形听牌与三暗刻的可能性。


\vspace*{\fill}
\begin{tcolorbox}[
    title={福地的碎碎念}, fonttitle=\bfseries\Large
]
我认为这次的课程内容相当基础。这里所说的“基础”并不是“简单”的意思,而是指能在根本上支撑整套打法的重要内容。如果有人觉得这些内容并不是理所当然,那么反复多读几遍、把里面的内容彻底掌握之后,恐怕对牌效率的理解会有明显提升。这也是本书相当高质量的部分之一。

至于那些容易忽略的役种可能性,我觉得对成绩的影响并不大,不\hyperref[lec7:pai11-13]{牌11}\hyperref[lec7:pai11-13]{牌13}里的做法,最好是能做到闭着眼都毫无疑问地处理才好。

不过呢,这也可以算是对我自己写的书的一个吐槽:\hyperref[lec7:pai5-7]{牌5}那样的讨论终究是纸上谈兵。因为在实际对局中,一旦进入中盘,就一定会出现\inlinemahjong{12m}再场上的枚数差异,人们会根据那些实际切出去的张数来决定打法。毕竟没有完全没有边张的场况。

另外,我跟ネマタ在大方向上也有很大不同。还是\hyperref[lec7:pai5-7]{牌5}来说,我经常会切\inlinemahjong{4p},原因是我不想在中盘之后,留下456这类牌作为“多余的手牌”。我宁愿最后剩下2378,从危险度上来说更好。
\hyperref[lec7:pai3]{牌3}也一样,我会因为讨厌听牌时多出的那张\inlinemahjong{6p},而倾向于保留\inlinemahjong{4z},切\inlinemahjong{6p}。

我和ネマタ在主要打牌的场合不同。ネマタ主要打网络麻将,所以往往走“先制立直”或“弃和”的两极化道路;而我是在歌舞伎町的自由雀庄,人们通常不会轻易弃和,而是激烈对攻。这样的场合差异,也会反映在组牌思路上。

至于听牌时把进张往外扩以占优势,还是尽量让听牌时多余的牌危险度更低来降低风险,究竟哪个更有利,目前并没有定论。这里就只能看你更想相信哪种思路了。

\end{tcolorbox}